\chapter{Conclusions and Future Work}
\label{ch:conclusion}

This work set out to evaluate the applicability of capacitive proximity sensors in the domain of smart environments. Technological advances in the last decades have enabled a large number of different devices and applications that aim at making our environments more aware and provide a variety of different services. When designing such applications the developers have to choose from a selection of sensor devices that provide environmental measurements or activity data. Capacitive proximity sensors have some unique properties that are beneficial, particularly in scenarios that require unobtrusiveness. To support the sensor selection process I wanted to provide a tool that allows to benchmark different sensor technologies against one another given a set of features and potential applications. This enables us to establish a set of application domains that are particularly well suited for capacitive proximity sensors. Any of those domains has a specific set of challenges regarding the required sensor technology, layout of the systems, and processing that can be tackled using a variety of new or improved methods. In order to establish a classification of capacitive proximity sensors with regards to competing sensor technologies and provide a set of implementation guidelines, a number of different prototypes should be developed and evaluated.

I have developed a benchmarking model for sensor technologies in smart environments that is adapted from previously presented benchmarks for pervasive systems, though adapted towards sensor systems only and having some specific metrics for smart environments. Based on the specific characteristics of the sensors, different ratings are given that act as weighting factor in calculating the final benchmarking score. The method was evaluated by measuring the popularity of four different sensor technologies, used in three different application domains using two scientific databases. Additionally, some extensions are provided that account for the tendency to prefer average ratings and normalize for different overall ratings. Thus, I was able to establish the method and identify four different application domains that are suited for capacitive proximity sensors out of a set of seven. 

Indoor localization, smart appliances, physiological sensing and gesture interaction are use cases that have been realized using capacitive proximity sensors in the past. However, there are still a number of challenges and potential novel approaches to improve the data processing. In situation where the number of sensors is constrained or distributed over a large area, gesture by example can be used to improve gestural interaction, or grid layouts to cover whole rooms with a limited number of sensors that are only attached on the sides. Abstraction allows to create analytical models of the capacitance measurements in different scenarios. A single body model is sufficient to precisely determine occupation and posture on a bed, while multi-body models on a chair allow a fine-grained calculation of the posture. Heterogeneous sensor systems can augment the potential range of applications or aid in alleviating limitations of a single sensor type. I presented a method to detect the intention to interact based on limb posture that allows to integrate capacitive interaction devices in areas, where unintended gestures can occur. A combined capacitive, acoustic system that enables detecting different types of touch, allows increasing the versatility of invisible capacitive interaction devices by providing a more reliable detection of selection events and additional input channels. Image-based processing of capacitive proximity sensor data has been used in the past primarily for shunt mode systems. I presented a method for application on loading mode systems that can be applied more flexibly. It allows tracking the position and orientation of two arms in proximity of a table, enabling to use this sensor technology with the plethora of available image processing methods. Finally, there are two methods for physiological sensing that detect respiratory rate based on chest movement and sleep-phases based on whole-body movement. Particularly operations in frequency space are able to capture periodic signals generated by muscle movements of the body.

In order to evaluate those methods and gather sufficient information to classify capacitive proximity sensors, six different prototypes are presented in detail. Each prototype can be associated to one or more of the specified application domains. The development of the prototypes is outlined, including the rationale of the selected electrode layouts and the different processing methods chosen for the specific purpose. I describe how specific design challenges are solved, in order to build a knowledge base for specifying and detailing the specific capabilities of the technology. For all prototypes some form of functional or qualitative evaluation has been performed, with setup and results described in the respective sections. For all prototypes it was shown that they perform well, given the specific application scenario. However, a limitation of these evaluations is that it is typically difficult to directly compare results to other sensing technologies or similar prototypes, as the data sets and sample users vary across different setups. Three additional prototypes based on capacitive proximity sensors that have been created in other collaborative projects are briefly introduced.
 
The combined knowledge gathered in designing the prototypes and evaluating their functionality allows me to build a base for classifying the technology and comparing it to other sensor technologies. Capacitive proximity sensors are uniquely affected by conductive objects and disturbing electric fields in the vicinity. However, they do not require a line of sight and optical and acoustical properties of the environment do not matter. The detection range is lower than most other sensing technologies, particularly radio frequency systems, however, the resolution compares favorable in low distances. The processing complexity is considered simple, due to the low number of sensor values that can be applied in typical applications, despite the high dynamic range a single sensor can achieve. However, in some instances more complex processing methods can be beneficial. For example the SVM classification methods used in the multi-body models. Capacitive proximity sensors can be applied completely invisible below non-conductive materials, which is also true for radio frequency sensing but not without limitations for the other systems. Materials that are applied above the different prototypes include a variety of different plastics, cloth, leather, or wood, in thickness of up to 5 cm.

A major limitation that was apparent in the development of several prototypes is the proneness to certain types of environmental influences. Induced external and internal electric fields and changing parameters within the environment can cause a disturbance of the sensor signal, e.g. periodic waves on the power supply lines that were resulting in unusable sensor readings. The vast majority of those can be mitigated by proper system design. However, in areas that have an expected permanent influence of electric fields some other sensing technology might be better suited. The physical range of the sensors is fairly low, e.g. compared to cameras. While many other sensing technologies are also operating in the electromagnetic field, the capacitance is an aggregate value of a local electric field and is thus limited according to the laws of physics. This also applies to distinguishing objects at a distance, even though constrained low-distance systems in industrial applications can be used for material detection, relying on the distinct dielectric properties. 

Based on the collected knowledge in identifying suitable application domains, creating novel methods for data processing and building a variety of different prototypes, I was able to create a set of guidelines that support the design process, starting from selection of a suitable technology, choosing a suitable hardware layout and discuss suitable processing methods for given applications. These guidelines are intended to support other researchers interested in developing systems based on capacitive proximity sensors that may be new to this technology.

I conclude this section by discussing some potential limitations of the work presented. The developed benchmarking method is based on a set of criteria that have to be established by the person applying it and thus may be considered subjective. Given the plethora of potential sensor variations, algorithms and applications it is difficult or even impossible to find a set of features and ratings that fits to all. It should be avoided to use this benchmarking in a very generic fashion and instead apply to a specific set of applications. In some instances the frequency of keywords in the technology popularity evaluation is low, making it difficult to estimate the significance of the technology-application combination. The method has been only used on smart environment systems so far. Therefore, no estimate can be given if this benchmarking is also applicable to other areas. And even of this domain only a limited subset of application domains has been investigated. In order to identify potential novel use cases the method would have to be evaluated on a larger scale.

\section{Future Work}
I will conclude this work by stating a few future directions of research for capacitive proximity sensors in smart environments.

A further exploration of the potential capacitive proximity sensor applications in car environments is desirable. Modern vehicles feature of plethora of sensors and actuators and can be considered a smart environment of their own regard. Apart from the presented Active Armrest and GestDisp there are additional applications both in human-machine interaction, where additional surfaces can be equipped with proximity sensors or in human sensing, where the sensors can be integrated into the seat to measure a variety of different properties from occupancy, and posture to more detailed information, such as identity and level of alertness. Another area of research are remote measurements of phyisiological parameters, similar to the system proposed by MacLachlan that sensed the breathing rate over a distance \cite{MacLachlan2004}. Similarly, capacitive proximity sensors could be used to measure the heart rate or provide a full ECG without physically connection to the person. Finally, an investigation of irregular interactive surfaces is an interesting research question. This may include curved surfaces that can be designed in a way to more naturally follow the movements of the human arm.

The presented benchmarking model can and should be expanded or adapted to include other domains outside of smart environments. This may require the omission or addition of different features. These areas should support and use different types of sensor technologies. One candidate area might be robotics. As previously mentioned even in the domain of smart environments there are number of application domains that have not been included in the benchmarking process. This could be performed to identify additional use cases that benefit from capacitive proximity sensors.



