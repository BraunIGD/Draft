\chapter{Conclusions and Future Work}
\label{ch:conclusion}

On the previous pages we have evaluated capacitive proximity sensing technology in smart environments. After presenting a basic overview of the technological principles we have identified and detailed several application domains that are relevant in this context. Using a set of self-designed prototypes and various devices created by other researchers, we showed how to apply capacitive proximity sensing to create viable systems in these domains. Afterwards, we discussed the benefits and limitations of the technology using a set of criteria to compare it to other widely used sensor systems. In the end we have given some guidelines that should aid interested parties in deciding for or against the technology.

Capacitive proximity sensors already have shown their great potential in various applications. However, it is foreseeable that the story does not end here. There are some trends that should allow capacitive proximity sensors to play an increased role in the future. A necessary step is a smarter, more adaptive sensor that is able to react independently and intelligently, with regards to changes in the environment. This will allow the usage in self-organizing sensor networks that are becoming more prevalent in modern smart environments.

A particular driver of innovation in the past years is the increased number of rapid prototyping systems, such as 3D and circuit printers that enable researchers and developers to quickly test out novel concepts. Conductive ink allows turning arbitrary surfaces into electrodes. Combining these techniques and attaching suitable capacitive proximity sensors let us create novel smart appliances with hidden interactive parts that enable interaction beyond the scope of what has been presented so far. 
The technology will always be bound by the laws of physics, restricting the potential applications to a specific spatial range. However, this should not discourage interested parties from experimenting with this technology, using the available toolkits or custom designs.

\section{Future Work}

\subsection{Technology}
Capacitive sensors in cars

Remote ECG

Sensor fusion

Irregular interactive surfaces

\subsection{Benchmarking model}
Expansion to other application domains

Identify further applications



