\chapter{Evaluating capacitive proximity sensors in smart environments}
\label{ch:eval}
In the previous sections I have presented background information on capacitive proximity sensors and various prototypes of this technology in different application domains within smart environments. These overview included both an analysis of existing prototypes from literature, as well as a collection of own prototypes that were developed in the last years. With this as basis, it is now possible to perform a meta-analysis to identify common patterns and identify benefits and limitations of the technology within smart environments. The section is distinguished into four different parts. First, I will compare capacitive proximity sensing to other sensing technologies that are commonly used in smart environments. Afterwards, limitations and benefits of the technology are identified and discussed in detail. I will conclude with a set of guidelines that may help other researchers when designing smart environment applications based on capacitive proximity sensors. This section is an extended form of the discussion found of a working paper to be released in a journal soon  \cite{braun2014capjaise}.
\section{Comparison to other sensing technologies}
\label{ch:sens_compare}
In order to properly place capacitive proximity sensing in the domain of smart environments it is necessary to include a comparison to other sensing technologies. I have given an overview of common technologies in Section \ref{ch:rel_sensor_tech}. The results are briefly recapitulated, in order to keep this section self-contained. 

\subsection{Overview of sensing technologies in smart environments}
% Table generated by Excel2LaTeX from sheet 'Tabelle1'
\begin{table}[htbp]
  \centering
  \footnotesize
  \caption{Qualitative comparison between capacitive proximity sensors and other senor technologies}
    \begin{tabularx}{\linewidth}{Xp{4cm}XXXX}
    \toprule
    \textbf{Name} & \textbf{Application Domains} & \textbf{Environmental Influences} & \textbf{Detection Range} & \textbf{Processing Complexity} & \textbf{Unobtrusiveness} \\
    \midrule
    \textbf{Capacitive proximity sensing} & indoor localization, smart appliances, physiological sensing, gestural interaction & electric fields, conductive objects & near distance   (< 100cm) & Few high dynamic range data sources  & invisible integration possible \\ \addlinespace
    \textbf{Capacitive touch sensing} & smart appliances, physiological sensing, gestural interaction & electric fields, conductive objects & touch  & Few binary sensors & thin cover above electrodes \\ \addlinespace
    \textbf{RGB cameras } & indoor localization, smart appliances, physiological sensing, gestural interaction & occlusion, external lights & far distance     (> 10m) & Complex image processing based on resolution & pinhole lenses \\ \addlinespace
    \textbf{Infrared cameras} & indoor localization, physiological sensing, gestural interaction & occlusion, external infrared light & medium distance (< 5m) & Complex image processing based on resolution & infrared source and camera \\ \addlinespace
    \textbf{Ultrasound sensing} & indoor localization, smart appliances, gestural interaction & acoustic occlusion, absorbing materials & medium distance (< 5m) & Few low dynamic range data sources & emitter and senders with exposed pinhole speaker, microphone \\ \addlinespace
    \textbf{Microphone arrays} & indoor localization, smart appliances, physiological sensing & environmental noise, absorbing materials & medium distance (< 5m) & Very high dynamic range data sources & exposed pinhole microphones \\ \addlinespace
    \textbf{Radiofrequency sensing} & indoor localization, smart appliances, gestural interaction & other RF devices & far distance     (> 10m) & Few low dynamic range data sources & hidden emitters and senders possible \\
    \bottomrule
    \end{tabularx}%
  \label{tab:eval_sensortech}%
\end{table}%
 
Capacitive touch sensing, as opposed to capacitive proximity sensing relies on an surface being touched instead of an object being in proximity and is ubiquitous in touch screen applications. The technology is actually very similar to capacitive proximity sensing, using the same measurement principle and similar electrode considerations. The electrodes are typically below a protective layer of non-conductive layer. However, they are tuned to detect very close proximity only. This restriction greatly increases the potential resolution in close distances.

RGB cameras are a class of image sensors operating in the same frequency domain as the human eye. They are capable of easily distinguishing different colors. The most common technologies nowadays are CMOS and CCD sensors that use specific filters to distinguish different wavelengths. They vary strongly in resolution and achievable frames per second. The major disadvantage is that they rely on external light sources and are therefore not well suited in dark environments.

Infrared cameras operate in near light frequencies that are invisible to the human eye. This allows for application in dark environments, as it is possible to project infrared light into the scene without disturbing the user. This is primarily used for illumination, but may also use particular patterns, the reflections of which can be analyzed. It uses the same type of sensors as RGB camera, while having different filters, resulting in a single channel image.

Ultrasound sensing uses low frequency range mechanical waves just above the audible limit of human hearing. The waves propagate similar to sound signals and one can perform reflection measurements or time-of-flight methods. The emitters and receivers can be easily modified to use different frequencies, thus enabling to use a number of sensors in a single setup or reduce the amount of required receivers. The systems detect most materials unless they have a strong dampening effect on mechanical waves in that frequency range.

Microphone arrays detect signals in the range of human hearing, and thus work with audible signals, such as human speech. They are typically based on piezoelectric sensors that transfer the vibration of a crystal to an electric current. The frequency ranges are associated to human hearing, thus covering at least the range between 10 Hz and 20 kHz, but often also exceeding this. Common sampling rates include 48 kHz or 96 kHz. 

Radiofrequency (RF) sensing uses signals in a range between several hundred kHz up to 5GHz, typically used for wireless communication. Commonly the signal strength or time of flight is used to gather information about the environment. This technology is very popular due to the prevalence of WiFi equipment in most modern environments, allowing to create applications based on existing hardware that can be additionally used for communication between different nodes. Other popular radio-technologies include ZigBee and Bluetooth.

A short overview can be found in Table \ref{tab:eval_sensortech}. Going back to the benchmarking model, I have included a comparison of suitable application domains, environmental influences, detection range, processing complexity and unobtrusiveness of the technology. This subset of features is a selection of the set presented in Section \ref{ch:eval} and more specifically Table \ref{tab:bench_cap_feat_weights}, but includes some more details on the particular aspects of the different sensing technologies related to the given feature.

\subsection{Classification of capacitive proximity sensors}
Most technologies are capable of supporting multiple application domains. Some non-intuitive examples include WiSee that enables whole-body gestural interaction using WiFi signals \cite{pu2013whole} or TapSense that uses a single microphone to enable gesture interfaces on various surfaces \cite{harrison2011tapsense}. 

Capacitive sensors are disturbed by conductive objects and electric fields, whereas cameras struggle with occlusion and additional light sources. A line of sight is required, even though some materials, such as glass have different opacity properties for the different wavelengths, in this case blocking infrared light. Some plastics however will block visible light, while infrared passes through. Acoustic sensors are prone to dampening materials and environmental noise interfering with the signal. Similar to cameras a multitude of sensors can be used to filter out environmental noise or locate the source of a specific signal. Radiofrequency signals usually propagate well through most materials. There is a dampening effect, particularly for high-frequency ranges and weak signals, obvious in a reduced reception indoors. Additionally, conductive objects may create a shield, significantly reduce the measurable effect of outside external fields. In the common WiFi frequency bands of 2.4 GHz and 5 GHz a multitude of devices are communication, leading to a limitation of potential bandwidth and potential interference. This has to be considered when designing applications for this sensor category. 

Regarding the detection range mostly applications in buildings are considered. Here it is not usually necessary to measure properties of objects that are very far away, e.g. more than 10 m. In this criterion deteriorating sensor readings at a certain distance also have to be considered. One example is radiofrequency sensing that has the best range, but considerable limitations when being applied over short distances. The fast signal propagation makes it difficult to measure the time-of-flight without specialized hardware. Visible light cameras have to consider the required distance in design of the optics. However, it is possible to use adaptive optics to enable a higher range in suitable distances. Infrared cameras have deteriorating quality, particularly when using static pattern projection, limiting their range to a fairly short interval. Some additions include using stereo infrared cameras, or adaptive projection patterns. Ultrasound also gets less precise when exceeding a certain distance, due to the wave properties and single measurements. However, the slower propagation makes it suitable for applications at close distances. Capacitive proximity sensors struggle to measure objects that are far away from the sensing electrode, as their influence on the electric field might be below the achievable resolution. Specific layouts of electrodes allow very precise measurements at small distances.

It is not trivial to find a good measure for the processing complexity associated to different sensing technologies. An approach is to take the dynamic range of a sensor and the number of sensors typically required. Dynamic range is the difference between the smallest detectable value and the largest detectable value. Microphones have a high dynamic range measuring over a larger frequency scale, whereas touch sensors only have two different states. Ultrasound usually measures simple amplitudes in time-of-flight systems and thus does not require complex processing. Image processing in both visible and infrared domains requires complex operations on large data arrays and is difficult to integrate in simple embedded systems. This discrepancy has a distinct influence on the number of sensor systems required in a system. Whereas, a single touch sensor is required on each position that is supposed to be interactive, using a single camera can account for watching a larger area, identifying numerous activities, or detecting fine activity information at a short distance. Capacitive proximity sensors have a high dynamic range for single sensors. However, as the electrodes need a certain size to enable detection at a higher distance, the number of sensors in a single setup is limited. Thus, unless complex processing methods are used the complexity can be considered low.

Regarding unobtrusiveness, capacitive sensors and RF sensors can be applied completely invisible, below any non-conductive material. Cameras, microphones and ultrasound need a direct line-of-sight or mechanical connection to the measured property and are thus more difficult to hide. As previously mentioned, certain plastics are transparent for infrared light and glass can be designed to provide varying transparency levels for different directions. Microphones can pick up any environmental noise and can be hidden below a surface that transmits mechanical waves well. One example is the CapTap prototype (\ref{ch:prot_captap}) that uses a microphone to detect acoustic events on a surface. Of the presented technologies capacitive proximity senors are the least obtrusive. The layer between object to be detected and electrode can be very thick, as shown in the Smart Bed system (\ref{ch:prot_smartbed}), having a thick mattress layer.

The actual performance of a specific sensor is difficult to assess from this limited selection of criteria and often is highly depending on the selected application. Having worked with a number of different prototypes and analyzing the advantages and disadvantages of the different technologies it is possible to build a set of benefits and limitations of capacitive proximity that will be presented in the following two sections.

\clearpage
\section{Benefits}
\begin{table}[htbp]
  \centering
  \caption{Overview of capacitive proximity sensing benefits}
    \begin{tabular}{p{4cm}p{6cm}}
    \toprule
    \textbf{Name} & \textbf{Examples} \\
    \midrule
    \textbf{Versatility} & Flexible electrode design, scalability, different sensing methods \\
    \textbf{Unobtrusiveness} & Invisible application, non-disturbing frequency range \\
    \textbf{Processing Complexity} & Small number of sensors, variable dynamic range \\
    \bottomrule
    \end{tabular}%
  \label{tab:cap_benefits}%
\end{table}%

Given the information collected in the previous section we can now talk about the specific benefits of capacitive proximity sensing. We are using three different groups for categorization, namely versatility, unobtrusiveness and processing complexity. Some examples within these groups are shown in Table \ref{tab:cap_benefits}. In the following section I will discuss those groups in detail.
\subsection{Versatility}
A main benefit of capacitive proximity sensors is the versatility in which they can be applied. With flexible choice of electrode materials, size and geometry it is possible to create highly individual applications. Example electrodes include transparent metal oxide layers, woven conductive thread, copper wires, PCB boards or simple aluminum foil. 

Additionally, the sensor systems are also highly scalable. By choosing appropriate voltages and frequencies it is possible to add a high number of sensors to a single object. Using smart measurement windows and different multiplexing methods, sensors can be placed close together and electrodes may act as both sender and receiver.
The different sensing methods presented - loading mode, shunt mode and transmit mode enable a variety of different sensing patterns. The human body can be used as both sender and receiver and smart electrode layouts allow using low-powered processors. 

In conclusion, it is possible to add capacitive sensing to most everyday objects to enable different forms of interaction, create natural interfaces and smart objects. Our prototypes are using different electrode materials, flexible or solid electrodes, conductive thread, wires, shielded or non-shielded layouts.
 
\subsection{Unobtrusiveness}
Electric fields are not usually perceived by persons, unless they are of exceptional strength. Furthermore, they propagate through many materials that are typically present in our environment, including most plastics, wood or tile. This allows us to invisibly apply capacitive proximity sensors without a strong effect on the measurement. Application below several centimeters of covering is possible, if the electrodes are designed properly for large distance sensing.

The frequency range in which the sensors are operating is usually not in an interval that disturbs other electronic systems. Thus it is feasible to use capacitive sensing even in environments, where non-disturbance is a main requirement.  Additionally the used frequencies are not considered to be biologically active, and good results can be achieved using small currents. 

It is possible to equip most conductive objects directly with capacitive proximity sensors and hide them below non-conductive objects with minimal spatial requirements. Our Smart Bed and Active Armrest prototypes are using sensor sets that are completely invisible from the outside and communicate wireless to a PC only using a power supply. .
 
\subsection{Processing Complexity}
An appropriate analogy to capacitive proximity sensors is a single photo diode. As opposed to a light intensity we are measuring capacitance. While the information we can gain from such a measurement is limited, the processing required to analyze the signal is also low. Performing signal analysis on an array of 24 capacitive sensors, as in the CapTap prototype (Section \ref{ch:prot_captap}) is comparable to processing the image of a 6x4 pixel camera. Therefore, it is easy to create highly integrated systems with low-power devices for performing any subsequent data analysis. 

While it is possible and in many cases beneficial to use complex data processing algorithms for object detection, it is in many cases possible to achieve a similar result with less complex methods. In many applications it is even viable to opt for a quantized capacitance measurement. In the case of a touch sensor a single binary measure is sufficient. However, it is also possible to select various different levels and reduce the dynamic range to an easily computable value that is 4 or 8 Bit long. Depending on the chosen algorithm this dynamic range reduction can occur either in pre-processing or high-level processing. 

With the exception of the Capacitive Chair (Section \ref{ch:prot_capchair}) and the CapTap that make extensive use of frequency domain operations, our prototypes are using simple data processing methods that can be easily applied on embedded systems. One of the presented examples is the weighted average algorithm for object detection. Regarding model-based data processing, even very simple cylindrical models, such as the one used for the Smart Bed (Section \ref{ch:prot_smartbed}), are capable to reliably predict numerous postures that are relevant in real world applications. In general, the low requirements for data pre-processing, allows dedicating more resources to high-level data processing algorithms if the specific application is resource constrained. 

The OpenCapSense toolkit that is the base for most of our prototypes has a fairly powerful micro controller that is able to implement all of the processing steps - thus enabling highly integrated, low-power capacitive proximity sensing prototypes that can be used in smart environment applications.

\clearpage
\section{Limitations}
Despite the potential benefits that have been outlined in the previous section, capacitive proximity sensors have a number of limitations that may hinder several applications. 
Similar to the benefits I am putting these into three different groups that are outlined in Table \ref{tab:cap_limitations}. Environmental influences have been briefly mentioned throughout this work and have to be considered in any setup. The physical range limits the object detection to a low distance from the sensing electrode. The objection detection similarly suffers from the physical limitations of electric field sensing. These three groups will be detailed in the following paragraphs.

\begin{table}[htbp]
  \centering
  \caption{Overview of capacitive proximity sensing limitations}
    \begin{tabular}{p{4cm}p{6cm}}
    \toprule
    \textbf{Name} & \textbf{Examples} \\
    \midrule
    \textbf{Environmental influence} & Static electric fields, dynamic electric fields, temperature, humidity, conductive objects \\
    \textbf{Physical range} & Small differences in capacitance, reduction due to influences, physical limitations \\
    \textbf{Object detection} & Small number of data points, a priori knowledge \\
    \bottomrule
    \end{tabular}%
  \label{tab:cap_limitations}%
\end{table}%

\subsection{Environmental Influence}
One of the main limitations of capacitive proximity sensors is their sensitivity towards environmental influences. Any factor that modifies an electric field will also affect the measurement of a capacitive sensor. The current environmental parameters, like temperature and humidity are having a considerable effect on the medium in which the electric field propagates. However, those changes are usually over a longer period of time and can be compensated using a factor for drift, as described in the previous sections about noise reduction. While the frequency range in which the sensors operate is usually not in an interval that disturbs other electronic systems, there are some electromagnetic sources in the environment. The most common  are power supplies (frequency range of 50-60 Hz) terrestrial radio (frequency range of 150 kHz to 100 MHz), mobile phone communication (870-970 MHz and 1700-1900 MHz), as well as WiFi (2,4GHz and 5GHz). While in theory there is some potential interference with the terrestrial radio spectrum, the effect is fairly small and can be covered by choosing intermediate frequency intervals. Thus it is feasible to use capacitive sensing even in environments, where non-disturbance is a main requirement.

The main source of disturbance occurring in our evaluations, was the influence of very close electromagnetic sources and disturbing signals within the power grid. One example for a disturbing source was a plasma TV installed in our Living Lab. When turned on the CapFloor prototype (\ref{ch:prot_capfloor}) suffered from severe noise, preventing any successful tracking of persons. The culprit are high frequency internal power supplies that drive the plasma cells. Newer TVs are using different internal electronics that significantly reduce the emitted electric field. Another disturbing factor can be faulty power supplies that are attached to the grid and produce frequency noise. As this noise typically operates in a limited frequency range, it is possible to use frequency domain filtering that attenuates these bands. Thus, it should be attempted to apply capacitive proximity sensing in an environment that adheres to all relevant standards regarding electromagnetic compliance. Additionally, the systems should be designed in a way that reduces potential external influences, e.g. by using internal power supplies that are stabilized and applying shielding where necessary. 
 
An additional issue might arise when placing sensors close to each other. The created electric fields may disturb the measurement if some electrodes are charged and create fields to adjacent electrodes while those are discharged for measurement. This is particularly challenging, when a larger number of sensors is involved. Here, this is the case for the CapFloor (\ref{ch:prot_capfloor}) and CapTap (\ref{ch:prot_captap}) prototypes. In the case of CapTap specific charge-discharge cycles for the electrodes are used that ensure that adjacent sensors are not interfering with each other. Similarly the CapFloor attempts to control the sensors in a way that prevents interference, taking into account the geometric layout and the position of the electrodes within the area. Both methods are a form of multiplexing that is based on time (TDMA). Other options include a variety of different multiplexing methods, such as the previously mentioned frequency method of Honeyfish (\ref{ch:prot_otherprot}). Code division multiplex is a last class of multiplexing methods that relies on the modification of the signal. This was employed by the presented capacitive sensor of Rob MacLachlan \cite{MacLachlan2004}. It can result in very precise measurements, but requires specialized hardware. Thus, it was not used in any of the developed prototypes.

Another major challenge are conductive objects that are permanently placed in the immediate sensing environment. It is difficult to distinguish the object we want to detect from a disturbing object, if their influence on the electric field is similar. Long term data analysis may help in performing a successful detection. The CapTap prototype, as a regular living room table, should not fail when disturbing objects are present, such as glasses full of water (\ref{ch:prot_captap}). This can be achieved by performing a feature analysis that detects the typical response of the disturbing object and performs a recalibration that takes the object into account. 

The CapFloor prototype is affected by environmental influences the most, given the small size of the electrodes relative to the interaction area and the changing environment on top of the floor (\ref{ch:prot_capfloor}). Strong noise reduction algorithms and drift compensation are used to create a more stable result that however causes a reduction of the detection range.
\subsection{Physical Range}
\begin{figure}[ht]
\centering
\includegraphics[width=0.8\textwidth]{images/limit_distance}
\caption{Reduced angular resolution on smaller, distant objects}
\label{fig:disc_ang_resolution}
\end{figure}

The physical range of the generated electric field is one of the main limiting factors of capacitive proximity sensing. In order to detect objects that are further away,  the electric field strength has to be increased sufficiently. This is easier for larger electrodes, as its potential capacitance is higher. However, this also leads to distant objects having an ever smaller influence on the overall capacitance. In this case we need more precise measurement circuits and longer measurement times to improve the signal-to-noise ratio. Additionally, looking at smaller objects the angular resolution will decrease as shown in Figure \ref{fig:disc_ang_resolution}. This makes it more difficult to get a precise localization as the immanent noise leads to an angular error. While this can be compensated using more sensors, the far distance would require us to use large electrodes that have to be placed further apart resulting in a huge area that would have to be equipped with sensor electrodes. The Capacitive Chair prototype uses large electrodes that allow us to detect the presence of a person at distances of around 50-60 cm (\ref{ch:prot_capchair}). It is designed to detect whole body parts and thus does not need any fine resolution. The Active Armrest on the other hand has a finger interaction array comprised of small electrodes that have a detection limit of approximately 15 cm (\ref{ch:prot_armrest}). Yet, it is able to detect fine movements and precise locations of the fingers within this range.

In general, the achievable resolution is not comparable to vision based system and has to be taken into consideration when designing the specific application. A balance between electrode size, physical range and achievable resolution has to be found. The MagicBox size does not allow an integration of very large electrodes (\ref{ch:prot_magicbox}). Instead, we are optimizing the available space in order to achieve a detection that lets us detect hands in a distance of approximately 30 cm and allows reliable tracking in distances between 15 and 20 cm. The system with the lowest detection distance is the CapFloor prototype (\ref{ch:prot_capfloor}). The long wire electrodes are not particularly well suited for achieving a good resolution and are prone to interference. However, they are suitable for this specific purpose and objects at a proximity of 10 cm or less can still be detected. 

\subsection{Object Detection}
\begin{figure}[ht]
\centering
\includegraphics[width=0.8\textwidth]{images/limit_detection}
\caption{Same response to differently sized objects (left), different response to varying materials (right)}
\label{fig:disc_obj_detection}
\end{figure}

Object detection using capacitive sensors, can be partially compared to object detection using camera systems, with a single sensor being equivalent to a single photo sensor. Smith et al. propose that loading mode measurements resemble light captured by a camera without a lens, as only one part of the electric system is constrained \cite{smith1998electric}. The light intensity measure is comparable to field intensity and likewise it is not possible to distinguish if the measurement is caused by a weak source in close proximity or a strong source at a further distance. As a practical example the capacitive sensor is not able to decide if one hand is close to the sensor or two hands are a bit further away. This effect makes it challenging to provide object detection and usually information from various sensors has to be combined to get a good idea about object shape and size. The CapTap prototype is able to detect two hands by combining the information of 24 sensors and additionally uses an analysis of the arm posture in the detection area (\ref{ch:prot_capfloor}). Due to the presented challenges in physical range and electrode size, capacitive proximity sensing systems do not have the same level of scalability as cameras, where millions of photo sensors can be placed in very small areas. Touch screens show that capacitive sensing technology has a very high resolution in near ranges. However, this can't be translated to similar resolution in further distances.

Additionally, the effect of an object on the electric field is not always closely correlated to the object dimensions, but instead based on conductivity, material and other factors. The same response to different objects may be measured at different distances, if they are differently sized or are made of different materials, as shown in Figure \ref{fig:disc_obj_detection}. In the presented application scenarios exclusively parts of the human body are tracked. Mostly, there is also an assumption that no other objects should be used. The CapTap will have to identify potential foreign objects from the hands of the user (\ref{ch:prot_captap}). In this case, the specific capacitance profile of objects at close proximity can be used, similar to the capacitance tags presented by SmartSkin \cite{rekimoto2002smartskin}. The Active Armrest has gestures for one and two fingers that are distinguished using a simple threshold (\ref{ch:prot_armrest}). If another object is entering the field, or the person is having a strongly different finger size, the system will fail to properly differentiate gestures. Accordingly, some other compensation methods should be used.


\clearpage
\section{Guidelines}
After discussing the limitations and benefits of capacitive proximity sensors, the final section of this chapter will give some general guidelines on their application. The first step of this process is a decision if capacitive sensors technology is suitable for the given application. This part should be driven by three questions.
What do I need to measure in my application scenario?
Capacitive proximity sensors can measure the presence and properties of conductive, grounded objects. This includes the various application scenarios shown in the previous sections. However, if the application requires measuring properties of unsupported objects that are non-conductive, a different technology should be chosen.


\textit{What sensing technologies are supporting the required measurements?}


It may be the case that multiple technologies sup-port the measurements required in this specific applications. Cameras often can provide similar recognition as capacitive sensors, e.g. in indoor localization applications. In this step all potential sensing technologies should be collected.
Are capacitive proximity sensors beneficial for my scenario?
An evaluation of the different candidates is the final step and should lead to a decision about the most suitable sensing technologies. If the distance is too high for capacitive proximity sensors or enough processing power is available and lighting conditions are static, cameras might be more suitable. This should be driven by the different benefits and limitations of  the technologies.
If there is a decision in favor of capacitive sensors the next step is to design the specific electrode layout. Similar to technology selection we can use a few basic questions to get an idea of what layout to use.


\textit{How many sensors are required to get the measurement?}


The number of sensors required is depending on the area we want to cover, the specific object parameters that have to be determined and the desired resolution. The electrodes are inherently limited in size, as a single sensor can only charge and discharge to a specific maximum capacity. Therefore, if a large area has to be covered more electrodes and sensors are necessary. If we just want to measure the presence of a hand a single electrode may suffice. If orientation and position are interesting we need to combine measurements from various sensors.


\textit{What should be the size and geometry of the electrodes?}


This is closely related to the previous question. If the application is not restricting the available space, the electrode should be approximately of the same size as the object that is to be detected. This generates the highest difference in capacitance when the distance is changing. 


\textit{What is the best electrode material to use?}


Copper is always a good first choice to create electrodes. If elasticity is necessary we can use copper foil and solid copper if that is of no concern. For transparent electrodes we will have to use one of the previously presented materials, such as ITO. If electrodes have to be integrated into cloth, conductive thread is a good candidate. Any conductive material will act as an electrode, thus the application and budget should be the primary driver of this decision.


\textit{Does my application require any shielding?}


Shielding allows detecting only objects approaching from a certain direction. If the application re-quires this additional hardware, because it is anticipated that other objects might disturb the measurement, shielding should be used.
Finally, if the hardware is designed as desired the different variations of data processing have to be chosen and configured according to the application.
Using baseline calibration is beneficial in the vast majority of applications. Having a distinct starting point simplifies all further steps of high-level data processing, such as normalization and setting different thresholds. This step may only be omitted in very stable environments and if the system has sufficient a priori information to operate on raw data. Drift compensation should be handled in a similar fashion. The common methods are not computationally expensive and having a stable baseline over time allows the same algorithms to be applied in a more robust fashion. The method and configuration of noise reduction are strongly depending on the specific case. Some form of noise reduction might be required in most applications. Yet, according to the type of noise different methods can be used. If outliers are an issue a median filter is appropriate; if a smoother signal is desired an average filter can be used. 
Regarding high-level data processing there are manifold variations of methods. Data-driven machine learning algorithms are a good method if we have a small set of potential outcomes of our applications, e.g. the different postures that could be recognized on a chair or couch. If our application has many different potential outcomes, e.g. the thousands of potential locations in a hand tracking system, it is typically beneficial to use a model-driven approach. However, these models may be supported by data-driven algorithms, such as particle filters. One example is the Swiss-Cheese object tracker by Grosse-Puppendahl et al. \cite{grosse2013swiss}. The data processing examples shown in the previous sections give an idea of the decision rationale in various application domains.
We can say in conclusion that capacitive proximity sensors are a viable, or even, ideal solution for a considerable number of different applications. However, a certain level of preparation is required in the design process to create a system that benefits from the technology.
