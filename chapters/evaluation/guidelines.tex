\section{Guidelines}
After discussing the limitations and benefits of capacitive proximity sensors, the final section of this chapter will give some general guidelines on their application. The first step of this process is a decision if capacitive sensors technology is suitable for the given application. This part should be driven by three questions.
What do I need to measure in my application scenario?
Capacitive proximity sensors can measure the presence and properties of conductive, grounded objects. This includes the various application scenarios shown in the previous sections. However, if the application requires measuring properties of unsupported objects that are non-conductive, a different technology should be chosen.


\textit{What sensing technologies are supporting the required measurements?}


It may be the case that multiple technologies sup-port the measurements required in this specific applications. Cameras often can provide similar recognition as capacitive sensors, e.g. in indoor localization applications. In this step all potential sensing technologies should be collected.
Are capacitive proximity sensors beneficial for my scenario?
An evaluation of the different candidates is the final step and should lead to a decision about the most suitable sensing technologies. If the distance is too high for capacitive proximity sensors or enough processing power is available and lighting conditions are static, cameras might be more suitable. This should be driven by the different benefits and limitations of  the technologies.
If there is a decision in favor of capacitive sensors the next step is to design the specific electrode layout. Similar to technology selection we can use a few basic questions to get an idea of what layout to use.


\textit{How many sensors are required to get the measurement?}


The number of sensors required is depending on the area we want to cover, the specific object parameters that have to be determined and the desired resolution. The electrodes are inherently limited in size, as a single sensor can only charge and discharge to a specific maximum capacity. Therefore, if a large area has to be covered more electrodes and sensors are necessary. If we just want to measure the presence of a hand a single electrode may suffice. If orientation and position are interesting we need to combine measurements from various sensors.


\textit{What should be the size and geometry of the electrodes?}


This is closely related to the previous question. If the application is not restricting the available space, the electrode should be approximately of the same size as the object that is to be detected. This generates the highest difference in capacitance when the distance is changing. 


\textit{What is the best electrode material to use?}


Copper is always a good first choice to create electrodes. If elasticity is necessary we can use copper foil and solid copper if that is of no concern. For transparent electrodes we will have to use one of the previously presented materials, such as ITO. If electrodes have to be integrated into cloth, conductive thread is a good candidate. Any conductive material will act as an electrode, thus the application and budget should be the primary driver of this decision.


\textit{Does my application require any shielding?}


Shielding allows detecting only objects approaching from a certain direction. If the application re-quires this additional hardware, because it is anticipated that other objects might disturb the measurement, shielding should be used.
Finally, if the hardware is designed as desired the different variations of data processing have to be chosen and configured according to the application.
Using baseline calibration is beneficial in the vast majority of applications. Having a distinct starting point simplifies all further steps of high-level data processing, such as normalization and setting different thresholds. This step may only be omitted in very stable environments and if the system has sufficient a priori information to operate on raw data. Drift compensation should be handled in a similar fashion. The common methods are not computationally expensive and having a stable baseline over time allows the same algorithms to be applied in a more robust fashion. The method and configuration of noise reduction are strongly depending on the specific case. Some form of noise reduction might be required in most applications. Yet, according to the type of noise different methods can be used. If outliers are an issue a median filter is appropriate; if a smoother signal is desired an average filter can be used. 
Regarding high-level data processing there are manifold variations of methods. Data-driven machine learning algorithms are a good method if we have a small set of potential outcomes of our applications, e.g. the different postures that could be recognized on a chair or couch. If our application has many different potential outcomes, e.g. the thousands of potential locations in a hand tracking system, it is typically beneficial to use a model-driven approach. However, these models may be supported by data-driven algorithms, such as particle filters. One example is the Swiss-Cheese object tracker by Grosse-Puppendahl et al. \cite{grosse2013swiss}. The data processing examples shown in the previous sections give an idea of the decision rationale in various application domains.
We can say in conclusion that capacitive proximity sensors are a viable, or even, ideal solution for a considerable number of different applications. However, a certain level of preparation is required in the design process to create a system that benefits from the technology.
