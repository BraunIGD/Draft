\section{Comparison to other sensing technologies}

% Table generated by Excel2LaTeX from sheet 'Tabelle1'
\begin{table}[htbp]
  \centering
  \footnotesize
  \caption{Add caption}
    \begin{tabularx}{\linewidth}{Xp{4cm}XXXX}
    \toprule
    \textbf{Name} & \textbf{Application Domains} & \textbf{Environmental Influences} & \textbf{Detection Range} & \textbf{Processing Complexity} & \textbf{Unobtrusiveness} \\
    \midrule
    \textbf{Capacitive proximity sensing} & indoor localization, smart appliances, physiological sensing, gestural interaction & electric fields, conductive objects & near distance   (< 100cm) & Few high dynamic range data sources  & invisible integration possible \\ \addlinespace
    \textbf{Capacitive touch sensing} & smart appliances, physiological sensing, gestural interaction & electric fields, conductive objects & touch  & Few binary sensors & thin cover above electrodes \\ \addlinespace
    \textbf{RGB cameras } & indoor localization, smart appliances, physiological sensing, gestural interaction & occlusion, external lights & far distance     (> 10m) & Complex image processing based on resolution & pinhole lenses \\ \addlinespace
    \textbf{Infrared cameras} & indoor localization, physiological sensing, gestural interaction & occlusion, external infrared light & medium distance (< 5m) & Complex image processing based on resolution & infrared source and camera \\ \addlinespace
    \textbf{Ultrasound sensing} & indoor localization, smart appliances, gestural interaction & acoustic occlusion, absorbing materials & medium distance (< 5m) & Few low dynamic range data sources & emitter and senders with exposed pinhole speaker, microphone \\ \addlinespace
    \textbf{Microphone arrays} & indoor localization, smart appliances, physiological sensing & environmental noise, absorbing materials & medium distance (< 5m) & Very high dynamic range data sources & exposed pinhole microphones \\ \addlinespace
    \textbf{Radiofrequency sensing} & indoor localization, smart appliances, gestural interaction & other RF devices & far distance     (> 10m) & Few low dynamic range data sources & hidden emitters and senders possible \\
    \bottomrule
    \end{tabularx}%
  \label{tab:addlabel}%
\end{table}%



In order to properly place capacitive proximity sensing in the smart environment domain it is neces-sary to include a comparison to other sensing tech-nologies. We have chosen systems that have a broad applicability and have been used in various smart environment applications. A short overview can be found in Table 7. We have included a comparison of application domains, environmental influences, de-tection range, processing complexity and unobtru-siveness of the technology. 
Capacitive touch sensing, as opposed to capacitive proximity sensing relies on an electrode being touched instead of an object being in proximity and is ubiquitous in touch screen applications.
RGB cameras are a class of image sensors operating in the same frequency domain as the human eye. They are capable of processing different colors.
Infrared cameras operate in near light frequencies that are invisible to the human eye. This allows for application in dark environments and we can project infrared light into the scene without disturbing the user.
Ultrasound sensing is using a low frequency range just above the audible limit of human hearing. The waves propagate similar to sound signals and we can perform reflection measurements or time-of-flight methods.
Microphone arrays detect signals in the range of human hearing, and thus work with audible signals, such as human speech.
Radiofrequency (RF) sensing uses signals in a range between several hundred kHz up to 5GHz, typically used for wireless communication. Commonly the signal strength or time of flight is used to gather information about the environment.
Most technologies are capable of supporting multiple application domains. Some non-intuitive examples include WiSee that enables whole-body gestural interaction using WiFi signals [86] or MoGees that uses a single microphone to enable gesture interfaces on various surfaces [87]. 
Capacitive sensors are disturbed by conductive objects and electric fields, whereas cameras struggle with occlusion and additional light sources. Occlusion is a weak point, and a line of sight is required. Sound sensors are prone to dampening materials and environmental noise interfering with the signal. RF signals usually propagate well through most materials and only external sources may be an issue. 
The detection range of the technologies varies strongly. RF ranges before light, sound and electric fields. However, this again strongly depends on apxplication and layout of the sensing devices.
It is not easy to find a good measure about the processing complexity associated to a different sensing technology. We are using a simplified model, taking the dynamic range of a sensor and the number of sensors typically required. Dynamic range is the difference between the smallest detectable value and the largest detectable value. Microphones have a high dynamic range measuring over a larger frequency scale, whereas touch sensors only have two different states. 
Finally capacitive sensors and RF sensors can be applied completely invisible. Cameras, microphones and ultrasound need a direct connection to the out-side world. However, there are very small variants available that are barely visible to the naked eye.

