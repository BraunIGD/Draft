\chapter{Introduction}
Smart environments are comprised of numerous sensing and computing devices that are supporting a number of users in this environment on performing their tasks. Capacitive sensors are a technology that uses electric fields to sense the presence and certain properties of the human body. In this work I present an overview of this technology, how it can be applied in different relevant application scenarios and  based on various prototypes evaluate the particular benefits and limitations of this sensing technology. 
\section{Motivation}
In the last decade the way we interact with computing machines has changed in a profound fashion. Today more than one billion people operate a smartphone, enabling ubiquitous access to communication tools, processing power and information. The vision of ubiquitous computing as proposed by Mark Weiser in the early 90s is inching closer to reality \cite{Weiser1991}. The required technologies of \begin{quote}
"cheap, low-power computers that include equally convenient displays, a network that ties them all together, and software systems implementing ubiquitous applications" 
\end{quote}
are now existing in the form of smartphones and tablets that are connected to the internet, using high-speed connections such as LTE and web-based services such as Google Now, that combine numerous data sources to provide personalized services.

While the vision and underlying ideas remain similar other names have been used in research, including Pervasive Computing and Ambient Intelligence. The concept has been expanded to not only consider devices that can be directly manipulated, but include determining the situation and reacting based on it. This context-aware computing proposes 
\begin{quote}
"systems that examine and react to an individual's changing context. Such systems can promote and mediate people's interactions with devices, computers, and other people" \cite{schilit1994context} 
\end{quote}
Different forms of context can be distinguished, ranging from location and the actual system state, to different activities or even the current mood of the user. In order to acquire this context, the input-and-output based systems originally proposed by Weiser, are augmented by an ensemble of devices that are very small (dust), coordinate in massive numbers (clay) or are flexible, unobtrusive extensions to everyday objects (fabric) \cite{poslad2011ubiquitous}. This devices can be invisibly integrated into our everyday environment and provide sensing capabilities that can be used by sufficiently smart systems. Examples of these devices are microelectromechanical systems (MEMS) or bendable technology, such as OLED screens. The number of computation and sensing devices that we carry with us is growing continuously, yet we want the technology to further disappear, allowing us to focus on the application instead of the underlying technology. 

The famous science fiction author Arthur C. Clarke proposed three laws of prediction, the  third of which is 
\begin{quote}
"Any sufficiently advanced technology is indistinguishable from magic." \cite{clarke1962hazards} 
\end{quote}
Capacitive proximity sensing allows us to measure the influence of the human body (or conductive objects in general) on an electric field. While we would not call this technology magic, a peculiarity of electricity is that humans have no specific sensing organs, thus we generally remain unaware of their presence, unless the field strength is very high. Consequently, when interacting with capacitive sensors there is no awareness of what they are sensing unless it is specifically exposed to the user. The technology is well-understood and varieties have become ubiquitous in some areas, such as touch screens. However, there are numerous other applications for this technology, ranging from industrial fluid level and material detection, to presence detection for cars. A particularly interesting domain for this sensing technology are smart environments that provide services based on unobtrusively acquired information about persons currently acting in this environment. There are numerous sensing technologies that provide similar detection capabilities. Looking at the recognition of simple activities, such as standing, walking and lying, cameras and accelerometers can lead to the same result. Accordingly, in order to discuss the use a sensing technology within a specific domain, it is necessary to provide a benchmark that takes into account abstracted sensor properties and different application domains. In this work we will provide a generic benchmark model for different sensor technologies in smart environments and based on this discuss the use of capacitive proximity sensor technologies in this area. We will establish the most suited application domains and provide prototypes to evaluate different aspects.
\section{Research Challenges}
In the past there have been numerous great works that gave an overview of technologies and applications in smart environments. Cook et al. identified common technologies, frameworks and applications in this domain and give an overview of ongoing research. Poslad specified a more detailed taxonomoy of device classes, provides concepts for interaction between humans and environments and gives an overview of intelligent systems. A different category  of previous work details the different sensing technologies that are supporting various different applications and give an overview of limitations and benefits. However, so far there has been no work that provides a benchmark that maps different sensor characteristics to applications in smart environments. As it was stated by Cook et al.:
\begin{quote}
"Finally, a useful goal for the smart environment research community is to define evaluation mechanisms. While performance measures can be defined for each technology within the architecture hierarchy [...], performance measures for entire smart environments still need to be established. This can form the basis of comparative assessments and identify areas that need further investigation."
\end{quote}
In this work, we will contribute to this challenge and provide a benchmark based on methods previously presented by x. Using this benchmark we can identify application domains that are particularly suited for capacitive proximity sensors and provide a tool that allows researchers to select a suitable sensing technology for the given application.

The past few years have seen several emerging trends in computing, ranging from an increased connectivity of devices, driven by the Internet of Things, ubiquitous usage of mobile computing and sensing devices in the form of smartphones and tablets, and novel, natural interaction paradigms, that aim to provide human-machine interaction similar to interpersonal communication means. 

Driven by improved embedded technology, materials and an increase in computing power, it is possible to provide integrated systems based on capacitive proximity sensing that contribute novel aspects to several of these trends. In the last years we have created various prototypes in the identified application domains and applied state-of-the art sensor technology and novel algorithms. In this process we are able to provide numerous improvements to previously presented systems and enable new applications. Based on this it is possible to provide a thorough review of capacitive proximity sensing technology in the domain of smart environments.
\section{Contributions}
\begin{itemize}
\item Identification of application domains in smart environments
\item Develop benchmark model for mapping sensor features to smart environments
\end{itemize}

\section{Acknowledgments}
While many consider writing a PhD to be a mostly personal endeavor there are always various sources of discourse, collaboration, support and inspiration. 
So in no particular order there are various persons or groups of persons that deserve credit: 