\section{The Capacitive Chair}
\begin{figure}[h]
\centering
\includegraphics[width=0.4\textwidth]{images/smartofficechair}
\caption{Smart office chair sketch - eight electrodes three in backrest, three on seat and two in armrests}
\label{fig:smartchair_sketch}
\end{figure}
The Capacitive Chair is a regular office chair equipped with eight capacitive proximity sensors that can detect different sitting postures and work-related stress levels by examining movement and breathing rate \cite{Braun2013ChairAid}. Seven solid copper electrodes that are placed below the covering are augmented by a single conductive thread electrode that is placed in a mesh on the backrest. In the past smart chairs have used pressure sensors to infer posture and occupation \cite{tan2001sensing}. Combining presence and proximity sensing it is possible to directly infer postures where parts of the body do not touch the surface, e.g. if the body is arched towards the front, or if an arm is raised from the armrests. Additionally higher area electrodes in the backrest allow detecting the breathing rate by measuring the movement of the chest.

The Capacitive Chair aims at providing different services to a typical office worker and office managers. Using the occupation detection it is possible to advise for some type of physical activity, if the time spent in front of the screen was too long. The system can also advise the user to change to a more back-friendly posture or regularly switch the stance to achieve a more general workout. Using the breathing rate detection we are able to get some sort of measure of the current stress level associated to the given working situation. By adapting the environment it is possible to improve the working atmosphere and reduce stress. The Capacitive Chair uses a multifacetted data processing approach. A machine learning algorithm is associating the sensing data to one of nine different typical sitting positions, inspired by a recent study of sitting positions for modern device usage \cite{globalPosture}. An adaptive body model that is fitted to the current sensor values allows for fine grained adaptation of those postures. Finally a combination of Fourier and data variation analysis is calculating the current breathing rate \cite{Braun2013ChairAid}.

\subsection{Data processing}
\begin{figure}[h]
\centering
\includegraphics[width=0.7\textwidth]{images/smartchair_software}
\caption{Screenshot of the Capacitive Chair application showing the fitted 3D model on the left, posture detection on the upper right and the recognized posture on the lower right}
\label{fig:smartchair_software}
\end{figure}
In Figure \ref{fig:smartchair_software} we can see a screenshot of the Capacitive Chair debug application. On the left side we see a 3D model that is fitted to a chair model according to the current sensor values, in the middle the results of the machine learning module and the recognized posture and on the right side the currently running breathing rate detection as both Fourier analysis and signal deviation analysis.
All processing methods work on filtered and normalized sensor data. The difference in shape, material and size of the electrodes necessitates slight adaptations to noise filtering and data processing. As an example only the conductive thread backrest electrode is used in the breathing rate detection. 
The 3D model is using a simplified human joint model comprised of 13 connected components. Based on the current sensor readings, single parts or groups of components are fitted to the virtual chair. The process is a mix of posture mapping as found in the smart bed and modification of the dynamic links between the single components \cite{Braun2013ChairAid}.
\begin{figure}[h]
\centering
\includegraphics[width=0.7\textwidth]{images/smartchair_thread}
\caption{Screenshot of the Capacitive Chair application showing the fitted 3D model on the left, posture detection on the upper right and the recognized posture on the lower right}
\label{fig:smartchair_thread}
\end{figure}
We use a simple RBF neural network and training data collected by two different persons to match the input from eight sensors to nine potential output postures that are associated to different working situations. An early observation is that certain postures are difficult to distinguish given the limited number of sensors and the similarity of the postures on the rigid chair. Either a higher number of sensors or a more versatile chair could be used that allows gathering additional information required to distinguish the different poses more reliably. 

The breathing rate detection is operating on a single electrode that is integrated into a mesh on the backrest using conductive thread. The setup is shown in Figure \ref{fig:smartchair_thread}. Consequently the surface of the electrode is large and able to pick up the chest movement. Two different methods of data processing are used and fused to get the final breathing rate. Using a fast Fourier transformation the signal is transformed into the frequency space. We are looking for significant signal portions in frequency areas that can be associated to breathing, between $0.2Hz$ and $10Hz$. The second method is to look for zero-crossings of the sensor signal through an adaptive baseline. If a person is breathing in the sensor value will decrease resulting in the signal dropping below the long-term average, and rise above when the person is breathing out. Accordingly the breathing rate can be calculated by counting the zero-crossings.
\subsection{Evaluation}
