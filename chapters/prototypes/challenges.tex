\section{Use cases and associated challenges}
% Table generated by Excel2LaTeX from sheet 'Tabelle1'
\begin{table}[htbp]
  \centering
  \caption{Application domains and derived implemented use cases for capacitive proximity sensing}
    \begin{tabularx}{0.8\linewidth}{p{4cm}XX}
    \toprule
    Application Domain & Applying capacitive proximity sensors & Implemented use cases \\
    \midrule
    Indoor Localization & Sensing system hidden in environment & Capacitive sensing below floor cover \\
    Smart Appliances & System detecting presence and other parameters of human bodies in range & Posture recognizing office chair, occupation sensing bed, arm detecting armrest \\
    Physiological Sensing & Determine physiological parameters associated to movement & Breathing rate detection via chest movement, long-term movement analysis \\
    Gesture Interaction & Hand interaction in near range & Finger gestures, single hand 3D gestures, combined multi-hand and touch tracker \\
    \bottomrule
    \end{tabularx}%
  \label{tab:usecase_list}%
\end{table}%

Looking at the previously defined application domains for capacitive proximity sensing we can have a closer look and think about actual use cases that belong in the different domains. While it is also possible to associate the different systems presented in the related works I will focus on the implemented prototype systems that are presented in the subsequent sections. Table \ref{tab:usecase_list} shows the different application domains, how capacitive proximity sensors can be applied and the use cases that were derived from it. In this section I will discuss the creation of this table and create a list of challenges that become apparent when designing the specific systems. Based on these challenges it is possible to identify a number of steps in the processing of capacitive proximity sensor data that can be improved, in order to enable the presented applications. These specific contributions to processing methods will be detailed in the following section.

Indoor localization has been presented as one of the application examples for smart environments in the related works section. The main advantage of capacitive systems is their unobtrusive application in the environment as presented in TileTrack \cite{Valtonen2009a} and SensFloor \cite{lauterbach2009}. Capacitive indoor localization systems can be hidden below any non-conductive material and enable tracking of users on a distance. Particularly interesting is the ability to place the sensing equipment below the floor cover as an additional layer, e.g. when installing a new carpet or wood parquet. SensFloor as commercially available solution is designed to be placed as an additional layer and integrates sensors and communication chips in this layer. While this enables a precise and reduced noise sensing close to the walking persons, it is costly and can lead to maintenance issues once the sensors integrated into the layer fail. Instead, it is also a viable option to separate sensor hardware and electrodes, e.g. by placing the sensors on the borders of the indoor area and use a specific electrode layout below the floor. The electrodes can be made of any conductive material and can be protected using non-conductive isolation to prevent corrosion and physical damage. The system components that are most prone for failure are the connections between electrode and sensor and the sensor hardware and its communication channels. Those can e.g. be placed within the border covers.

The main challenge of this solution is to balance the number of sensors and the required resolution. Using a limited set of sensors placed on the border it should nonetheless be possible to determine the positions of one or more persons on the area above and potentially additional information, such as the status of a person, if it is standing, sitting or lying. In addition to the cost factor a larger number of sensors also causes several other issues, such as cross-talk between the different electrodes that has to be avoided using a variety of multiplexing methods. The achievable resolution of the single electrodes is depending on several factors, including the measurement time, the applied voltage, distance between electrodes and floor surface, or the geometric layout of the electrodes. Thus, it is important to find an electrode layout and processing methods that achieve this balance.

Smart appliances as presented in the related works can be a very diverse group of devices that are in the current environment. There is a huge variety of sensor categories and processing that can be applied to any given task. Looking at capacitive proximity sensors, the major advantage is the invisible application that allows to create smart appliances that are indistinguishable from systems without sensor devices. Using different conductive materials for the electrodes this integration can range from solid antennas hidden within the appliance to conductive threads that can be woven into fabric. The main application for capacitive proximity sensors in smart appliances is the sensing of different parameters of persons interacting with the system. For example the sensors can be used to recognize the posture of a user and use it to adapt certain parameters of the appliance or the environment. This type of interaction has also been called implicit interaction, as the user does not directly attempt to manipulate the environment, but instead the activities are interpreted as input according to the given situation \cite{schmidt2001build}. In many instances it is sufficient to get information about the presence of the user. A simplified version of the posture recognition can be used to detect presence or occupation, based on the data acquired by one or more sensors. Finally, it is often also interesting to detect if certain body parts are currently at a given location, e.g. the arm resting on the armrest of a chair, indicating a specific situation that the system can react on.

The challenges in this domain are manifold. Existing posture recognition systems might rely on a different sensor category, supporting hundreds of measurement spread over a larger area. Again, capacitive proximity sensors are distributed sparsely and need methods that enable gaining a similar amount of higher-level information. Here it is necessary to create models of the human body that are suited for processing of capacitive data. According to the parameters that are supposed to be detected the models can be more or less complex and thus improving the required processing time. A sensing bed that wants to detect how a person is lying on it would require a simpler model, as opposed to a sensing chair that would require to detect a larger variety of different postures. Often the capacitive systems are combined with other systems, or use a custom non-uniform distribution of electrodes in the device that require methods of data fusion and processing of heterogeneous signals to acquire higher level information, e.g. an armrest that combines the detection of an arm and an interactive area that allows gesture interaction.

Physiological sensing allows us to measure signals generated by the different process of the human body. One common application is for athletes that track the effects of training on their body parameters and might measure heart rate or respiration. There are numerous medical applications ranging from long-term blood pressure sensing, to blood glucose level sensing or tracking the quality of sleep throughout the night. Additionally, there are physiological signals derived from long-term monitoring, such as movement-based sleep-phase detection. Measuring electric properties is the most common variety to detect physiological parameters, ranging from EEG, measuring the brain activity, or ECG measuring the heart rate, or sensors for skin conductance that can infer the stress level. For all these applications electrodes are placed very close to the measured property, often even requiring contact. Capacitive proximity sensors on the other hand are by definition used over a distance. However, the systems can be designed to enable a high resolution that can track very small movements of the body. Rob MacLachlan has created a spread spectrum system that is able to measure the chest movement associated to breathing over a distance of more than 30cm \cite{MacLachlan2004}. Other examples include the detection of swallow movements \cite{cheng2010active}. 

There are various challenges when trying to gather physiological signals from capacitive proximity sensors. A major problem is to distinguish the measured property from other signals generated by movement of the body. Here it is possible to use the effect that the measured properties often are prevalent in a specific frequency range. Thus, if the signals are analyzed in the frequency domain it is possible to extract the physiological properties from the overall signal, e.g. when analyzing the chest movement associated to respiratory rate and focusing on the frequency areas most important. Regarding long-term physiological signals, capacitive proximity sensors can be used to aggregate data on movements. In this regard, it is interesting how the sensor data in time-domain can be associated to particular movements that can be used in long-term analysis of the user’s physiological patterns, e.g. to detect sleep phases. Based on the particular setup of the system a different set of features has to be selected and evaluated.

Gesture interaction is a very diverse application area that reaches from the acquisition and interpretation of whole body gestures to small movements of the fingers registered on surfaces. It is maybe the most thoroughly researched domain of capacitive proximity sensors, starting with Leon Theremin's musical instrument. The MIT research group experimenting with capacitive interaction in the 90s created some concepts for touchless interaction, e.g. the Field Mouse that allowed to control the third dimension in certain applications \cite{Smith1996a}, or an art installation that could be controlled using a set of gestures \cite{smith1998electric}. A new category of interaction devices such as Wii remote, Kinect or Leap motion led to the proclamation of more natural interaction between human and machine \cite{Valli2008}. While capacitive touch sensors have become ubiquitous in mobile devices, the proximity variety is less frequently used. Wimmer integrated several sensors into a table to enable a regional interaction on the surface \cite{Wimmer2006}. 

While the area has been well-researched there is still a number of challenging aspects. In many instances the capacitive interaction devices will have a different resolution according to the direction. In the last years there has been a rise in methods that allow a generic recognition of gestures in two dimensions, e.g. from mouse cursor movement and finger movement on touch screens. It is interesting to evaluate if this is also possible for 3D positions acquired from capacitive proximity sensors. The acquisition of the hand position is also challenging, as the sensors can't distinguish between hands and other conductive parts of the body. Thus it is interesting to investigate different methods of fitting arms and hands, particularly on larger area interaction devices. Additionally, it is challenging to enable gestures via multiple hands and arms. In many gesture interaction applications fatigue is a challenge, if the hands have to be moved too much or the arms have to be held in free air for a longer time. Designing specific graphical user interfaces that are suited for this. If the capacitive proximity sensors are placed under thicker layers of non-conductive material it is difficult to detect touch events from capacitance data alone. It becomes interesting to combine capacitive proximity sensors with other sensor categories that can detect touch or even different touch events, thus allowing a richer interaction. 
% Table generated by Excel2LaTeX from sheet 'Tabelle1'
\begin{table}[htbp]
  \centering
  \caption{Challenges associated to the different use cases for capacitive proximity sensors}
    \begin{tabular}{lr}
    \toprule
    Use cases & Challenges \\
    \midrule
    Capacitive floor sensors & Sparse sensor distribution in large areas, geometric electrode layout \\
    Posture chair & Multi-body models, electrode material \\
    Occupation sensing bed & Single-body models, movement tracking \\
    Armrest supporting gestures & Heterogeneous capacitive arrays \\
    Breathing rate detection & Frequency spetrum analysis \\
    Sleep phase detection & Long-term movement features \\
    Finger micro gestures & Small 3D movements \\
    Multi-arm tracking & Arm and hand fitting, interaction design \\
    Combined touch sensing & Combining position tracking and touch events \\
    \bottomrule
    \end{tabular}%
  \label{tab:usecase_challenge}%
\end{table}%

In consequence there is a large number of specific challenges that can be tackled in the different domains. They can be associated to the identified use cases using Table \ref{tab:usecase_challenge}. In the following section I will present the methods that have been developed in this regard and how they contribute to the different challenges in processing the data generated from capacitive proximity sensors.

   