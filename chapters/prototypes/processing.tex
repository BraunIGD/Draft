\section{Processing methods}
\label{ch:prot_proc}
After analyzing the challenges associated to the different use cases for capacitive proximity sensor I will use the following section to specify a number of novel or improved data processing methods that can be used in this context. They are grouped into five specific areas:

Sparsely distributed sensor arrays refer to configurations that have a limited number of sensors spread over larger areas. In this regard it is important to find methods that allow acquiring sufficient information about the object to be detected. Typically information about the electrode geometry and interpolation methods are used to meet these requirements.

The second area are model-driven fitting methods. Using a simplified model of the object to be detected, it is possible to fit these to the received sensor data. The area can be distinguished according to the complexity of the models that are either comprised of a single body or multiple parts that are connected to each other.

Heterogeneous sensor systems can be described as a combination of multiple sensors that are not uniform. In terms of capacitive proximity sensors this can refer to either arrays of different capacitive sensors that use a geometric layout of varying sizes and shapes or the combination of capacitive sensors with other categories of sensing systems in a meaningful fashion.

Image-based processing describes the method of creating an image from capacitive sensor data and applying different algorithms associated to visual computing. An uniform array of capacitive proximity sensors resembles an array of light sensors on a different frequency interval of electromagnetic radiation. Thus, with a few limitations it can be treated similar to a camera system with operations applied on a pixel level.

A last group is the processing of physiological signals in time domain and frequency domain. Many physiological activities rely on the movement of muscles, e.g. the beat of the heart or the chest movement associated to breathing. These movements have an effect on the electric field generated by capacitive proximity sensors and can be analyzed using a variety of different methods.

\subsection{Sparsely distributed sensor arrays}
Sparsely distributed sensor arrays refer to layouts that limit the number of available sensors either by environmental parameters or by design. 
\subsubsection{3D location tracking}
 \begin{figure}[h]
\centering
\includegraphics[width=0.6\textwidth]{images/magicbox_data_zaxis}
\caption{Piecewise linear hand distance estimation \cite{Braun2011MultiInputDevice}}
\label{fig:magicbox_data_zaxis}
\end{figure}
%Figure 29 Piecewise linear hand distance estimation [78]
The first data processing step of the MagicBox is the planar localization of the hand, following the weighted average algorithm previously presented. In order to calculate the distance of the hand from the plane we are using a piecewise linear interpolation, that resembles the response curve of a single sensor \cite{Braun2011MultiInputDevice}.
\begin{figure}[h]
\centering
\includegraphics[width=0.7\textwidth]{images/magicbox_data_gest}
\caption{Gesture overview module (left) and gesture recorder (right)}
\label{fig:magicbox_data_gest}
\end{figure}
%Figure 30 Gesture overview module (left) and gesture recorder (right)
An addition of the MagicBox was a generic gesture recognition module based on methods similar to mouse gesture recognition \cite{braun2013capacitive}, albeit adapted for three dimensional locations. The developed debug software allows defining an arbitrary set of potential gestures and adding training data, as shown in Figure \ref{fig:magicbox_data_gest}. The module is looking for matches based on the most recent set of locations. 
The general functionality of a gesture recognition framework that is using learning by example is shown in Figure 1. A set of features is extracted from incoming raw data. These are distributed to training sets that are used to associate certain features to given gestures. After this learning process is completed current feature sets that are acquired on-the-fly are tested against the training features in the repository. If certain criteria are met, these lookups lead to successfully recognized gestures and
subsequently, an action may be performed. 

Figure 2 - Gesture recognition framework layer model
Our proposed framework is implementing gesture recognition by example, employing a multi-level design as shown in Figure 2. The lowest level - User Control - is a collection of aspects that are controllable by the user. The Algorithm Control allows setting the parameters of the gesture recognition algorithm, Current Feature allows selecting between different methods of feature acquisition from the raw data and Current Gestures is the set of gestures that
can be recognized. The Object Level is encompassing all objects in the frameworks, the group of gesture recognition algorithms the user can choose from and their current settings. Furthermore, there are the available features including settings and the set of available gestures. Above is the Module Management Level that controls feature acquisition and gesture management on a contextual level, meaning that, based on the current situation, the framework can select different features and gestures to process. Finally the Framework Management Level is controlling high-
level functionality of all modules and provides interfaces to external applications to control the framework and access registered gestures.
The framework is implemented in the Microsoft .NET environment (version 4.0) using C\#. It requires a gesture
recognition device that provides location data in three dimensions.
In our cases this is an array of capacitive proximity sensors that will be detailed in the prototype section.
Figure 3 - Screenshots of gesture manager and gesture
recorder
The key aspect of gestures by example – providing examples - is realized in a debug application. It provides a simple way to record exemplary movements and associate them to gesture sets. The main screens realizing this functionality are shown in Figure 3.
On the left side we can see the management screen that allows
adding and deleting of gestures, as well as a preview window that is an average of the sample data associated to this gesture. The process of entering data is shown on the right side where several samples can be recorded and associated to the selected gesture and the user can decide, whether the current movement should be stored or discarded.
\subsubsection{Large-area location tracking}
Using long wire electrodes may result in considerable noise and influence from outside electric fields. Therefore CapFloor requires preprocessing to reduce the noise and achieve a more robust high-level data processing. The localization uses the weighted average algorithm that has been presented previously. 
\begin{figure}[h]
\centering
\includegraphics[width=0.8\textwidth]{images/floor_shapes}
\caption{Shapes of a standing and lynig person on top of the CapFloor grid}
\label{fig:capfloor_shapes}
\end{figure}
The fall detection is using a time-series analysis of the aggregated values of the sensors that are currently detecting an object. This method is using the assumption that the overall sensor response is roughly equivalent to the shape of the object that is closest to the surface, resulting in a higher capacitance of the overall system, similar to the plate capacitor model. This effect is shown in Figure \ref{fig:capfloor_shapes}. The sum $s$ of all n sensor values $r$ is the closest equivalent to the system capacitance and therefore a viable measure. If the overall value is beyond a certain threshold $v_l$ we can consider a lying person $p_l$.
\begin{equation}
s=\sum^n_{i=0}{r_i}\ \ \ ,\ \ \ p_l=\left\{ \begin{array}{c}
1,\ \ \ s\ge v_l \\ 
0,\ \ \ s<v_l \end{array}
\right.
\end{equation}
In order to increase the robustness this threshold has to be exceeded for a certain amount of time $t_m$. In consequence a fall $f$ is detected if the following equation is 1.
\begin{equation}
f=\prod^{t_m}_{j=0}{p_{l,t_j}}
\end{equation}
\subsection{Model-driven fitting methods}
\label{ch:proc_model}
When acquiring sensor data from physical objects, it is often difficult or even impossible to analytically describe the resulting value. There are numerous environmental factors influencing the signal and the properties of the object might not be clearly determined. Considering the human body, there is a mostly unconstrained number of sizes, shapes and biological properties that influence the response to an electric field. Thus, in order to fit sensor outputs to the potential object configurations, simplified models can be used that resemble the actual physical effects and can be described analytically. Regarding capacitance of the human body relative to a single sensor, a common abstraction is a sphere having a diameter close to the height of an average human \cite{seaver1997human}.

Models based on a single geometric objects are considered single-body, while connected geometric objects that comprise a single model can be called multi-body. Smith used a model of multiple spheres to approximate arm position and rotation above an array of capacitive proximity sensors \cite{smith1998electric}. Another possibility is adapting the models to a derived physical effect. Harada et al. are using the projected pressure distribution of a virtual skeleton and body model on a flat surface to create a pressure distribution that can be compared to the actual pressure effect generated by an actual human body resting on a set of sensors \cite{harada2000human}. In this section I will describe two novel methods to fit abstracted models of the human body to sensor readings acquired from smart furniture systems. The first method uses a cylindrical human body model to match the posture of one or two bodies on a bed, the second method uses a multi-body skeleton that is fitted to sensor readings determining posture on a chair.

\subsubsection{Single-body models}
\begin{figure}[ht]
\centering
\includegraphics[width=0.8\textwidth]{images/proc_hetero_flexpressure}
\caption{Object on mattress decreases distance and changes geometry of flexible electrode \cite{braun2012context}}
\label{fig:proc_hetero_flexpressure}
\end{figure}

While models are more often applied directly to the capacitive sensor data, it is also possible to use a different property that is derived from it. Together with Henning Heggen, I developed a system that uses flexible electrodes that bend on pressure. In previous experiments we observed that bending a calibrated flexible electrode results in a higher value of the sensor. Thus, the idea was created to use this effect for combining presence and pressure detection. A first application was to use such a system on the slatted frame of a bed. The following section is based on two different publications \cite{Hamisu2010,braun2012context}. 

If an object is applying force to the mattress there are two cumulative effects. The object will be in detection range of the sensors, decreasing the distance by deforming the mattress and the applied pressure changes the geometry of the flexible electrode, resulting in a higher sensor value. The basic idea is shown in Figure \ref{fig:proc_hetero_flexpressure}. If a model is used that approaches the pressure distribution of a human body, a small number of capacitive proximity sensors equipped with flexible electrodes should allow to detect different posture and occupation configurations on the bed. Additionally, sitting and lying can be distinguished, for one or two persons. While the sensor values are increasing according to higher pressure, this relationship is highly non-linear and influenced by three groups of system parameters. The sensor system parameters include the sensor design, e.g. excitation voltage and frequency, as well as geometry and material of the electrode. These parameters additionally determine range and precision of the overall system. The second group are environmental parameters that influence the conductivity of the excited electric field, including humidity and temperature. The third group are the bed parameters. These include the hardness of the slatted frame that influences the geometric deformation of the affixed electrodes and the mattress parameters, comprised of hardness, thickness, material and type, influencing how far the body is away from the electrodes and how the pressure is distributed onto the slatted frame.

\begin{figure}[ht]
\centering
\includegraphics[width=0.8\textwidth]{images/prot_model_bed}
\caption{Cylindrical human body model and various poses on mattress \cite{braun2012context}}
\label{fig:prot_model_bed}
\end{figure}

To identify occupation and positioning we use a very simple model for estimating the effect of a human body on the sensor values. As previously mentioned, the sensors react to both presence of a body and applied pressure. The human body is modeled as a cylindrical object on the mattress. The object can be either sitting or lying and there might be multiple persons on a single bed. A few potential poses are shown in Figure \ref{fig:prot_model_bed}. It is assumed that a sitting person will cause a high pressure within a small region, and that a lying person has a moderate pressure distributed over a larger area. Thus, the challenge of determining posture and orientation from a limited number of capacitive sensors, tuned to detect pressure, can be formulated as an inverse problem. If we assume a constant density of the cylinder and a uniformly deforming mattress, the idealized pressure distribution is uniform as shown in Figure \ref{fig:prot_model_pressure}.

\begin{figure}[ht]
\centering
\includegraphics[width=0.6\textwidth]{images/prot_model_pressure}
\caption{Pressure distribution of a uniform cylinder \cite{braun2012context}}
\label{fig:prot_model_pressure}
\end{figure}

The system is further simplified by reducing the significant values of the pressure distribution to just two distinct values. Be $p_c$ is the center of pressure and $p_{-0.5}$, $p_{0.5}$ the points enclosing half of the pressure distribution. The center of pressure and standard deviation $\sigma$ are sufficient to describe the system, as shown in the following equations:

\begin{align}
p_c&=\frac{a+b}{2} & \sigma&=\sqrt{\frac{(b-a)^2}{12}}
\end{align}
\begin{align}
p_{-0.5}&=p_c-\sigma &	p_{0.5}&=p_C+\sigma
\end{align}

The raw data from the sensor is considered as random, uniform sampling, a discretization of the continuous distribution. The center of pressure and the standard deviation are calculated using the geometric information, the position of the sensor $\overrightarrow{x}$.

\begin{equation}
p_c=\frac{\sum_{i=1}^n{v_i\overrightarrow{x}}}{\sum_{i=1}^n{v_i}}
\end{equation}

\begin{equation}
\sigma=\sqrt{\frac{1}{n}(\sum_{i=1}^n{\overrightarrow{x}^2}-\frac{1}{n}(\sum_{i=1}^n{\overrightarrow{x}})^2}
\end{equation}

Using this model, a set of potential poses that cover the most common situations has been determined. Potential poses for one and two occupants are distinguished. One person may sit at a certain location or lie on the bed in various angles. It is assumed that the head is always at the upper part of the bed. Two persons may either both sit, both lie down, or one is sitting while the other is lying. 
The limitations of this model concerning the actual system are the non-uniform pressure propagation throughout the mattress, as well as the non-linear sensor response on different pressure levels. Therefore, it is not expected that the deviations adhere to the theoretical model, but instead configurable thresholds are used that allow for increased robustness in exchange for precision.

\begin{figure}[ht]
\centering
\includegraphics[width=0.7\textwidth]{images/smartbed_cog}
\caption{Calculating centers of pressures and deviation \cite{braun2012context}}
\label{fig:smartbed_cog}
\end{figure}

Occupation and posture detection are performed by dividing the two person bed into a left and right section. For each side the total sensor values, assumed center of pressure  and the standard deviation are calculated individually(Figure \ref{fig:smartbed_cog}). The same calculation is performed between the two areas, allowing to distinguish the specific side of activity, or detecting persons that are lying on the bed diagonally.
Using these six intermediate values it is possible to map the different poses. If all activity is on one side and the horizontal deviation is low, it can be assumed that one person is sitting. Additionally, the intermediate values can be used to gather more information, e.g. the exact location a person is sitting at. 

\subsubsection{Multi-body models}
If it is necessary to identify more complex human behaviors than sitting and lying, a single body model is no longer sufficient. In the briefly mentioned work by Harada et al. the underlying model was a 3D representation of an average human body that could move various joints freely \cite{harada2000human}. This allows to detect a variety of different postures, in this case using an iterative process based on potential energy, momentum and difference between the actual and simulated pressure distribution. Using such full body models, based on an internal skeleton of connected joints is also common in full body gesture tracking systems. However, while in those cases the volume of the different body parts is important, there is not necessarily any additional physical simulation of properties, such as weight and density \cite{Shotton2013}. 

In a project together with student Sebastian Frank, a concept for a smart chair was developed that unobtrusively integrates a set of capacitive proximity sensors to detect presence, posture, activity and breathing rate of a person sitting on the chair \cite{Braun2013ChairAid}. This requires to place suitably sized electrodes strategically, at locations that allow getting a good measure of the most significant body parts. Using eight electrodes distributed between seat, backrest and armrests, the sensor values are mapped to postures using two different methods, according to the specific purpose of the underlying application. The first method directly translates the sensor values to body part positions, while the second uses a machine learning classification.

\subsubsection*{Direct manipulation of skeleton model}
\begin{figure}[ht]
\centering
\includegraphics[width=0.8\textwidth]{images/smartchair_skeleton_model}
\caption{Smartchair skeleton model and associated body parts}
\label{fig:smartchair_skeleton_model}
\end{figure}

In Figure \ref{fig:smartchair_skeleton_model} the skeleton model can be seen. It is comprised of 15 different groups. As those are combined to each other, the degrees of freedom are significantly reduced to just 10. Sensors close to the different body parts will modify the associated parts of the model. As an example, the distance of the left lower arm group is determined by the sensor in the left armrest. However, as the groups are connected, this will also modify the orientation of the left upper arm. Accordingly, if the sensor behind in the lower backrest detects a receding lower back, the position and orientation of the other back groups, head and arms are also modified. Thus, the fitting method attempts to place the different groups of the model according to current sensor readings. A collection of rules is used that follows the flowchart outlined in Figure \ref{fig:capchair_flow}.

\begin{figure}[ht]
\centering
\includegraphics[width=0.9\textwidth]{images/capchair_flow}
\caption{Flowchart of the model fitting process of the grouped skeleton parts and performed calculations}
\label{fig:capchair_flow}
\end{figure}


The calculation of the different steps varies and is outlined shortly in the following enumeration:
\begin{enumerate}
\item \emph{Check sensor activity} uses a threshold of the different sensors to determine the presence of any objects in detection distance
\item \emph{Calculate proximity} use a second thresholds of seat sensors to determine sitting or non-sitting status
\item \emph{Calculate sitting position} use a weighted average of the seat area sensors to determine position on seat
\item \emph{Calculate back, leg, arm pose} calculate back pose by determining distance from back to backrest based on sensor values. Raise leg according to proximity to front seat sensor. Calculate arms based on proximity to arm rests.
\end{enumerate}

This method allows a fine-grained fitting of the model to the sensor values that closely resembles the pictures of a person moving in the chair. A number of poses and the accordingly fitted 3D models using this method can be seen in Figure \ref{fig:smartchair_skeleton_poses}.

Using a fitting method that allows to detect fine movements enables a number of unique applications. One example is to track a set of different exercises that can be performed on an office chair, in order to prevent future back problems. Following an exercise routine can significantly reduce the indicative risks \cite{robertson2009effects}. Some of the common exercises can be tracked for accuracy and number of repetitions using the capacitive proximity sensors in the chair.

\begin{figure}[ht]
\centering
\includegraphics[width=0.7\textwidth]{images/smartchair_skeleton_poses}
\caption{Screenshots of the Capacitive Chair application showing different poses of the skeleton model}
\label{fig:smartchair_skeleton_poses}
\end{figure}

A limitation of the above method are the fixed threshold levels that are set based on experience. Either a calibration method can be used that personalizes the different thresholds, or another form of classification. This should be based on training with a larger variety of different body volumes. It is suitable to use machine learning methods for this task, as they are designed to learn from a larger set of training data and provide numerous classification methods. However, this leads only to a discrete set of different postures, limiting some potential applications. One method is described in the following section.

\subsubsection*{SVM classification}
\begin{figure}[ht]
\centering
\includegraphics[width=0.7\textwidth]{images/smartchair_gps_poses}
\caption{Selected set of postures from Global Posture study and own gestures. \emph{From top left to bottom right:} The strunch, the draw, the smart lean, the take it in, upright, no person (first four taken from \cite{globalPosture}}
\label{fig:smartchair_gps_poses}
\end{figure}

Support vector machines (SVM) are a supervised learning method that is primarily used for linear classification of n-dimensional features \cite{hearst1998support}. They are clustering data by calculating a hyperplane from training data that maximizes the distance from the closest features. A fast learning method is using sequential minimal optimization and was proposed by Platt \cite{platt1999fast}. The algorithm requires normalized values. A dynamic normalization algorithm constantly analyzes the sensor data for minimum and maximum values and accordingly calculates the normalized value. There is a variety of different software frameworks for machine learning that support training and recall of SVMs, thus there is no need for reimplementing these methods. As there is an implicit weighting of features according to significance, there is no need to pre-process or weigh the sensor data.

The training data is collected from a set of persons that have a significant variance in body shapes in both height and girth. SVMs support an arbitrary number of different groups for classification. However, the number of significant poses on a chair is limited. The Global Posture Study by office furniture manufacturer Steelcase Inc. analyzes the most common poses with a focus on information consumption from modern technical devices, such as smart phones or tablets \cite{globalPosture}. The different postures are shown in Figure \ref{fig:smartchair_gps_poses}. A capacitive office chair should be able to distinguish most of these poses if training data has been collected from a sufficiently large number of suitable candidates. Additionally, using sensors clearly positioned on a certain side of the chair, e.g. the armrest, it is possible to associate directional varieties of the asymmetric postures.
\subsection{Heterogeneous sensor systems}
\subsubsection{Heterogeneous capacitive arrays}
\begin{figure}[h]
\centering
\includegraphics[width=0.4\textwidth]{images/armrest_dataproc}
\caption{Data processing pipeline of Active Armrest}
\label{fig:armrest_dataproc}
\end{figure}
%Figure 25 Data processing pipeline of Active Armrest
As we already mentioned, the Active Armrest electrodes are put into two groups. The data processing for both groups is distinctly different. In order to detect the presence of the arm using the two-electrode group a simple threshold on the accumulated values is used. The six sensor array in the front (touch area) is using the presented weighted average method to calculate finger positions. Additionally a threshold is used to distinguish one and two fingers. Overall there is a data processing pipeline as shown in Figure \ref{fig:armrest_proto}. The finger tracking and gesture recognition will be inactive until it is ensured that no arm is present. 

\subsubsection{Heterogeneous sensor fusion}
\begin{figure}[h]
\centering
\includegraphics[width=0.7\textwidth]{images/captap_peg}
\caption{Suspended peg knock detection system for CapTap \cite{Braun2013ChairAid}}
\label{fig:captap_peg}
\end{figure}
%Figure 34 Suspended peg knock detection system for CapTap [80]
The hand location of the CapTap is similar to the methods presented for the MagicBox. We add the additional component of knock detection to provide selection events when touching the surface. Figure \ref{fig:captap_sketch} shows a sketch of the knock detection system. The table has a glass plate that is suspended on some rubber supports. In the center of the table we attach a small peg (enlarged in sketch) that creates a connection between the glass plate and a piezo sensor. If the glass plate starts vibrating from a touch we can measure this using the piezo sensor \cite{Braun2013ChairAid}. If a notable vibration is measured we are collecting the next 50 samples, resulting in a window of 250 milliseconds. To distinguish single and double knocks we calculate the weighted average within this window to get a measure for the distribution of sensor values within. If the average is closer to the beginning of the window the resulting event should be a single knock, and a double if the average is closer to the end of the window.
Hand localization and knock detection are working independently and are combined later in the software. It is reasonable to combine this, e.g. to ignore knock events that are occurring without a hand present. They may be indicative of a person doing a strong step close to the table.
\subsection{Image-based processing}
Their ability to detect changes in the electric field over a distance has led to capacitive proximity being regarded as similar to cameras. Smith et al. consequently called their approach electric field imaging, as particularly shunt mode measurements and their constrained electric fields allow applying certain image processing methods, e.g. tomography \cite{smith1999thesis}. They were critical of using similar methods for shunt mode, noting the following statement.
\begin{quote}
Loading mode measurements can be likened
to images formed without a lens, since only one "end" of
each field line is constrained by the measurement. \cite{smith1998electric}
\end{quote}
Nonetheless, loading mode has certain advantages, particularly if all electrodes are in a single plane and we would like to have a higher sensitivity at a distance from the plane it is advantageous if there is no receiving potential nearby. One example for this planar electrode setup is large area gesture interaction devices, e.g. a table that is able to track the position of arms and hands in three dimensions. There is a plethora of image-based object detection and tracking algorithms that can be also used for capacitive proximity sensor data processing. There is a short process that I propose to realize this arm and hand tracking that includes some general steps that can be used to identify a variety of objects \cite{Braun2013captap}. The process is distinguished into four distinct steps:
\begin{itemize}
\item Creating a grayscale image from the acquired sensor data
\item Apply a feature-preserving image upscaling method
\item Find the contours of the present objects according to pixel values
\item Analyze the image moments of the contour areas and fit human arms
\end{itemize} 

\begin{figure}[h]
\centering
\includegraphics[width=0.5\textwidth]{images/proc_im_pixels}
\caption{Pixel array mapped from sensor values}
\label{fig:proc_im_pixels}
\end{figure}

The most challenging aspect of the first step is the low resolution of a reconstructed image. In order to achieve a mid-range distance resolution that allows detecting objects within 30 or 40 cm it is necessary to use electrodes that are sufficiently large. Thus, an example device uses an array of 6x4 sensor electrodes, resulting in an image of only 24 pixels. Typically the sensor values are an integer value in a range between 0 and 15000. Accordingly we can create a single-channel image with a channel depth of two bytes. In our case we use a linear mapping of sensor values to pixel intensities. An exemplary result image of this mapping is shown in \ref{fig:proc_im_pixels} (with enlarged pixels). In this format it is difficult to gather information about the exact position of the arms and thus we need to apply further processing before finding the contours and fitting arm objects.
\begin{figure}[h]
\centering
\includegraphics[width=0.9\textwidth]{images/proc_im_interpol}
\caption{Effect of different upscaling methods on shape, (A) nearest neighbor, (B) bicubic, (C) bilinear, (D) Lanczos4 - shown as thresholded binary images (pixel intensity > 30)}
\label{fig:proc_im_interpol}
\end{figure}

\subsubsection{Acquire and optimize contours}
In order to get the relevant contours of objects in the interaction area we have to apply some further processing. The first step is to enlarge the image using a feature-preserving scaling method. As all sensors are prone to environmental noise I apply some thresholding based on the pixel intensities before looking for contours. The result is an enlarged binary image of black and white pixels. Four different image scaling methods have been tested, nearest neighbor, bilinear interpolation, bicubic interpolation and Lanczos interpolation. Exemplary results are shown in \ref{fig:proc_im_interpol}. The Lanczos interpolation showed the best results, but is most processing intensive. However, since only small images are used it is reasonably fast in this context. The contours are calculated based on those binary images, defined as the borders between black and white regions. For further processing the distribution and the intensities of the pixels within the specified region are used.
\begin{figure}[h]
\centering
\includegraphics[width=0.9\textwidth]{images/proc_im_arms}
\caption{Overhead camera picture of the scene overlaid with live arm and palm reconstruction for one arm (left) and two arms (right)}
\label{fig:proc_im_arms}
\end{figure}

\subsubsection{Palm and arm fitting}
The last step of identifying and tracking the arms is to fit the position and orientation of the palms and arm into the acquired object contours. For this task the image moments within the contours are analyzed. These are certain particular weighted averages of pixel intensities, or a function thereof \cite{hu1962visual}. They can be calculated using the following equation, whereas $j$ and $i$ define the order and $I(x,y)$ is the pixel intensity at a given position. We can use this to calculate the center point $(\overline{x},\overline{y})$, leading to the central moments $mu_ji$ that are required to determine the orientation of the contour as angle $\gamma$.

\begin{equation}
m_ji=\sum_{(x,y)}{I(x,y)x^jy^i}
\end{equation}
\begin{equation}
\overline{x}=\frac{m_10}{m_00}, \overline{y}=\frac{m_01}{m_00}
\end{equation}
\begin{equation}
mu_{ji}=\sum_{(x,y)}{I(x,y)(x-\overline{x})^j(y-\overline{y})^i}
\end{equation}
\begin{equation}
\gamma=0.5\cdot arctan\frac{2\cdot{mu_{11}}}{mu_{20}-mu_{02}}
\end{equation}
 
The center point and orientation are used to calculate the estimated position of the palm of the hands. These points are the basis for the subsequent gesture recognition. Additionally, separate Kalman filters are integrated to smooth the different palm positions and arm orientation. The resulting arm reconstruction and the actual arm position in a photo are shown in \ref{fig:proc_im_arms}. For this image a simple webcam was installed above the table and its output registered to the table position. 
The arm reconstruction so far is mostly used to determine the arm position. Another potential use of the arm orientation is to improve the merging of two hands. While the system can't distinguish from a single sensor if one hand is close or two hands are further away, the presence of two arms in the detection range can be used to build a heuristic that allows determining the overall number of objects. 
\subsubsection{Intensity-based elevation estimate}
A distinct challenge of the capacitive hand tracking is the considerable directional difference in available resolution. While we can use the presented image analysis to track the planar position of the arms over the whole table area of 80cm width and 50cm depth, estimating the elevation of the arm above the table is restricted by the proximity range of the single sensor. Typically the achievable range maxes out at around 35cm, depending on environmental conditions. In a plate capacitor system the distance $d$ is proportional according to size of the plates $A$ and resulting capacitance $C$. Due to the linear mapping of sensor capacitance measurements to pixel intensities $I$ we can use the image moment within a contour $S$ as estimate of the actual capacitance, and calculate the elevation $e$ according to the following equations:

\begin{align}
d&\propto{\tfrac{C}{A}} & S&\propto{\tfrac{m_{00}}{\int{S}}}
\end{align}

The same thresholds discussed in the contour retrieval phase apply to this step, thus leading to discarding objects at a larger distance that are difficult to detect. Starting from this threshold the resulting elevation is normalized according to a maximum threshold for $m_{00}$ that denotes a very close object (such as touch). The actual touch recognition is performed using acoustic methods. 
As previously explained the sensors are prone to environmental influences, thus this just allows to get an estimate of the actual elevation and no absolute distance value.
\subsection{Physiological signals in frequency- and time-domain}
\label{ch:proc_physio}
All underlying physiological processes are based on the transmission of electrical signals, resulting in a number of electrical measurements in the medical domain, such as EEG or ECG. However, these systems rely on a close contact to the skin in order to measure the electrical signals produced inside the body. Remote sensors have to use a different strategy - the effect that many physiological effects are accompanied by muscle movements that result in temporary changes of posture or volume. Capacitive proximity sensors can be used to measure various physiological parameters that are related to movement of different body parts, including internal organs, most notably the heart. Cheng et al. have presented a system that allows measuring motions and shape changes of body parts using capacitive sensors embedded in garment \cite{cheng2010active}. They were able to detect swallowing and breathing rate. One example for an industrial application is non-contact electrocardiogram (ECG) sensing in cars, intended to detect drowsiness in drivers. Using three electrodes it is possible to detect the heart rate or even acquire a full ECG through various layers of clothing \cite{plessey2012ecg}. MacLachlan presented a system that detects the respiratory rate of a person lying on a bed from a distance of up to 50cm using a single electrode and a highly sensitive sensing method based on spread spectrum methods that are commonly used in wireless communication \cite{MacLachlan2004}.

In this section I present two contributions. The first are approaches to detect the respiratory rate using an analysis in the frequency domain, based on readings by a single capacitive sensor applied close to the chest. The second system uses movement sensing of a person lying in a bed to identify sleep phases.

\subsubsection{Respiratory rate}
\begin{minipage}{\linewidth}
\centering
\includegraphics[width=0.8\textwidth]{images/breathing}
\captionof{figure}{Chest movement when breathing in and out}
\label{fig:breathing}
\end{minipage}

The volume changes of the chest while breathing have been a topic of research for a long time \cite{wade1954movements}. If the body of a person is not moving and can be considered at a static distance from a capacitive proximity sensor, the chest movement should translate into a periodically changing sensor value. This system has been developed in cooperation with Sebastian Frank and exemplified using a chair \cite{Braun2013ChairAid}. The breathing rate detection is operating on a single electrode that is placed close to the chest. The basic idea is shown in Figure \ref{fig:breathing}. The surface of the electrode is large, close to the surface and therefore able to pick up the chest movement. Two different methods of data processing are used and fused to get the final breathing rate. Using a fast Fourier transformation the signal is transformed into the frequency space. We are looking for significant signal portions in frequency areas that can be associated to breathing, between $0.1Hz$ and $3Hz$. 

\begin{minipage}{\linewidth}
\centering
\includegraphics[width=0.8\textwidth]{images/smartchair_thread}
\captionof{figure}{Conductive thread electrode integrated into the backrest of a chair}
\label{fig:smartchair_breathing_curve}
\end{minipage}

The above Figure \ref{fig:smartchair_breathing_curve} shows an example of the sensor data curve generated by the conductive thread sensor behind the back of a person. The chest movement is clearly visible as sinusoidal oscillation of the sensor value. If we have a sufficiently stable baseline the zero-crossings can be calculated. However, as this can't be guaranteed in all cases an adaptive baseline should be used that is reconfigured according to changing states of the sitting person. 

\subsubsection{Sleep phase recognition}
Using movement data to detect sleep phases is an unobtrusive method of sleep monitoring. Studies have shown that the magnitude of movement is typically associated to the following phases in decreasing order: wake, stage 1, REM, stage 2, stage 3 \cite{wilde1983movement}. Another method is distinguishing between awake phase, active sleep and quite sleep and takes into account the order of those phases. This information allows to correlate the actual sleep phases with good certainty \cite{salmi86}. Together with Maxim Djakow and Alexander Marinc I have been working on a system to distinguish sleep phases based on movement data acquired from capacitive proximity sensors \cite{Djakow2013movibed}. 
A typical distribution of sleep phases throughout the night is shown in Figure \ref{fig:proc_phys_sleepphase}. It can be easily seen that the sleep is distributed into different cycles, whereas the sleeping person is moving through the different sleep phases until having a REM phase and then going back to deep sleep. If the only available data is body movements it is becoming more difficult to reliably determine the sleep phase.

\begin{minipage}{\linewidth}
\centering
\includegraphics[width=0.6\textwidth]{images/proc_phys_sleepphase}
\captionof{figure}{Example of human sleep phases throughout the night}
\label{fig:proc_phys_sleepphase}
\end{minipage}

The most reliable way to track sleep phases is by using an electroencephalography (EEG); that is measuring the electrical activity of the brain by placing electrodes on the scalp. Various different types of neural oscillations can be distinguished - the most important for sleep phase detection are alpha waves, theta waves, delta waves and sleep spindles. The American Academy of Sleep Medicine (AASM) distinguishes three different phases of non-rapid eye movement sleep (NREM) and REM phase \cite{schulz2008rethinking}. 
\begin{itemize}
\item Stage 1 - occurs mostly in the beginning of sleep. It has slow eye movement, alpha waves disappear and the theta wave appears. 
\item Stage 2 - dreaming is very rare and no eye movement occurs. The sleeper is quite easily awakened. EEG recordings have a tendency for characteristic "sleep spindles"
\item Stage 3 - was previously divided into stages 3 and 4. It is slow-wave sleep (SWS) or deep sleep. Stage 3 used to be the transition between stages 2 and 4 where delta waves began to occur, while delta waves are dominant in stage 4. 
\item REM sleep - is a phase of sleep characterized by random and rapid movement of the eyes. It is considered the lightest phase of sleep and occurs all through the night but gets longer close to morning.
\end{itemize}

Capacitive proximity sensors enable us to detect the presence of suitable object and their relative proximity to the electrode. Consequently a moving object will cause a change of sensor values. If we aggregate these data deviations from an array of sensors we get a reliable measure of objects moving above the electrodes. In the case of our system we can assume that there is a limited number of persons moving on top of the sensors and thus it is possible to associate the sensor values to movement. In the following I will present a suitable method to achieve a reliable detection of the movements of a sleeping person, whereas we are following a similar approach as Salmi and Leinonen \cite{salmi86}.
At any given time t a set of the latest values of all n sensors can be stored as a tuple in the following form: 
\begin{equation}
\overrightarrow{s_t}=\begin{pmatrix}
s_{1_t}\\ 
s_{2_t}\\ 
\vdots \\ 
s_{n_t}
\end{pmatrix}
\end{equation} 

As capacitive proximity sensors are particularly susceptible to external influences, such as temperature, humidity and other electric fields it is necessary to apply filtering on the sensor values. A suitable candidate is a median filter - a low-pass filter method that selects the median object of a sorted set of values, thus discarding outliers and strongly deviating values. This is particularly suited if transmission errors may occur.
If a person is moving on the bed the value of all sensors in detection distance of the moved body parts will change accordingly, the most relevant example in our case being a person moving in its sleep. We can generate a measure of movement intensity by comparing the values at time t with those at time t-1 resulting in:
\begin{equation}
\overrightarrow{d_t}=\left | \overrightarrow{s_t}-\overrightarrow{s_{t-1}} \right | = \begin{pmatrix}
\left | s_{1_t}-s_{1_{t-1}} \right |\\ 
\left | s_{2_t}-s_{2_{t-1}} \right |\\ 
\vdots \\ 
\left | s_{n_t}-s_{n_{t-1}} \right |
\end{pmatrix}
\end{equation}

In subsequent calculations we will use $\overrightarrow{d_t}$ as combined measurement. For distinguishing between wake, active sleep and quiet sleep we are solely interest in the most intense movement. Thus we are testing for the largest value over a set of $m$ samples, generating the value $b_t$.
\begin{equation}
b_t=max(\overrightarrow{d_t1, d_t2, \hdots, d_tm}
\end{equation}

The value $b_t$ is affected by changes in the speed of movement. Therefor as a final step we generate a centered average value of order $2q-1$:
\begin{equation}
\overline{b_t}=\frac{1}{2q-1}\sum_{i=-1}^q{b_{t-i}}
\end{equation}

The resulting value $\overline{b_t}$ allows us to quantify the intensity of movements over a given period. In order to extract an actual body movement from this value we have to quantify a threshold $s(t)$ that is determined by the average of $q$ previous values of $\overline{b_t}$ multiplied with a factor $f$ that has to be evaluated individually for each configuration of bed and sensors. This threshold $s(t)$ allows us to identify a movement $m$ at any time $t$. This behavior is denoted in the following equations:
\begin{equation}
s(t)=\left ( \frac{1}{q}\sum_{i=1}^{q+1}{\overline{b_{t-1}}} \right )\cdot f
\end{equation}
\begin{equation}
m_t=\left\{\begin{matrix}
1,if \; \overline{b_t}> s(t)>\overline{b_{{t-1}}}\\ 
0,else
\end{matrix}\right.
\end{equation}

As previously mentioned it is difficult to determine sleep phases solely by monitoring the movement. Instead following the example of Salmi and Leinonen and distinguish three phases - wake, active sleep and quiet sleep \cite{salmi86}. These are determined by dividing the sleep time into $a$ three-minute epochs $e_{i_a}$ and qualify these as active or quiet by counting the number of movements occurring in those intervals and comparing it to the average amount of movements in all epochs $\overline{e_a}$ determined by the following equations:
\begin{align}
e_{i_a}&=\sum_{e_{i_Start}}^{e_{i_End}}{m_i} & \overline{e_a}&=\frac{1}{n}\sum_{i=0}^n{e_{i_a}}
\end{align}

In consequence we determine the status of any epoch with this final equation:
\begin{equation}
e_{i_a}=\left\{\begin{matrix}
active,if \; e_{i_a}>\overline{e_a}\\ 
quiet,if \; e_{i_a}\leq \overline{e_a}
\end{matrix}\right.
\end{equation}
These active and quiet periods can be semi-autonomously interpreted by humans in order to determine the actual sleep phases. For example initial activity for 20 to 40 minutes followed by a quiet period can be attributed to a person falling asleep. Following quiet phases are a good indicator for deep sleep phases.
 


