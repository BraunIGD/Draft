\chapter{Use cases for capacitive proximity sensors}
\label{ch:usecases}
After having defined the potential application domains for capacitive proximity sensors, in the following section I will evaluate the actual use cases by presenting associated challenges, presenting data processing methods on how to tackle them, and present a number of prototypes that have been created by implementing these methods. This allows to gather the information required to discuss the application of capacitive proximity sensors in smart environments and validate their use in Chapter \ref{ch:eval}. In addition to the application specific challenges, there are also numerous challenges in the implementation phase that are detailed in the descriptions of the associated prototypes.

\section{Use cases and associated challenges}
\begin{itemize}
\item CapDisp
\item Honeyfish
\item GestDisp
\end{itemize}

\section{Processing methods}
\subsection{Sparsely distributed sensor arrays}
\subsubsection{3D location tracking}
 \begin{figure}[h]
\centering
\includegraphics[width=0.6\textwidth]{images/magicbox_data_zaxis}
\caption{Piecewise linear hand distance estimation \cite{Braun2011MultiInputDevice}}
\label{fig:magicbox_data_zaxis}
\end{figure}
%Figure 29 Piecewise linear hand distance estimation [78]
The first data processing step of the MagicBox is the planar localization of the hand, following the weighted average algorithm previously presented. In order to calculate the distance of the hand from the plane we are using a piecewise linear interpolation, that resembles the response curve of a single sensor \cite{Braun2011MultiInputDevice}.
\begin{figure}[h]
\centering
\includegraphics[width=0.7\textwidth]{images/magicbox_data_gest}
\caption{Gesture overview module (left) and gesture recorder (right)}
\label{fig:magicbox_data_gest}
\end{figure}
%Figure 30 Gesture overview module (left) and gesture recorder (right)
An addition of the MagicBox was a generic gesture recognition module based on methods similar to mouse gesture recognition \cite{braun2013capacitive}, albeit adapted for three dimensional locations. The developed debug software allows defining an arbitrary set of potential gestures and adding training data, as shown in Figure \ref{fig:magicbox_data_gest}. The module is looking for matches based on the most recent set of locations. 
\subsubsection{Large-area location tracking}
\subsubsection{Data processing}
Using long wire electrodes may result in considerable noise and influence from outside electric fields. Therefore CapFloor requires preprocessing to reduce the noise and achieve a more robust high-level data processing. The localization uses the weighted average algorithm that has been presented previously. 
\begin{figure}[h]
\centering
\includegraphics[width=0.8\textwidth]{images/floor_shapes}
\caption{Shapes of a standing and lynig person on top of the CapFloor grid}
\label{fig:capfloor_shapes}
\end{figure}
The fall detection is using a time-series analysis of the aggregated values of the sensors that are currently detecting an object. This method is using the assumption that the overall sensor response is roughly equivalent to the shape of the object that is closest to the surface, resulting in a higher capacitance of the overall system, similar to the plate capacitor model. This effect is shown in Figure \ref{fig:capfloor_shapes}. The sum $s$ of all n sensor values $r$ is the closest equivalent to the system capacitance and therefore a viable measure. If the overall value is beyond a certain threshold $v_l$ we can consider a lying person $p_l$.
\begin{equation}
s=\sum^n_{i=0}{r_i}\ \ \ ,\ \ \ p_l=\left\{ \begin{array}{c}
1,\ \ \ s\ge v_l \\ 
0,\ \ \ s<v_l \end{array}
\right.
\end{equation}
In order to increase the robustness this threshold has to be exceeded for a certain amount of time $t_m$. In consequence a fall $f$ is detected if the following equation is 1.
\begin{equation}
f=\prod^{t_m}_{j=0}{p_{l,t_j}}
\end{equation}
\subsection{Model-driven fitting methods}
\subsubsection{Single-body models}
\begin{figure}[h]
\centering
\includegraphics[width=0.7\textwidth]{images/smartbed_proc}
\caption{Data processing components \cite{braun2012context}}
\label{fig:smartbed_proc}
\end{figure}
The different components of the Smart Bed data processing are shown in Figure \ref{fig:smartbed_proc}. Raw sensor data is distributed to three different modules, the calibration which is determining the initial parameters for the sensor data fusion, the drift compensation that alters those parameters according to long term trends and finally the sensor data fusion module that processes the data and does feed it to the occupation \& position detection. Calibration and drift compensation follow the previously presented model \cite{braun2012context}. 
\begin{figure}[h]
\centering
\includegraphics[width=0.7\textwidth]{images/smartbed_cog}
\caption{Calculating centers of pressures and deviation \cite{braun2012context}}
\label{fig:smartbed_cog}
\end{figure}
Occupation and position detection is performed by dividing the two person bed into left and right and individually calculating for each side the total sensor values, assumed center of pressure using weighted average and the standard deviation (Figure \ref{fig:smartbed_cog}). The same calculation is done between the two sides to distinguish where is activity or if one person is lying diagonally.
Using these six intermediate values we can now map various poses. If all activity is on one side and the horizontal deviation is low, we can assume that one person is sitting. We can additionally use the intermediate values to calculate more information, e.g. the exact location a person is sitting at. 
The data processing for the sleep phase recognition is based on detecting the sensor data variations in order to analyze movement. Discriminating between sleep phases using movement is a common approach that has been used in the past \cite{salmi86}. Using a sparse set of sensors it is possible to detect movement by comparing subsequent sensor readings and associate it to different sleep phases using different activity profiles. The system is based on the same prototype as the posture recognition system \cite{Djakow2013movibed}.
\subsubsection{Multi-body models}
\begin{figure}[h]
\centering
\includegraphics[width=0.7\textwidth]{images/smartchair_software}
\caption{Screenshot of the Capacitive Chair application showing the fitted 3D model on the left, posture detection on the upper right and the recognized posture on the lower right}
\label{fig:smartchair_software}
\end{figure}
In Figure \ref{fig:smartchair_software} we can see a screenshot of the Capacitive Chair debug application. On the left side we see a 3D model that is fitted to a chair model according to the current sensor values, in the middle the results of the machine learning module and the recognized posture and on the right side the currently running breathing rate detection as both Fourier analysis and signal deviation analysis.
All processing methods work on filtered and normalized sensor data. The difference in shape, material and size of the electrodes necessitates slight adaptations to noise filtering and data processing. As an example only the conductive thread backrest electrode is used in the breathing rate detection. 
The 3D model is using a simplified human joint model comprised of 13 connected components. Based on the current sensor readings, single parts or groups of components are fitted to the virtual chair. The process is a mix of posture mapping as found in the smart bed and modification of the dynamic links between the single components \cite{Braun2013ChairAid}.
\begin{figure}[h]
\centering
\includegraphics[width=0.7\textwidth]{images/smartchair_thread}
\caption{Screenshot of the Capacitive Chair application showing the fitted 3D model on the left, posture detection on the upper right and the recognized posture on the lower right}
\label{fig:smartchair_thread}
\end{figure}
We use a simple RBF neural network and training data collected by two different persons to match the input from eight sensors to nine potential output postures that are associated to different working situations. An early observation is that certain postures are difficult to distinguish given the limited number of sensors and the similarity of the postures on the rigid chair. Either a higher number of sensors or a more versatile chair could be used that allows gathering additional information required to distinguish the different poses more reliably. 

The breathing rate detection is operating on a single electrode that is integrated into a mesh on the backrest using conductive thread. The setup is shown in Figure \ref{fig:smartchair_thread}. Consequently the surface of the electrode is large and able to pick up the chest movement. Two different methods of data processing are used and fused to get the final breathing rate. Using a fast Fourier transformation the signal is transformed into the frequency space. We are looking for significant signal portions in frequency areas that can be associated to breathing, between $0.2Hz$ and $10Hz$. The second method is to look for zero-crossings of the sensor signal through an adaptive baseline. If a person is breathing in the sensor value will decrease resulting in the signal dropping below the long-term average, and rise above when the person is breathing out. Accordingly the breathing rate can be calculated by counting the zero-crossings.
\subsection{Heterogeneous sensor systems}
\subsubsection{Heterogeneous capacitive arrays}
\begin{figure}[h]
\centering
\includegraphics[width=0.4\textwidth]{images/armrest_dataproc}
\caption{Data processing pipeline of Active Armrest}
\label{fig:armrest_dataproc}
\end{figure}
%Figure 25 Data processing pipeline of Active Armrest
As we already mentioned, the Active Armrest electrodes are put into two groups. The data processing for both groups is distinctly different. In order to detect the presence of the arm using the two-electrode group a simple threshold on the accumulated values is used. The six sensor array in the front (touch area) is using the presented weighted average method to calculate finger positions. Additionally a threshold is used to distinguish one and two fingers. Overall there is a data processing pipeline as shown in Figure \ref{fig:armrest_proto}. The finger tracking and gesture recognition will be inactive until it is ensured that no arm is present. 
\subsubsection{Evaluation}
\begin{figure}[h]
\centering
\includegraphics[width=0.8\textwidth]{images/armrest_proto}
\caption{Active Armrest prototype, left - outside view, right - detail view of electronics}
\label{fig:armrest_proto}
\end{figure}
%Figure 26 Active Armrest prototype, left - outside view, right - detail view of electronics
In order to evaluate the Active Armrest we have built the prototype shown in Figure \ref{fig:armrest_dataproc}. An aftermarket armrest was equipped with an OpenCapSense toolkit. The demonstration application is based on the SenseKit debug software supplied with the toolkit. As of now there is a simple USB connection to a nearby PC.
\begin{figure}[h]
\centering
\includegraphics[width=0.6\textwidth]{images/armrest_eval}
\caption{Active Armrest demo software, left - finger tracker, right - OSM based navigation application}
\label{fig:armrest_eval}
\end{figure}
%Figure 27 Active Armrest demo software, left - finger tracker, right - OSM based navigation application
Figure \ref{fig:armrest_eval} shows a screenshot of the finger tracking application on the left, with a two-finger touch registered on the upper left part of the touch area. It is interfaced with a TUIO \cite{kaltenbrunner2005tuio} based maps application using OpenStreetMap \cite{haklay2008openstreetmap} data. The map is moved around using simple swipe movements of the finger that are directly associated to pan-features of the demonstration application. Zooming is activated by two-finger hold gestures on the upper or lower part of the touch area. We have used public displays of this prototype to get an idea of how easily unaffiliated persons learn to use the system. While the majority agreed on the potential of the application, there have been some reservations regarding the current gesture set, particularly that a closer relationship to smartphone touch screen gestures would be welcome.
\subsubsection{Heterogeneous sensor fusion}
\begin{figure}[h]
\centering
\includegraphics[width=0.7\textwidth]{images/captap_peg}
\caption{Suspended peg knock detection system for CapTap \cite{Braun2013ChairAid}}
\label{fig:captap_peg}
\end{figure}
%Figure 34 Suspended peg knock detection system for CapTap [80]
The hand location of the CapTap is similar to the methods presented for the MagicBox. We add the additional component of knock detection to provide selection events when touching the surface. Figure \ref{fig:captap_sketch} shows a sketch of the knock detection system. The table has a glass plate that is suspended on some rubber supports. In the center of the table we attach a small peg (enlarged in sketch) that creates a connection between the glass plate and a piezo sensor. If the glass plate starts vibrating from a touch we can measure this using the piezo sensor \cite{Braun2013ChairAid}. If a notable vibration is measured we are collecting the next 50 samples, resulting in a window of 250 milliseconds. To distinguish single and double knocks we calculate the weighted average within this window to get a measure for the distribution of sensor values within. If the average is closer to the beginning of the window the resulting event should be a single knock, and a double if the average is closer to the end of the window.
Hand localization and knock detection are working independently and are combined later in the software. It is reasonable to combine this, e.g. to ignore knock events that are occurring without a hand present. They may be indicative of a person doing a strong step close to the table.
\subsubsection{Evaluation}
\begin{figure}[h]
\centering
\includegraphics[width=0.7\textwidth]{images/captap_proto}
\caption{Detail views of the prototype system: left - electrode and sensors, right - knock detection box \cite{Braun2013ChairAid}}
\label{fig:captap_proto}
\end{figure}
%Figure 35 Detail views of the prototype system: left - electrode and sensors, right - knock detection box [80]
The CapTap prototype is integrated into a common living room table. Some photos can be seen in Figure \ref{fig:captap_proto}. On the left side we see the 24 electrodes made of non-etched circuit boards. A sensor is attached to each. The knock detection box with fixation, housing and piezo sensor is shown on the right side.
\begin{figure}[h]
\centering
\includegraphics[width=0.7\textwidth]{images/captap_system}
\caption{Abstracted view of CapTap prototype including capacitive sensing electrodes and knock detection sensor \cite{Braun2013ChairAid}}
\label{fig:captap_system}
\end{figure} 
%Figure 36 Abstracted view of CapTap prototype including capacitive sensing electrodes and knock detection sensor [80]
The overall abstracted layout of the prototype is shown in Figure \ref{fig:captap_system}. The capacitive sensors are con-trolled by three OpenCapSense boards; the knock detection is performed on an Arduino Uno microcontroller board. The data fusion is outsourced to a Mini-PC that can be placed in the table.
Various evaluations have been performed with the CapTap. We have benchmarked the hand localization against the Leap Motion, concluding that the algorithm works reasonably precise in most parts of the interaction area. The next study was a quantitative study of the percentage of correctly recognized knocks, resulting in considerable misattribution of single and double knocks, due to strongly varying knocking styles. However, the presence of any knock was detected with a precision of about $90\%$ \cite{Braun2013ChairAid}. Our main evaluation of the system was concerned with the influence of our knock detection on the overall interaction speed of the system. The results concluded that merely adding the knock detection is not enough but that additionally the interfaces have to be adapted towards capacitive systems \cite{Braun2013ChairAid}.

\subsection{Image-based processing}

\clearpage

\section{Application prototypes}
In the last few years I have designed a number of different prototypes using capacitive proximity sensors in various usage scenarios within smart environments that were realized with the help of numerous students and colleagues. They tackle specific application domains and implement one or more of the data processing methods that have been specified in the previous section. A short overview can be found in Table \ref{tab:prot_listown}. During the next few pages I will describe in detail how the prototypes have been created, how they implement the different data processing methods and outline the results of any technical evaluation and usability study that has been performed.

% Table generated by Excel2LaTeX from sheet 'Tabelle2'
\begin{table}[htbp]
  \centering
  \footnotesize
  \caption{Overview of developed capacitive proximity sensing prototypes}
    \begin{tabularx}{\linewidth}{Xp{3.5cm}Xp{3.5cm}p{3.5cm}}
    \toprule
    \textbf{Name} & \textbf{Description} & \textbf{Application Areas} & \textbf{Measuring Layout} & \textbf{Data Processing} \\
    \midrule
    CapFloor & Capacitive system for indoor localization and fall detection based on electrode grid below the floor. & Indoor Localization & Loading mode, variable number of sensors based on area size & Binary activity association and using geometry for positioning. Monitoring of overall value for falls. \\
    Smart Bed & Capacitive sensors placed below mattress able to determine sleeping postures and breathing rate. & Smart Appliances, Physiological Sensing & Loading mode, four sensors on each side of bed  & Posture fitting using a static model. Fourier analysis for breathing rate recognition. \\
    The Capacitive Chair & Office chair equipped with capacitive sensors to distinguish different typical work postures and stress levels. & Smart Appliances, Physiological Sensing & Loading mode and shunt mode, eight sensors, heterogeneous sensing capabilities & Model fitting using a dynamic model. Fourier analysis for breathing rate detection. Posture recognition using machine learning. \\
    Active Armrest & Heterogeneous system for finger gesture recognition and arm rest identification for automotive applications. & Smart Appliances, Gestural Interaction & Loading mode, heterogeneous layout & Finger positioning using direct calculation. Binary arm presence detection. \\
    MagicBox & Mobile 3D gesture interaction device using an array of electrodes. & Gestural Interaction & Loading mode, six wireless sensor nodes & Geometric detection of hand relative to plane. Adapted mouse methods for gesture recognition. \\
    CapTap & Table capable of detecting 3D gestures and knocks to realize tactile interaction in a living room. & Smart Appliances, Gestural Interaction & Loading mode, 24 capacitive sensors and a single touch detecting microphone & Image-based hand and arm detection. Independent touch detection. Tracking of multiple objects. \\
    \bottomrule
    \end{tabularx}%
  \label{tab:prot_listown}%
\end{table}%

\subsection{CapFloor}
\label{ch:prot_capfloor}
\begin{figure}[h]
\centering
\includegraphics[width=0.8\textwidth]{images/capfloor}
\caption{CapFloor sketch - grid layout of electrodes is placed below a floor layer with sensors attached on the sides}
\label{fig:capfloor_sketch}
\end{figure}

CapFloor is a capacitive system for indoor localization and fall detection that is based on a grid array of sensing electrodes placed below a floor covering \cite{Braun2012CapFloor}. A sketch of the system is shown in Figure \ref{fig:capfloor_sketch}. The grid is comprised of insulated wires that are placed orthogonal to each other. Sensors are placed on two sides of the room. Each sensor is performing loading mode measurements. The system is intended to act as both indoor localization system and fall detector. CapFloor can be placed below any non-conductive material, like wood, tiles and PVC, if the distance between the wires and the floor surface is not too high. It can discriminate between a foot being above an electrode or a whole body. Combining this information from various sensors we are able to get a reliable detection of lying, sitting and standing persons. Using only two sides of the room for sensors it is possible to cut the wires without considerably affecting the signal; allowing easy installation in non-rectangular rooms.
Accordingly CapFloor is able to be used in various application scenarios. Indoor Localization in the home domain can be useful in energy saving and fall prevention by appropriately activating and deactivating the environment lighting. It can also be used in security-restricted areas to detect unauthorized movement. The fall detection should be used in a system that has various levels of escalation. E.g. it is not easy to distinguish between a person doing exercises on a floor and a person that has fallen down. Accordingly the system should query if the person is well and not autonomously call for outside help.

\subsubsection{Evaluation}
\begin{figure}[h]
\centering
\includegraphics[width=0.8\textwidth]{images/capfloor_evaal}
\caption{Floor mats with integrated CapFloor system used at the EvAAL 2011 competition \cite{Braun2012CapFloor}}
\label{fig:capfloor_evaal}
\end{figure}
The CapFloor system was evaluated in the scope of the Indoor Localization Track of EvAAL 2011, where it participated out of competition \cite{chessa_eval}. In Figure \ref{fig:capfloor_evaal} we can see a picture of the demonstration setup installed in the living lap using the system integrated into different mats that are placed in the environment. The system was tuned to detect a single person and was able to perform this reasonably in the areas covered. The resolution of the system is strongly depending on the given density of electrode wires. While there is a certain measure of proximity, it is not possible to detect objects that are more than a few centimeters away from the wires. Later iterations of the system are using higher voltages and shunt mode measurements to improve the tracking reliability and enhance the fall detection.

\clearpage
\subsection{Smart Bed}
\label{ch:prot_smartbed}
\begin{figure}[h]
\centering
\includegraphics[width=0.7\textwidth]{images/smartbed}
\caption{Smart Bed sketch - flexible plate electrode are attached on spring board}
\label{fig:smartbed_sketch}
\end{figure}

The Smart Bed is a regular bed frame that has been equipped with capacitive proximity sensors in order to determine occupation, posture and sleep phases \cite{braun2012context}\cite{Djakow2013movibed}. A sketch can be seen in Figure \ref{fig:smartbed_sketch}. The electrodes are comprised of copper foil that is attached to the flexible wooden panels of the slatted frame. This allows the electrodes to be sensitive to both proximity and applied pressure, resulting in a superposed combined sensor value that is considerably higher as opposed to proximity measure on its own. The electrodes are equally distributed, with four being on both sides of the two person bed. The system is able to determine different sitting and lying postures of one or two persons, including less regular lying positions such as lying diagonal or lying orthogonal to the long side of the bed. Using an analysis of the movement gathered by variation in the sensor signal, the sleep phases can be analyzed, similar to accelerometer-based systems that are popular for smartphones \cite{krejcar2011}.

The Smart Bed can be used for various purposes. A main application is connecting the occupation detection to a home automation system and timer in order to activate ambient lighting if the person is getting up in the night, presumably to find the way to the restroom. Accordingly, in a single person household the lights in unoccupied rooms could be turned off in order to conserve energy. In the domain of personal health the Smart Bed is able to give the user a feedback on sleep quality based on the sleep phase measurement performed in the night. Another potential application is to use the acquired pressure distribution as indicator for back-friendly lying positions that may be harmful over a longer period of time \cite{Hamisu2010}.
The occupation and posture detection relies on a simplified body model to approximate the pressure distribution and sensor values to a certain posture \cite{braun2012context}.  

\begin{figure}[ht]
\centering
\includegraphics[width=0.7\textwidth]{images/smartbed_proc}
\caption{Data processing components \cite{braun2012context}}
\label{fig:smartbed_proc}
\end{figure}

The different components of the Smart Bed data processing are shown in Figure \ref{fig:smartbed_proc}. Raw sensor data is distributed to three different modules, the calibration, which is determining the initial parameters for the sensor data fusion, the drift compensation that alters those parameters according to long term trends and finally the sensor data fusion module that processes the data and does feed it to the occupation \& position detection. Both calibration and drift compensation follow the previously presented model \cite{braun2012context}. 

\begin{figure}[ht]
\centering
\includegraphics[width=0.8\textwidth]{images/disc_unob_bed}
\caption{Electrodes and sensors hidden below mattress of Smart Bed \cite{braun2012context}}
\label{fig:disc_unob_bed}
\end{figure}

The system prototype is shown in Figure \ref{fig:disc_unob_bed}. The positions of the electrodes on the slatted frame are indicated in red. The picture only shows one side of the bed. The same electrode positions are used on the other half of the bed.

The data processing for the sleep phase recognition is based on detecting the sensor data variations in order to analyze movement. Discriminating between sleep phases using movement is a common approach that has been used in the past \cite{salmi86}. Using a sparse set of sensors it is possible to detect movement by comparing subsequent sensor readings and associate it to different sleep phases using different activity profiles. The system is based on the same prototype as the posture recognition system \cite{Djakow2013movibed}.
 
\subsubsection{Evaluation}
\begin{figure}[h]
\centering
\includegraphics[width=0.9\textwidth]{images/smartbed_sleepphase}
\caption{Sleep movement data over three hours in one night \cite{Djakow2013movibed}}
\label{fig:smartbed_sleepphase}
\end{figure}
The Smart Bed posture recognition is able to successfully distinguish eight typical sitting and lying states. Using adaptation of the intermediate values it is possible to fit the state to an actual position on the bed, e.g. a \emph{person sitting on the right side of the bed} state can be modified to any location on that specific side of the bed. 

Regarding the detection of sleep phases there has been an evaluation and benchmarking of three nights \cite{Djakow2013movibed}. The Smart Bed was able to achieve a comparable performance to smartphone applications that detect sleep phases based on accelerometers. Figure \ref{fig:smartbed_sleepphase} gives an example of movement recordings using the capacitive proximity sensors over one night. The activities are grouped into distinct chunks that are later associated to the sleep phases. Currently, breathing rate detection is added to the Smart Bed that can be used to improve the sleep phase detection and also can potentially detect anomalies that may be indicative of a certain health risk.

\clearpage
\subsection{The Capacitive Chair}
\begin{figure}[h]
\centering
\includegraphics[width=0.4\textwidth]{images/smartofficechair}
\caption{Smart office chair sketch - eight electrodes three in backrest, three on seat and two in armrests}
\label{fig:smartchair_sketch}
\end{figure}
The Capacitive Chair is a regular office chair equipped with eight capacitive proximity sensors that can detect different sitting postures and work-related stress levels by examining movement and breathing rate \cite{Braun2013ChairAid}. Seven solid copper electrodes that are placed below the covering are augmented by a single conductive thread electrode that is placed in a mesh on the backrest. In the past smart chairs have used pressure sensors to infer posture and occupation \cite{tan2001sensing}. Combining presence and proximity sensing it is possible to directly infer postures where parts of the body do not touch the surface, e.g. if the body is arched towards the front, or if an arm is raised from the armrests. Additionally higher area electrodes in the backrest allow detecting the breathing rate by measuring the movement of the chest.

The Capacitive Chair aims at providing different services to a typical office worker and office managers. Using the occupation detection it is possible to advise for some type of physical activity, if the time spent in front of the screen was too long. The system can also advise the user to change to a more back-friendly posture or regularly switch the stance to achieve a more general workout. Using the breathing rate detection we are able to get some sort of measure of the current stress level associated to the given working situation. By adapting the environment it is possible to improve the working atmosphere and reduce stress. The Capacitive Chair uses a multifacetted data processing approach. A machine learning algorithm is associating the sensing data to one of nine different typical sitting positions, inspired by a recent study of sitting positions for modern device usage \cite{globalPosture}. An adaptive body model that is fitted to the current sensor values allows for fine grained adaptation of those postures. Finally a combination of Fourier and data variation analysis is calculating the current breathing rate \cite{Braun2013ChairAid}.

\subsubsection{Capacitive layout}
The Capacitive Chair is based on a single OpenCapSense board that supports eight different electrodes. In order to get the posture measurements we need to distribute the electrodes equally on the different areas of the seat. The measurement of the breathing rate requires a larger electrode near the chest area. Consequently the electrodes are placed as follows:
\begin{enumerate}
\item Electrode on the upper part of the backrest (covered by faux leather)
\item Electrode in the central part of the backrest (using conductive thread)
\item Electrode in the lower part of the backrest (covered by faux leather)
\item Electrode below the right armrest
\item Electrode below the left armrest
\item Electrode for the left hip area below the left part of the seat
\item Electrode for the right hip area below the right part of the seat
\item Electrode for detecting both legs below the front part of the seat
\end{enumerate}

\begin{minipage}{\linewidth}
\centering
\includegraphics[width=0.8\textwidth]{images/prot_capchair_electrode_layout}
\captionof{figure}{Capacitive Chair electrode positions}
\label{fig:prot_capchair_electrode_layout}
\end{minipage}

The electrode is connected to channel 0 (CH0) of the OpenCapSense evaluation board. The following figure shows the layout of the electrode (2) including sensing electronics (5). The shield electrode is additionally the support structure for the whole setup here. The shield electrode is comprised of copper sheet bedded in duct tape. On the duct tape there are strips of copper sheet applied using conductive glue (2). The copper sheet is the sensing electrode connected to the sensor (5) using the blue wire (4). The shield electrode is connected using the red wire. The frame of the backrest is indicated using the number (6). 
This electrode in the lower part of the backrest is connected to CH2 of the OpenCapSense evaluation board.
The layout is analog to the one on the upper part of the backrest, comprised of shield electrode (2) covered by duct tape and a copper sheet electrode. The electrode on the right armrest is connected to CH3, the one on the left side to CH4 of the OpenCapSense evaluation board. Both electrodes are comprised of a copper sheet fixed to the armrest using duct tape. 
The electrode below the right hip area is connected to CH5, the one below the left hip area to CH6 and the leg electrode to CH7.
The figure above shows the electrodes. All of them are made of unprocessed, two-layer copper PCBs. They are isolated to the environment using duct tape. The electrode for the leg area is comprised of two distinct PCBs (2,3) that are connected using copper wire (5). The hip electrodes (1) are similarly comprised of copper PCBs. The wires are guided through the wooden seat using small drill holes (4,6). The red wire leads to the sensing electrode while the grey wire leads to the shielding.

\begin{minipage}{\linewidth}
\centering
\includegraphics[width=0.8\textwidth]{images/prot_capchair_threadelectrode}
\captionof{figure}{Detail view of conductive thread electrode}
\label{fig:prot_capchair_threadelectrode}
\end{minipage}

The electrode in the central part of the backrest is connected to CH1 of the OpenCapsense evaluation board.
The electrode (1) is comprised of conductive thread that was woven into the covering of the backrest. The ends of the conductive thread are connected to a conductive copper foil. This foil is formed in such a way (4 left) that a terminal (4 right) can be applied and connected to the sensor electronics (3). This type of electrode does not support any shielding. In order to remove the covering the terminal should be disconnected from the electrode first.
 
\subsubsection{Processing}
The first step of data processing is filtering. We have implemented different types of filters, including static average and floating average filters. In this case we are using a median filter that is taking the median value of eight previous samples. After filtering we are building the baseline for each sensor channel. The baseline is the minimal sensor value that is created by sampling the environment without any object present. There is a plausibility check in this step to discard values that deviate too far from the norm. The maximum values of a channel are collected on run-time. Again we are using a plausibility check. The final step of pre-processing is a normalization based on acquired minimum and maximum values. For further processing we additionally need information about short-term value variance that we gather by calculating the difference quotient using a sample of ten average filtered measurements.
Afterwards we are performing a fast fourier transformation (FFT) to get information about the frequency spectrum of the sensor values, in order to perform breathing rate detection. 
\subsubsection*{Breathing rate detection} 
We are using the FFT values of sensors attached to the central backrest to get the current breathing rate. A binning operation is performed to look for significant signals in a reasonable frequency interval (0.1Hz-3Hz).
In order to increase the reliability of the breathing rate detection we use a second method. Based on the normalized values a mean value curve is calculated. The intersection points of this mean value curve and the current sensor values are additionally stored. We are using a dynamically weighted combination of both values to increase the reliability of the breathing rate detection.
\subsubsection*{Posture recognition, kinematics of the human body}
The processed values of all sensors are compared to previously trained sitting positions of a user. The position with the lowest deviation is considered the current posture. Currently the system supports nine different postures; however it can be dynamically extended or reduced.
Based on the normalized sensor values and geometric positions of the sensors the data is interpreted as position of the different joints of a user. 
4.2.4	Output
The GUI allows displaying of raw and processed data. In the following section we are presenting the different forms of interaction.
 
Figure 13 GUI with four opened windows
Figure 11 gives on overview of the GUI. Selecting the desired output in the ToolBox (1) opens the associated window. In this case we can see the data display of sensor channel 1 (2), the recognized breathing rate (4), the FFT of sensor channel 1 (3) and the recognized postures and their deviations (5).
 
Figure 14 GUI with two windows
Figure 12 shows two additional windows. On the left side (1) we can see a picture depicting the currently recognized posture; on the right side (2) we can see the human model with recognized joint positions. The 3D joint recognition is still in strong development and will be remodeled in the future.
Additional screens that have not been shown in this overview are a serial monitor that displays the raw data acquired from the USB connection, the collection of measurements using software queries, the display of all sensor values in table format and a repositioning of the different windows.
\subsubsection*{Distinguish work activity levels}
 
Figure 15 Work Activity aggregation over a single work day (mock-up)
Figure 13 shows a mock-up of a typical work day activity over a single work day. We assume the work day of a typical office worker and support three different aggregated activities: 
Active work as indicated by a certain level of movement while on the chair
Passive work as being present on the chair while not moving a lot
Not present at desk, whereas no one is currently sitting on the chair.


\subsubsection{Evaluation}
The Capacitive Chair was partially supported by the EIT ICT Labs project Cognitive Endurance during 2013. In this scope it was evaluated in two distinct studies. The first aimed at testing the aggregated recognition of working activities with several persons over various days. The second study was testing the posture recognition with various users that were additionally queried about their general impression of the system. In this section we are presenting results of both studies.
\subsubsection*{Working situation recognition}
The sensing chair supports distinguishing two different working situations that are determined using the method described in the previous section. The system also supports sending to the Cognitive Endurance server.
 
Figure \ref{fig:prot_capchair_eval_work} shows an example of this generated activity log. We have performed a test over 3 days between December 4th 2013 and December 6th 2013 on a typical work day in the office. The resulting activity logs were used to generate a chart as shown in the previous section. An example chart is shown in Figure 15.

\begin{minipage}{\linewidth}
\centering
\includegraphics[width=0.8\textwidth]{images/prot_capchair_eval_work}
\captionof{figure}{Example chart of work activity data collected}
\label{fig:prot_capchair_eval_work}
\end{minipage}	

We can clearly see some phases of not at chair - usually for lunch break or some meetings and the work is distributed between active work, such as writing and typing and longer phases of inactivity (such as reading).

\subsubsection*{Posture recognition - test 1}
In a second evaluation we were testing the posture recognition of the chair in a short study with 10 participants. Our system was tuned to distinguish three poses and a non-pose:
\begin{itemize}
\item Sitting upright
\item Sitting hunched
\item “Slouching on chair”
\item Close to chair - disturber
\end{itemize}

The persons were given a short introduction, the different postures were displayed, and finally the persons were asked to perform the postures in order. When testing “close-to-chair” the subjects were asked to rattle at the chair, stand close, move it around and thus disturb the potential sensor readings. Each class was tested for 10 seconds, collecting 200 samples. Some impressions can be found in the following pictures:

\begin{minipage}{\linewidth}
\centering
\includegraphics[width=0.8\textwidth]{images/prot_capchair_eval_pos1}
\captionof{figure}{Disturber position of a participant (left) and sitting upright (right)}
\label{fig:prot_capchair_eval_pos1}
\end{minipage}	 

\begin{minipage}{\linewidth}
\centering
\includegraphics[width=0.8\textwidth]{images/prot_capchair_eval_pos2}
\captionof{figure}{Slouching position (left) and sitting hunched (right)}
\label{fig:prot_capchair_eval_pos2}
\end{minipage}

Overall the results were very convincing. Of the 40 different measurements series only two were not achieving 100\% accuracy. The Upright and Disturbance positions were classified correctly for all candidates. A single candidate had an 86\% rating on the hunched posture. A different candidate had a 55\% rating on the slouching position. The average of correctly classified postures is 98,5\%.



\clearpage
\subsection{Active Armrest}
\label{ch:prot_armrest}
\begin{figure}[ht]
\centering
\includegraphics[width=0.4\textwidth]{images/active_armrest}
\caption{Active armrest sketch - six electrodes for finger gesture detection in front, two for arm detection in back}
\label{fig:armrest_sketch}
\end{figure}

Touch screens are by now also a trend in vehicles, with touch screens and touch pads becoming more common. The Tesla Model S provides a large area touch screen that completely replaces conventional button-based interfaces.  However, touchscreens have been identified as potentially distracting for the driver \cite{rumelin2013make}. My idea was to create a gesture input device based on capacitive proximity sensors, and unobtrusively integrate it into the car interior \cite{braun2013ActiveArmrest}. A suitable area for creating an interactive zone is the armrest, as it is the intended resting position in the first place. However, this creates an additional challenge. As the majority of interactions between arm and armrest are not intended to control aspects of the car system, we need concepts to infer the intention of the driver to interact with the car. The method has been described previously in Section \ref{ch:proc_hetero}. A sketch of this concept is shown in Figure \ref{fig:armrest_sketch}. To test the validity of the invisible interactive areas and the two interaction concepts, we have created the Active Armrest, a prototype comprised of an aftermarket armrest with an integrated heterogeneous array of eight capacitive proximity sensors - two using large electrodes for detecting arm proximity and orientation and six using small electrodes to create an interaction area in front of the armrest. The system uses the classification method described in Section \ref{ch:proc_hetero}.

\begin{figure}[ht]
\centering
\includegraphics[width=0.8\textwidth]{images/armrest_proto}
\caption{Active Armrest prototype, left - outside view, right - detail view of electronics \cite{braun2013ActiveArmrest}.}
\label{fig:armrest_proto}
\end{figure}

In order to evaluate the Active Armrest we have built the prototype shown in Figure \ref{fig:armrest_proto}. An aftermarket armrest was equipped with an OpenCapSense toolkit. The kit had to be modified due to the constrained interior and uses fixed wiring instead of USB connections. The demonstration application is based on the SenseKit debug software supplied with the toolkit. As of now there is a simple USB connection to a nearby PC. The gesture recognition framework was implemented in Java using the WEKA machine learning framework for SVM classification. A car interior demonstration application was created using Java and the Swing framework and mimics typical menu systems found on a touch screen.

\begin{figure}[ht]
\centering
\includegraphics[width=0.8\textwidth]{images/armrest_postures}
\caption{Postures of limbs on armrest - resting position (left),  arm raised position (middle), hand raised position (right) \cite{braun2013ActiveArmrest}.}
\label{fig:armrest_postures}
\end{figure}

In Figure \ref{fig:armrest_postures} we can see the three different positions arm and hand can have on the armrest. On the left, arm and hand are in resting position with both close to the surface. The middle image shows the arm raised position and fingers touching the front of the armrest. The right picture shows the arm resting on the back and the hand in proximity of the front area. The system is able to distinguish between the three different positions using the methods presented in Section \ref{ch:proc_hetero}. A set of four different gestures has been defined for both interaction methods. The number is sufficient to control the user interface that was developed and supports both navigation and selection. The type of gestures has been defined after looking at previous research into touch and hand gestures \cite{bragdon2011experimental, wachs2011vision}. For the touch interaction, left and right swipes performed with either one or multiple fingers are supported. Regarding the free-air interaction we are using left and right swipes, as well as planar circles either clockwise or counter-clockwise.

\begin{figure}[ht]
\centering
\includegraphics[width=0.8\textwidth]{images/armrest_demo_app}
\caption{Active Armrest demo software, left - finger tracker, right - OSM based navigation application \cite{braun2013ActiveArmrest}}
\label{fig:armrest_demo_app}
\end{figure}

Figure \ref{fig:armrest_demo_app} shows two screenshots of the created demonstration application. It allows to control radio stations, selecting different audio files and looking at images. It is controlled using the gesture sets explained above. The applications use a Next/Previous pattern for navigation within a UI level and Select/Back gestures to switch between the different levels of the UI.

\subsubsection{Evaluation}
\begin{figure}[ht]
\centering
\includegraphics[width=0.8\textwidth]{images/armrest_eval_confustion}
\caption{Confusion matrices of recognized gestures for touch interaction (left) and free-air interaction (right)}
\label{fig:armrest_eval_confustion}
\end{figure}

We performed a study with 11 participants investigating three different aspects - the detection rate of the gesture recognition system, differences in interaction speed between the two methods and getting a general feedback on the usability of our system. After a short introduction, the participants were asked to perform each gesture 4 times for both sets in alternating starting order. The results are shown in Figure \ref{fig:armrest_eval_confustion}. The touch interaction performed reasonably with a detection rate between $79.5\%$ and $90.9\%$ for each gesture. The detection rates of the free-air interaction were considerably lower, ranging from $45.5\%$ for counter-clockwise circles to $81.8\%$ for swipes to the right. The main issue is distinguishing between single and multi-touches. A personalized threshold that is calibrated to the user might alleviate this issue in future iterations. The interaction zone above the finger area is limited to a range of about 15 cm. While performing the circular gestures the participants often left this area, leading to misattribution to swipe gestures.

In the second part of the evaluation the participants had to perform a task in the presented demonstration application - selecting and playing back a certain music file. We calculated the time required to perform the task. The average time for free-air gestures ($\mu=125.67s, \sigma=95.12s$) was considerably higher than the average task completion time for touch gestures ($\mu=34.26s, \sigma=28.61s$). It is noticeable that there is a very high deviation of the different runs, while the touch gestures fare better in general. While many users were able to quickly perform the task, others had a high number of errors and required several minutes. We can assume that a certain amount of training can reduce the required time. A trained user not participating in the study required 11 s for the touch gestures and 18 s for the free air gestures.

Finally, we asked the participants to fill a general questionnaire on their experience with the Active Armrest comprised of a number of Likert-scale (1-10) questions. There was a strong preference for the touch gestures, in line with the results of the interaction time and gesture recognition study (1=touch gestures, $\mu=1.72, \sigma=0.84$). Most participants could imagine using the system for a longer period of time in their cars (10=strong agree, $\mu=8.00, \sigma=2.00$) and considered the device intuitive to use (10=strong agree, $\mu=8.36, \sigma=1.67$) and is an interesting interaction device (10=strong agree, $\mu=8.63, \sigma=1.26$). The touch interaction pattern was considered easier to use (10=very easy, $\mu=8.00, \sigma=2.45$) than the free-air interaction (10=very easy, $\mu=3.64, \sigma=2.68$). The opinions on the device precision were mixed (10=very precise $\mu=7.27, \sigma=2.43$).


\clearpage
\subsection{Magic Box}
\begin{figure}[h]
\centering
\includegraphics[width=0.6\textwidth]{images/magicbox}
\caption{MagicBox sketch - six electrodes uniformly distributed below surface}
\label{fig:magicbox_sketch}
\end{figure}
%Figure 28 MagicBox sketch - six electrodes uniformly distributed below surface
The so-called MagicBox was our first attempt to create an interaction device based on capacitive proximity sensing. It is using an array of six individual wireless capacitive sensors that communicate to a central station \cite{Braun2011MultiInputDevice}. The electrodes are using a large surface area and are made of aluminum foil. A sketch is shown in Figure \ref{fig:magicbox_sketch}. The system is able to track the position of a single hand in three dimensions up to a distance of approximately $20cm$, and uses different methods to infer gestures from the hand movement. 
It is designed to be a generic interaction device that can potentially be hidden below non-conductive surfaces. As it can be used without touching it is also applicable in sterile environments. A suite of demonstration applications has been created that showcase typical scenarios for the MagicBox. This includes multimedia applications, like image viewer and media player but also a 3D object viewer intended as demonstrator for potential medical applications, allowing a surgeon to check MRT or CT images in a sterile environment without touching any surface.

\subsubsection{Evaluation}
\begin{figure}[h]
\centering
\includegraphics[width=0.8\textwidth]{images/magicbox_proto}
\caption{MagicBox conceptual rendering (left) and detail view of electronics (right) \cite{Braun2011MultiInputDevice}}
\label{fig:magicbox_proto}
\end{figure}
%Figure 31 MagicBox conceptual rendering (left) and detail view of electronics (right) [78]
The MagicBox prototype is based on the Cypress First Touch starter kit \cite{cypressfirst} and combines six capaci-tive sensors communicating wirelessly to a single base station, that are put together with a USB-rechargeable power supply into a casing. A concep-tual rendering showing the interaction area and a detail view of the prototype electronics are shown in Figure \ref{fig:magicbox_proto}.
\begin{figure}[h]
\centering
\includegraphics[width=0.8\textwidth]{images/magicbox_eval}
\caption{MagicBox demonstration application - 3D object viewer (left) and image viewer (right) \cite{Braun2011MultiInputDevice}}
\label{fig:magicbox_eval}
\end{figure}
%Figure 32 MagicBox demonstration application - 3D object viewer (left) and image viewer (right) [78]
The different iterations of the MagicBox have been evaluated in conjunction with various demon-stration applications. A usability study with 18 per-sons led to general approval of the system \cite{Braun2011MultiInputDevice}. Two of the applications used in this study are shown in Figure \ref{fig:magicbox_eval}. On the left is a 3D object viewer that has to be controlled by a combination of menu and direct manipulation of the screen content. On the right side there is an image viewer that was controlled by gesture to trigger the next/previous images or perform zooming operations. The most common positive remarks gathered in this study can be roughly put into three groups:
\begin{itemize}
\item{The device very intuitive to use}
\item{The idea of interacting this way is novel and interesting}
\item{It is easy to control applications with those gestures}
\end{itemize}
Likewise we identified three main groups for negative comments about the prototype:
\begin{itemize}
\item{The device is not very precise}
\item{The interaction speed is slow}
\item{It can be tiring for the arm}
\end{itemize}
Later iterations have been trying to improve some of the weaknesses presented above, e.g. by using a more sophisticated gesture recognition system and faster sensor refresh rates. Accordingly there were fewer complaints about interaction speed and precision \cite{braun2013capacitive}. However, the final complaint about the device being tiring for the arm, requires a different approach, that we are investigating in the final prototype to be presented in this system.

\begin{figure}[h]
\centering
\includegraphics[width=0.7\textwidth]{images/magicbox_data_gest}
\caption{Gesture overview module (left) and gesture recorder (right) \cite{braun2013capacitive}}
\label{fig:magicbox_data_gest}
\end{figure}
%Figure 30 Gesture overview module (left) and gesture recorder (right)
The overall method is similar to mouse gesture recognition, albeit adapted for three dimensional locations. The developed system allows defining an arbitrary set of potential gestures and adding training data. In Figure \ref{fig:magicbox_data_gest} the defined gestures can be seen u. The module is looking for matches based on the most recent set of locations. 

In our cases this is an array of capacitive proximity sensors that will be detailed in the prototype section.
Figure 3 - Screenshots of gesture manager and gesture
recorder
The key aspect of gestures by example – providing examples - is realized in a debug application. It provides a simple way to record exemplary movements and associate them to gesture sets. The main screens realizing this functionality are shown in Figure 3.
On the left side we can see the management screen that allows
adding and deleting of gestures, as well as a preview window that is an average of the sample data associated to this gesture. The process of entering data is shown on the right side where several samples can be recorded and associated to the selected gesture and the user can decide, whether the current movement should be stored or discarded.

\clearpage
\subsection{CapTap}
\begin{figure}[h]
\centering
\includegraphics[width=0.7\textwidth]{images/captap_v2}
\caption{CapTap sketch - 24 electrodes placed under table surface}
\label{fig:captap_sketch}
\end{figure}
%Figure 33 CapTap sketch - 24 electrodes placed under table surface
The CapTap is a large area interaction device unobtrusively integrated into a living room table. It is comprised of 24 capacitive sensors and a single sen-sor for knock detection that supports selection events within the demonstration applications [80]. In the domain of free-air gestural interaction there are two prevalent challenges. The physical demands of pro-longed interaction with such systems  is high [81], [82]. Additionally it is difficult to adapt selection events to gestural input. The latter is typically real-ized using time- or position-based gestures [81], [83]. There is no trivial solution to these challenges and any approach has to take into account the specific application scenario covered. Several systems are trying to provide specific GUIs, while others include additional input devices assisting the interaction [84], [85]. CapTap presents an approach to improve the interaction speed of invisible input devices based on capacitive proximity sensors. We have developed a method to unobtrusively detect knocks on a table equipped with a hand tracking system based on capacitive proximity sensors that allows emulating selection events that would typically require an additional time- or movement-based gesture. 


\clearpage
\section{Other capacitive prototypes}
\begin{itemize}
\item CapDisp
\item Honeyfish
\item GestDisp
\end{itemize}
\clearpage
\section{Capacitive prototypes from related work}
While I already have presented capacitive systems in the related work section, they are briefly revisited here, to classify them given the specified application domains. Similar to the created prototypes I will give some additional detail regarding their measuring layout and data processing. The prototypes are shortly listed in Table \ref{tab:related_cap_proto}. This additional information will be used to give a more informed discussion of benefits and limitations of capacitive proximity sensing technology in smart environments and will contribute to the set of guidelines for developers of capacitive sensing applications.

% Table generated by Excel2LaTeX from sheet 'Tabelle1'
\begin{table}[htbp]	
	\centering
  \footnotesize
  \caption{Measuring layout and data processing of different prototypes from related works}
    \begin{tabularx}{\linewidth}{Xp{3.5cm}Xp{3.5cm}p{3.5cm}}
    \toprule
    \textbf{Name} & \textbf{Description} & \textbf{Application Areas} & \textbf{Measuring Layout} & \textbf{Data Processing} \\
    \midrule
    \textbf{SensFloor \cite{lauterbach2009}} & System for indoor localization and fall detection as floor underlay & Indoor Localization & Loading mode, variable number of sensors based on area size & Individual coding of zones on floor - analysis of activity based on trajectories \\
    \textbf{TileTrack \cite{Valtonen2009a}} & Indoor localization using transmitters below floor and receiver electrodes in wall or furniture & Indoor Localization & Transmit mode, large transmitter electrodes below floor, different receiving electrodes & Location by calculating center-of-gravity on most active tiles \\
    \textbf{Touché \cite{Sato2012}} & Swept-frequency sensing to detect different types of touches on a conductive material & Smart Appliances & Swept-frequency sensing, single electrode & SVM classification using features in different frequency ranges \\
    \textbf{Botanicus Interactus \cite{poupyrev2012botanicus}} & Using plant tissue as conductive material as application for swept-frequency sensing & Smart Appliances & Swept-frequency sensing, electrode coupled to plant tissue & SVM classification of touches that are transferred to input events \\
    \textbf{HandSense \cite{wimmer2009handsense}} & Using capacitive sensors to detect grasp style & Smart Appliances & Four loading mode sensors on CapToolKit & Classification of holding styles based on threshold levels \\
    \textbf{Active capacitive sensing \cite{cheng2010active}} & Conductive textile electrodes to sense different parameters of the human body, based on location & Physiological Sensing & Loading mode, single electrode attached to body part & Different filtering methods, based on electrode position, activity classification using LDA \\
    \textbf{Spread spectrum sensor \cite{MacLachlan2004}} & Single electrodes using a spread spectrum technique for improved sensitivity & Physiological Sensing & Loading mode, single electrode placed remotely & Spread spectrum technique to improve SNR, amplitude measurement for respiratory rate \\
    \textbf{School of Fish \cite{smith1999thesis}} & Array of shunt mode sensors that can track 3D position and orientation of two hands  & Gesture Interaction & Shunt mode, flexible array of sensors & Modeling hands as collection of spheres and fit into area based on sensor values and position \\
    \textbf{Thracker \cite{Wimmer2006}} & Four electrodes placed around display that can sense spatial position of hand in front of display and certain gestures & Gesture Interaction & Loading mode, four electrodes placed spatially around display & Position based on distance to electrodes or gesture based on nearest object to electrode \\
    \textbf{Transparent electric field sensor \cite{le2014low}} & Transparent shunt mode array enabling near distance gesture interaction above mobile devices & Gesture Interaction & Shunt mode, four receiver electrodes & Positioning based on single proximity values and random decision forest learning \\
    \bottomrule
    \end{tabularx}%
  \label{tab:related_cap_proto}%
\end{table}%

\subsection{Indoor localization}
One example system based on capacitive sensing is the previously presented TileTrack that uses a combination of transmit mode and center-of-gravity calculation between different floor tiles to calculate the position of multiple persons \cite{Valtonen2009a}. They are using special floor tiles comprised of chipboard and a steel layer. The steel layer acts as transmitting electrode that is coupled to persons moving above it. The initial system used wire and plate electrodes that were integrated into a wall and running to a height of 1.90 m, thus ensuring receiving the field transmitted by tall persons. In order to properly localize a person the readings of multiple tiles relative to a single receiving electrode are combined. Similar to CapFloor (\ref{ch:prot_capfloor}) only passive materials are used below the floor level and the measurement systems can be hidden on the sides, however it requires large tile electrodes, thus it is mostly suited for hidden floor systems.

A second, already commercialized system is SensFloor that uses an integrated solution of capacitive sensors and wireless communication hidden below a floor covering that is able to detect the position of several users and other parameters such as falls, based on analyzing activity above single sensor areas or the movement trajectories over time \cite{lauterbach2009}. The sensors work in loading mode and track activity in specific areas that are bordered by a sensing wire circle. Eight areas are attached to a single sensor, comprising a rectangular tile. The system is designed as a floor underlay, with power supplies being attached on the side and the single sensors communicating wireless to a base station. While this setup allows a high resolution, it is fairly expensive to build and potentially difficult to maintain. 

\subsection{Smart Appliances}
Sato et al. have presented Touché, a swept-frequency capacitive sensor that allows distinguishing different types of touches on any suitable surface and medium \cite{Sato2012}. Some examples include recognizing different hand postures in liquids and touching different body parts to control mobile devices. Their system is based on analyzing a broader range of frequencies that have a different effect on the resulting capacitance. Using a classification method they are able to distinguish different categories of events. It supports an arbitrary number of electrode categories, yet only one single sensor attached to the measurement circuit. One additional feature is placing the electrode next to a conductive medium, such as water. Another example of this technology is touching different parts of a plant to control an interactive art installation  \cite{poupyrev2012botanicus}, showing the versatility of this sensing technology.

HandSense by Wimmer and Boring uses four loading mode electrodes placed around a box mimicking a mobile phone to detect different grasping styles. The electrodes are made of tin sheet and sized 30x15 mm and placed on both small sides of the plastic box. Using different thresholds for "No proximity", "Near", "On hand", "Gripped" and "Held" they are able to classify six different types of holding the phone with varying accuracy. The basic idea is to use the grasping style as additional input device, e.g. by switching thumb-based interface items, when it is held in different hands.

\subsection{Physiological sensing}
Capacitive proximity sensors can be used to measure various physiological parameters that are related to movement of different body parts, including internal organs, most notably the heart. Cheng et al. have presented a system that allows measuring motions and shape changes of body parts using capacitive sensors embedded in garment \cite{cheng2010active}. They combined four sensors based on loading mode that were exposed to electrodes made from conductive thread. They were attached to different body parts, including chest, wrist, upper legs and neck. According to position different forms of activities can be recognized, such as locomotion for the leg electrodes or swallowing for the neck electrodes. Additionally, the neck system could be used to detect the respiratory rate for some individuals. 

\begin{figure}[ht]
\centering
\includegraphics[width=0.8\textwidth]{images/spread_breath}
\caption{\emph{Left:}Spread spectrum capacitive sensor detecting respiration 30 cm (green) and 90 cm (blue) from chest. \emph{Right:} Same sensor detecting person walking by at 100 cm (left), 70 cm (middle) and 35 cm (right) \cite{MacLachlan2004}}
\label{fig:spread_breath}
\end{figure}

MacLachlan presented a system that detects the respiratory rate of a person lying on a bed from a distance of 30 cm using a single electrode and a highly sensitive sensing method based on spread spectrum methods that are commonly used in wireless communication \cite{MacLachlan2004}. Using a single electrode of 14x56 mm and the coding method he is able to detect very small capacitance changes. Figure \ref{fig:spread_breath} shows on the left the changes in capacitance caused by chest movement at a distance of 30 cm, respectively 90 cm from the moving chest. The response of three different sensors to a person passing at a distance of 100 cm, 70 cm and 35 cm can be seen in Figure \ref{fig:spread_breath} on the right. The system requires coding hardware next to the electrode, thus making it more complicated to manufacture, as opposed to the previous systems.

\subsection{Gestural interaction}
The School of Fish by Smith, is a shunt mode system that he developed in the scope of his PhD thesis and the basis for most of the capacitive sensor research at MIT in the 1990s \cite{smith1999thesis}. Comprised of a collection of hexagonal copper electrodes distinguished into senders and receivers. Larger arrays are able to detect position and orientation of two hands moving at a distance above it. Though it was not used for the school of fish the system could easily placed below non-conductive materials, e.g. integration into a table. The hands are modeled using a collection of spheres, enabling a calculation of the orientation.

Thracker by Wimmer et al. uses four electrodes placed around a display to detect the movement of a hand in proximity \cite{Wimmer2007a}. The electrodes are angled away from the screen to improve the resolution at a distance. Two different modes were evaluated. One tracked the position of the hand in front of the screen, while the second one investigated pick and drop gestures. 

A recent work by Le Goc et al. is based on the dedicated GestIC capacitive gesture tracking microchip \cite{le2014low}. Using transparent electrodes in shunt mode they are able to precisely track the 3D position of fingers moving in the electric field. The electrodes are based on an ITO layer on acrylic glass placed above a mobile phone. A second ITO layer is used to shield from interference generated by the capacitive touch screen of the phone. They use a random decision forest method to detect the position of a finger.


 


\section{Discussion}
In this chapter I have identified several distinct use cases for capacitive proximity sensors in smart environments and their associated challenges. Based on this, new and improved processing methods in five different areas have been proposed. These are sparsely distributed sensor system, model-based approaches, heterogeneous sensor systems, image-based processing and physiological sensing. The methods have been implemented and evaluated using a set of six prototypes, that are suited for one or more application areas. This has been put into context with some additional collaborative prototypes and systems known from literature. The rationale of designing the sensor layout and the different processing methods are detailed in the specific sections. Overall, this allows to get a knowledge base that enables me to evaluate the applicability of capacitive proximity sensors in smart environments. The next chapter will describe this evaluation, that is based on a comparison to the other introduced sensor technologies and the specific benefits and limitations of capacitive proximity sensors. The primary purpose is to develop a set of guidelines for parties interested in developing applications for smart environments that are based on capacitive proximity sensors.