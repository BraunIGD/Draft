\section{Applications in smart environments}
The field of smart environments is not strictly and conclusively limited and distinguished from other fields in technology, using influences from disciplines including electric engineering, behavioral psychology, computer science or mechanical engineering. Accordingly, it is difficult to formally list or distinguish all applications that are relevant or have been tackled in previous work. Thus, I will refer to previous collections of surveys, books and state-of-the-art in the associated disciplines smart environments, ambient intelligence and ubiquitous computing to get an informed selection of relevant applications that could potentially be supported by capacitive proximity sensors. The chosen collections of applications are taken from Cook et al. that presented a survey on recent developments in smart environments research in 2007 \cite{cook2007smart}. Augusto et al. edited a book on ambient intelligence in 2009, including chapters on various domains and applications. The final source is the book \emph{Ubiquitous Computing} by Poslad, released in 2011 \cite{poslad2011ubiquitous}.

The applications presented in those works are analyzed and grouped into a limited set of higher level applications. With this grouping I have tried to emphasize the technological similarities of the different applications in preparation for the later benchmarking process. The results can be found in Table . On the following pages the different groups will be introduced with some examples and specific information on the sensor measurements that are required. The determined groups are the basis for further references to different use cases and will be referred to throughout this work.
\subsection{Localization}
\subsection{Activity Recognition}
\subsection{Smart Appliances}
\subsection{Human-Computer Interaction}
\subsection{Physiological Sensing}
\subsection{Augmented and Virtual Reality}


