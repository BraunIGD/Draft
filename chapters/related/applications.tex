\section{Applications in smart environments}
The field of smart environments is not strictly and conclusively limited and distinguished from other fields in technology, using influences from disciplines including electric engineering, behavioral psychology, computer science or mechanical engineering. Accordingly, it is difficult to formally list or distinguish all applications that are relevant or have been tackled in previous work. Thus, I will refer to previous collections of surveys, books and state-of-the-art in the associated disciplines smart environments, ambient intelligence and ubiquitous computing to get an informed selection of relevant applications that could potentially be supported by capacitive proximity sensors. The chosen collections of applications are taken from Cook et al. that presented a survey on recent developments in smart environments research in 2007 \cite{cook2007smart}. Augusto et al. edited a book on ambient intelligence in 2009, including chapters on various domains and applications. The final source is the book \emph{Ubiquitous Computing} by Poslad, released in 2011 \cite{poslad2011ubiquitous}.

The applications presented in those works are analyzed and grouped into a limited set of higher level applications. With this grouping I have tried to emphasize the technological similarities of the different applications in preparation for the later benchmarking process. The results can be found in Table . On the following pages the different groups will be introduced with some examples and specific information on the sensor measurements that are required. The determined groups are the basis for further references to different use cases and will be referred to throughout this work.
\subsection{Localization}
Reliably localizing and tracking multiple users is one of the main challenges of smart environments. The position of a user is important for systems that acquire contextual information in periodic intervals. For example in order to decide whether or not they are supposed to influence the actual state of their environment via available actuators. In many cases basic motion sensors are able to deliver sufficient information, if a single person should be followed. The tracking of multiple persons typically requires more sophisticated solutions. This is equally im-portant when the system needs to distinguish be-tween users and non-critical actors in the environ-ment, such as pets.
Various indoor tracking and localization ap-proaches have been proposed in conjunction with Ambient Intelligence. There are even specific com-petitions with the intention of comparing the dif-ferent methods’ performances against one another [47]. Three different categories of localization methods can be distinguished, active marker-based solutions, passive marker-based solutions, and marker-free solutions. Both active and passive marker-based solutions require a person to carry some type of tag in order to enable localization. For a number of reasons make this type of solutions less favorable. The cost is higher and some users have the tendency to forget the tags. Marker-free solutions are capable of localizing persons inde-pendently of whether they are carrying additional accessories. Examples for this latter category in-clude using microphones for the detection of subtle noises caused by movement [1], or camera-based approaches [48]. The three main criteria used to compare these localization solutions are the total provisioning costs per area, their reliability, and the amount of persons that can be tracked and distin-guished. One example system based on capacitive sensing is the previously presented TileTrack that uses a combination of transmit mode and center-of-gravity calculation between different floor tiles to calculate the position of multiple persons [28]. A second, already commercialized system is SensFloor that uses an integrated solution of capac-itive sensors and wireless communication hidden below a floor covering that is able to detect the position of several users and other parameters such as falls, based on analyzing activity above single sensor areas or the movement trajectories over time [10].

\subsection{Activity Recognition}
\subsection{Smart Appliances}
Smart appliances are devices that are attentive to their environment [49]. This is usually achieved by integrating different sensors and actuators to pro-vide additional functions and services to a user. Some examples include intelligent furniture that can detect their occupation, internet-connected household items, or single-purpose devices, e.g. providing reminder services. An overview of dif-ferent examples was created by Park et al. [50]. One recent example of smart furniture is a system for object localization using smart drawers and RFID technology presented by Nickels et al. [51]. Another example is the usage of various simple sensors that can be easily integrated into furniture to provide activity recognition [52]. Sato et al. have presented Touché, a swept-frequency capacitive sensor that allows distinguishing different types of touches on any suitable surface and medium [53]. Some examples include recognizing different hand postures in liquids and touching different body parts to control mobile devices. Their system is based on analyzing a broader range of frequencies that have a different effect on the resulting capaci-tance. Using a classification method they are able to distinguish different categories of events. An-other example of this technology is touching dif-ferent parts of a plant to control an interactive art installation  [54]. Capacitive sensing provides the ability to add interactive features to many different appliances and allows for unobtrusive placement.
\subsection{Human-Computer Interaction}
Gesture recognition enables the detection of meaningful expressions of motion by a human body, including the hands, arms, face, head and body [57]. If these expressions are translated into machine commands the result is gestural interac-tion. The most expressive and explicit form of ges-tures are performed by the hands, further distin-guished into free-air gestures and touch gestures that typically involve one or more fingers interact-ing with a surface, the latter being called multi-touch.
Apart from the capacitive touch technologies shortly presented in section 2.2, there are also nu-merous other means of realizing finger tracking on surfaces. Jeff Han showed a low-cost system based on frustrated total internal reflection (FTIR) of in-frared light. This system allows tracking ten or more objects in real-time on large surface areas [58]. Acoustic systems are another popular tech-nology in this domain. Surface acoustic wave (SAW) uses the signal decrease of ultrasonic waves as they pass through an object touching the surface to infer its location  [59].  
Throughout the years there have been various at-tempts to enable the tracking of gestures in free air. Capacitive proximity sensors have been first pre-sented almost 100 years ago by the Russian physi-cist Leon Theremin, who invented the eponymous touch-free electronic instrument [60] The theremin uses two electrodes to control pitch and volume of a generated sine wave. Capacitive hand tracking has been a research interest at MIT in the 1990s [13] and has been investigated recently by other groups, enabling touch control even through thick-er non-conductive materials [32], [61]. Various systems use optical tracking using visible light that may either rely on markers [62] or use methods based on discovering the hands without markers, e.g. by detecting the skin color [63]. 
In the last few years there have been two com-mercially successful and popular systems that track gestures using infrared light for depth sensing. The Microsoft Kinect is using an infrared projector and camera to enable whole body gesture tracking sens-ing of multiple persons at a longer distance [64]. The second device is the Leap Motion that allows fine-grained tracking of fingers and hands in a smaller area above the device. It is based on two cameras and infrared diodes that are illuminating the interaction area [65].

\subsection{Physiological Sensing}
We previously introduced bioimpedance as the response of a living organism to an external electric field [15], caused by the human body being mostly composed of ionized water. Capacitive proximity sensors can be used to measure various physiologi-cal parameters that are related to movement of dif-ferent body parts, including internal organs, most notably the heart. Cheng et al. have presented a system that allows measuring motions and shape changes of body parts using capacitive sensors em-bedded in garment [55]. They were able to detect swallowing and breathing rate. One example for an industrial application is non-contact electrocardio-gram (ECG) sensing in cars, intended to detect drowsiness in drivers. Using three electrodes it is possible to detect the heart rate or even acquire a full ECG through various layers of clothing [56]. MacLachlan presented a system that detects the respiratory rate of a person lying on a bed from a distance of up to 50cm using a single electrode and a highly sensitive sensing method based on spread spectrum methods that are commonly used in wire-less communication [12].
Generally, capacitive proximity sensing is a powerful technology that is able to gather physio-logical information over a distance, while being unobtrusively integrated into various appliances. In applications that require this information, e.g. to detect the state of alertness or fitness of a user it is a viable alternative to body-worn sensors that are more intrusive by nature.

\subsection{Augmented and Virtual Reality}


