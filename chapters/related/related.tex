\chapter{Related Work}
\label{ch:related_work}
In this chapter I will introduce the most relevant literature that inspired this work, or that is required to introduce a specific topic. The aim of this chapter is to provide a basis for both, the benchmarking model that is developed in Chapter \ref{ch:benchmark}, and the capacitive proximity sensing methods described in Chapter \ref{ch:usecases}. The related works are distinguished into four parts. At first, I will give a general introduction to electric field sensing, including a discussion on different properties, physical background, the influence of materials and geometry, as well as different data processing methods. Following that, I will  present relevant applications using capacitive proximity sensing, ranging from historical works to recent systems. In the next section, various sensing technologies are introduced that are used in typical smart environment applications. Finally, I will identify and group different applications in smart environments, providing a basis for the applications covered by the benchmarking model.
\section{Electric field sensing}
Different electric charges apply either a repelling or attracting force to each other. For any point in space these forces have a distinct direction and magnitude. The resulting collection of force vectors is called the electric field. Conductive objects that are present in this area modify the properties of the field. Electric field sensing enables measuring field properties at a certain point in space. Using continuous monitoring it is possible to gather information about conductive objects passing through the field by associating measured disturbances to properties of the object. It is possible to gather a multitude of different information about a project. In this section I will give an overview of the physical background, different measurement modes and how to process data acquired by digital sensors.
\subsection{Physical properties}
A complete overview about the electrostatic principles of capacitive proximity sensing can be found in the book by Baxter \cite{Baxter1996}, chapters 2 and 6. We will give a very brief introduction to this topic in the following section.
\begin{figure}[h]
\centering
\includegraphics[width=0.4\textwidth]{images/cap_blackbox.png}
\caption{Black box setup of a capacitive proximity sensor}
\label{fig:cap_blackbox}
\end{figure}
The basic setup of a typically used sensor is shown in Figure \ref{fig:cap_blackbox}. The proximity capacitance \(C_{x}\) can be determined using a combination of serial and parallel circuits of capacitors, resulting in the following equation:
\begin{equation}
C_{x}=\left(\left(C_{hb}+\frac{C_{h}C_{b}}{C_{h}+C_{b}}\right)^{-1}\frac{1}{C_{fe}}\right)^{-1}
\end{equation}
Additionally there are parasitic capacitance components, i.e. disturbing capacitance values within the system. Sources are:
\begin{itemize}
\item Sensing electrode capacitance
\item Capacitance between sensing electrode and ground plane
\item Intercapacitance between neighboring traces on the board
\end{itemize}
The present parasitic capacitances \(C_{par}\) amount to values approximately between \(10pF\) and \(300pF\) and are therefore considerably larger than the value of the proximity capacitance \(Cx\), being between \(0.1pF\) and \(10pF\). The total capacitance sensed is the sum of parasitic and proximity components. 
\begin{equation}
C_{S}=C_{X}+C_{par}
\end{equation}

It is obvious that this parasitic capacitance is considerably higher than the capacitance induced by an approaching object. However, this parasitic capacitance is typically static and can therefore be calibrated in a way not affecting the measurement. 
\begin{figure}[h]
\centering
\includegraphics[width=0.4\textwidth]{images/cap_procedure.png}
\caption{Capacitive sensing procedure}
\label{fig:cap_procedure}
\end{figure} 
Now we will shortly discuss how we can estimate the capacitance of common objects that approach the sensor. Any object exhibits capacitance in respect to infinity. Surveying simple geometric shapes this capacitance is analytically determinable, e.g.:
\begin{equation}
C=8\epsilon_{0}r_{Disk}
\end{equation}
\begin{equation}
C=4\pi\epsilon_{0}r_{Sphere}
\end{equation}

\(\epsilon_{0}\) is the vacuum permittivity and \(r\) the respective radius. This free space capacitance is increasing as soon as another object is approaching, caused by the capacitance of this second object, resulting in mutual capacitance. Looking at generic formulas, determining capacitance between parallel plates this behavior can be described analytically.
\begin{align}
C&=\frac{Q}{V} & C&=\epsilon_{0}\epsilon_{r}\frac{A}{d}
\end{align}
The capacitance is directly proportional to the plate area \(A\) and inversely proportional to the distance d between the plates, with \(\epsilon_{r}\) being the relative static permittivity of the dielectric between the plates. Sensor electronics are grounded with the body acting as ground itself. The sensor plate is continuously charged using a constant voltage \(V\). A higher capacitance allows the system to hold a larger charge. If the system is connected to the ground, the sensor capacitor is discharged through a resistor. The resulting voltage is depending on the available charge, shown in the equation above. Furthermore the required time to discharge the capacitor is increased. This process is symbolized in Figure \ref{fig:cap_procedure}.

\subsection{Proximity sensing versus touch sensing}
\begin{figure}[h]
\centering
\includegraphics[width=1.0\textwidth]{images/cap_projected_sensing_methods.png}
\caption{Different projected capacitive sensing methods based on distance}
\label{fig:cap_proj_sensing_methods}
\end{figure} 
The most ubiquitous usage of capacitive sensing technology can be found in touch screens. As the trend went from pen-controlled mobile systems to finger controlled devices with the first iPhone in 2007, projected capacitance touch is the most prevalent technology for touch screens. It uses various layers of transparent electrodes or nanowires to detect the mutual capacitance as objects enter the detection area \cite{Barrett2010}. The commercially available devices have gained additional abilities over the last few years, leading to the development of “floating touch” systems that are able to track fingers in gloves or fingers that are hovering above the surface \cite{Cypress2012, Nokia2012}. Applications are the usage of mobile devices in cold outdoor temperatures or additional navigation fea-tures based on the hovering fingers. In consequence we can distinguish the three different projected capacitive sensing methods as shown in Figure \ref{fig:cap_proj_sensing_methods}:
\begin{itemize}
\item Touch sensing - densely distributed sensors are tuned to project a weak electric field in order to detect one or more objects touching the interactive surface. The sensors have to be close to the surface.
\item Floating touch - densely distributed high-sensitivity sensors are able to detect both touches and very near objects (\(<2cm\)) to enable usage using protective gear or additional navigation feature. The sensors have to be close to the surface.
\item Proximity sensing - sparsely distributed sensors create a stronger electric field that propagates into space in order to detect larger objects, such as hands, that are in proximity of the interactive surface. Achievable distances are up to 30 centimeters and the sensors may be applied below thick non-conductive material.
\end{itemize}
\subsection{Measuring modes}
\begin{figure} [h]
\centering
\includegraphics[width=0.6\textwidth]{images/cap_sensing_modes.png} 
\caption{Three measurement modes for capacitive proximity sensing \cite{Smith1996a}}
\label{fig:cap_sensing_modes}
\end{figure}
A classic work in the field of capacitive proximity sensing that will be referenced occasionally in this work is “Electric Field Imaging” by Joshua Smith \cite{smith1999thesis}. One contribution was the introduction of different measurement modes in capacitive sensing, as shown in Figure \ref{fig:cap_sensing_modes}. 
Transmit mode is using a transmitting electrode that is coupled to a conductive object; in case of interaction applications typically the human body. The properties of an electric field generated with respect to a receiving electrode will therefore be dependent on the distance of this body, thus extending the achievable range.
Shunt mode similarly uses both a receiving and transmitting electrode generating a static field. However, there is no body coupled and any conductive object will ground the field, thus reducing the energy stored, which is measured. This setup is able to work with various transmitters on a single receiver, enabling a higher amount of virtual sensors using limited hardware. The third measurement mode is called loading mode. An oscillating field is induced on a single electrode measuring the capacitance relative to the environment. Any approaching grounded object results in an increased capacitance that is measured periodically.
\subsection{Materials and geometry}
Two major factors that have to be considered when designing an application based on capacitive sensors are the materials and geometry of the electrodes performing the measurements. The material of the electrode should be picked according to the desired application, i.e. if the interaction device has a flexible surface, conductive thread could be used, if it is solid and opaque, the application of solid metal electrodes is viable. Additionally there are other options for transparent materials. While we traditionally associate solid metals to antennas and electrodes this view can no longer be upheld. Transparent conductive layers have been in use for decades now, e.g. in car windows or solar technology. They typically rely on metal oxide layers, polymer layers or in recent years carbon nanotubes \cite{Moon2005}. The most common technology for usage in displays is projected capacitive touch that uses a multi-layer design of insulated ITO electrodes that are able to detect the movement of several objects close to the surface \cite{Barrett2010}. However, they are typically tuned to allow operation within a small distance of \(1cm\) or less. However, they are typically tuned to allow operation within a small distance of \(1cm\) or less. One recent work was evaluating different types of electrode materials in terms of their spatial resolution at different distances between object and electrode \cite{grosse2013opencapsense}, focusing on larger distance proximity measurements. They benchmarked both ITO and PEDOT:PSS. The first is a thin layer of indium-titanium-oxide, a highly conductive metal layer that possesses good optical properties. PEDOT:PSS is a conductive polymer that has a lower conductivity and slightly less appealing optical properties. In conclusion they evaluated that while copper has still the best properties, at least ITO can be considered a suitable alternative in applications that require optical clarity, as shown in the achievable spatial resolution given in Figure \ref{fig:cap_spatial_resolution}. 
\begin{figure} [h]
\centering
\includegraphics[width=0.6\textwidth]{images/cap_spatial_resolution.png} 
\caption{Spatial resolution of different materials at various distances \cite{grosse2013opencapsense}}
\label{fig:cap_spatial_resolution}
\end{figure}
The most common technology for usage in displays is projected capacitive touch that uses a multi-layer design of insulated ITO electrodes that are able to detect the movement of several objects close to the surface \cite{Barrett2010}. However, they are typically tuned to allow operation within a small distance of 1cm or less. 
Another area that is strongly influenced by the intended application is the geometry, whereas the electrode is considered the part of the electronics directly attached to the measurement circuit. This may range from simple straight wires or plate electrodes to complex optimized multidimensional structures specifically designed for a single task. Even though it is aimed at touch or near-proximity sensing we will give a short overview of  multi-layer designs for touch screens that have been reviewed by Barrett and Omote \cite{BarrettScreen}. They are designed to measure mutual capacitance, i.e. the resulting capacitive properties between a sending and a receiving electrode that are intersecting. If a sensible excitation and measuring process is used, multiple nearby objects may be reliably detected. 
\begin{figure} [h]
\centering
\includegraphics[width=0.8\textwidth]{images/ito_multilayer.png} 
\caption{Examples of multilayer layouts for touch screens - grid (a), interlocking diamonds (b) and  trademarked complex patterns (c) \cite{BarrettScreen}}
\label{fig:ito_multilayer}
\end{figure}
A simple example is two layers of perpendicular straight line electrodes - used by the first iPhone (Figure \ref{fig:ito_multilayer} - a). Another example uses an interlocking diamond shape \cite{Dietz2001a} to create a good spatial coverage (Figure \ref{fig:ito_multilayer} - b). Finally, there are numerous other complex patterns that are often trademarked by the companies that have developed the respective controller. One example is given in (Figure \ref{fig:ito_multilayer} - c). 

Capacitive proximity sensing applications are typically less concerned about intricate designs, but instead use varying electrode sizes and placement over a larger area. As previously mentioned the purpose of capacitive proximity sensing is the detection of objects and their properties. There are numerous factors that can influence the geometrical layout, but they can be abstracted into the following categories:
\begin{itemize}
\item	Number of objects
\item	Object size
\item	Desired spatial resolution
\end{itemize}
Going back to our example of touch screens, we have small objects, a higher number of those (usually up to 10) and require a high spatial resolution to select small items on the screen. The result is a fine multilayer grid, using mutual capacitance to simplify multi-object recognition, fine electrode spacing to achieve a high spatial resolution and thin or transparent electrodes to guarantee good optical properties. A similar rationale can be applied to other applications. If we take the smart couch by Große-Puppendahl et al. the aim is to detect the presence and posture of one or more persons on a couch \cite{Couch2011}. This necessitates detecting large body parts such as head, torso or limbs. There is no fine-grained spatial resolution required, allowing a reduction the number of sensors and it was assumed that a maximum of two persons are on the couch. Furthermore the electrodes are placed below the upholstery, thus requiring a reasonable detection distance. 
\begin{figure} [h]
\centering
\includegraphics[width=0.7\textwidth]{images/couch_electrodes.png} 
\caption{Electrode placement below upholstery \cite{Couch2011}}
\label{fig:couch_electrodes}
\end{figure}
The resulting electrode placement can be seen in Figure \ref{fig:couch_electrodes}. The layout was designed under the additional constriction of using a single sensor kit, supporting up to eight electrodes. Regarding placement it is most important to distinguish two persons and different sitting positions, thus four electrodes are placed below the sitting area. In the back there are two electrodes spread over the entire width to determine the presence of the upper body close to the backrest. The electrodes in the armrests determine a head and are primarily suitable for detecting lying positions. In consequence this setup is suitable for detecting multiple sitting persons, infer information about their sitting position and recognize lying persons. Regarding those postures it showed good results in the prototype's evaluation \cite{Couch2011}.
 
A third and final example for the rationale of electrode placement is the TileTrack system by Valtonen et al, a capacitive person tracking system using floor tiles \cite{Valtonen2009a}. It is a transmit mode system that has the transmitting electrodes placed below the floor tiles and the receiving electrodes are placed in the walls of the area. The main goal of the system is the tracking of persons on the surface. Thus the floor area should be mostly covered by electrodes to establish a good transmission link to the bodies. The receiving electrodes should be able to pick up all signals generated by the body. Valtonen et al. picked wire or plate electrodes that went from floor level to a height of 190cm that covers most typical body sizes. While the system has some shortcomings with regard to applicability in larger rooms, the design rationale is appropriate for narrow rooms or when only movement close to walls has to be detected and had a reasonable precision in their evaluation.
Looking at the above examples it becomes apparent that the proper selection of materials and geometry is highly depending of the desired application. In consequence it is difficult to give generic guidelines independent of the application. After reviewing the different application domains in the next section we will revisit this topic in section 5.4.

\subsection{Data processing}
\begin{figure}[h]
\centering
\includegraphics[width=0.4\textwidth]{images/proc_pipe}
\caption{Abstracted sensor data processing pipeline}
\label{fig:rel_proc_pipe}
\end{figure} 
%Figure 10 Abstracted sensor data processing pipeline
In order to acquire usable data from any digital sensor an analog signal has to be acquired and processed. A simplified typical processing pipeline for this is shown in Figure \ref{fig:rel_proc_pipe}. This basic structure is also applicable to the processing of capacitive proximity sensor data. The analog signal is the capacitance of an electric circuit that can be digitized using different methods, e.g. by using the quantized discharge time of the circuit. In the following section some typical steps of raw data processing and high-level processing for capacitive proximity sensors are presented and discussed. 
\subsubsection{Raw data processing}
Raw data processing of capacitive proximity sensor data is primarily intended to compensate for sensor noise and environmental influences. Noise is an inherent property of any measurement system and describes random unwanted data that is added to a signal. Environmental parameters can have strong influence on the signal of a capacitive sensor system. These effecting factors include temperature, humidity, composition of the air, or grounded objects in close proximity. There are numerous additional preprocessing steps that can be taken, such as different multiplexing methods that may be required in some hardware settings, or signal quantization that reduces the outgoing data to a distinct set of values in order to simplify post processing of different applications. These will not be further discussed in the scope of this work.
\paragraph{Noise Reduction}
In order to deal with noise, some sort of filtering is typically applied. Filtering describes a set of methods that attenuate the parts of a signal that are relevant in a given application. In capacitive proximity sensing we are dealing mostly with high-frequency noise that is added to the signal. Therefore, low-pass filtering can be used to deal with this influence. The most typical examples are average filters that take various samples and calculate an average value, and median filters that are sorting a set of samples and select the median element. Each of those filters has a plethora of potential adaptations that are not too specific to discuss in this limited space. Some adaptations are discussed in the specific prototype sections.
\begin{table}[htbp]
  \centering
  \caption{Baseline calibrations terms and methods}
    \begin{tabular}{lp{6cm}p{5cm}}
    \toprule
    \textbf{Name} & \textbf{Description} & \textbf{Application} \\
    \midrule
    \textbf{Initial calibration} & First set-up of baseline at system start, e.g. by taking the average over various samples & Required for any application \\ \addlinespace
    \textbf{Static baseline} & Baseline that does not change at run-time & For static environments \\ \addlinespace
    \textbf{Dynamic baseline} & Baseline that changes over time & For non-static environments \\ \addlinespace
    \textbf{Drift } & Change of system response to environmental factors at run-time & - \\ \addlinespace
    \textbf{Drift compensation} & Methods to account for occurring drift, by changing the baseline value & Non-static applications \\ \addlinespace
    \textbf{Recalibration} & Change of the baseline value at a specific point in time given a set of rules & Non-static applications \\
    \bottomrule
    \end{tabular}%
  \label{tab:rel_baseline}%
\end{table}%


%Table 1 Baseline calibrations terms and methods
%Name	Description	Application
%Initial calibration	First set-up of baseline at system start, e.g. by taking the average over various samples	Required for any application 
%Static baseline	Baseline that does not change at run-time	For static environments
%Dynamic baseline	Baseline that changes over time	For non-static environments
%Drift 	Change of system response to environmental factors at run-time	-
%Drift compensation	Methods to account for occurring drift, by changing the baseline value	Non-static applications
%Recalibration	Change of the baseline value at a specific point in time given a set of rules	Non-static applications

\paragraph{Baseline Calibration}
A very important aspect of capacitive raw data processing is signal calibration. The generated electric field is subject to changes over time, if either intrinsic parameters change or the environment is modified. Some specific examples include the electronic components heating up, the environmental temperature changing, or objects being moved in and out of detection range. Therefore it is essential to have a well-calibrated and adaptive baseline; that is the sensor signal generated in the environment without the presence of any object that we want to detect. Again, there are numerous methods to adapt and configure the baseline. We have collected a few common terms and methods and give some pointers regarding their application. The results are shown in Table \ref{tab:rel_baseline}. 
\begin{figure}[h]
\centering
\includegraphics[width=0.5\textwidth]{images/baseline_reset}
\caption{Example of baseline reset using a threshold rule}
\label{fig:rel_base_reset}
\end{figure} 
%Figure 11 Example of baseline reset using a threshold rule
If a dynamic baseline is used, a set of rules will have to be defined that determines at which points in time the baseline has to be recalibrated, what specific methods should be used and the set of parameters that control the methods. One simple example is to define a threshold level that triggers a baseline calibration, as shown in Figure \ref{fig:rel_base_reset}. The raw signal is above the threshold, indicating the presence of a detectable object. Afterwards, it falls back down below the threshold, yet stays for a certain time above the baseline. This triggers a reset of the baseline after a certain amount of time.
\subsubsection{High-level processing}
High-level processing assumes that we already have calibrated (and possibly normalized) sensor values that are used in further steps. The goal of any capacitive sensing application is the acquisition of information about a detectable object, e.g. its current position, the material used or the shape. In order to get this information we need to use knowledge about the object and intrinsic properties of the sensor system. In this section we will discuss methods to combine data from various sensors using the system properties, how to track the position of an object using different methods and how to recognize specific features. An overview of the methods in abridged form is given in Table \ref{tab:rel_highlevel}. 
\begin{table}[htbp]
  \centering
  \caption{Overview of high-level processing methods for capacitive proximity sensors}
    \begin{tabular}{lp{5cm}}
    \toprule
    \textbf{Name} & \textbf{Description} \\
    \midrule
    \textbf{Sensor data fusion} & Combining sensor data into a shared representational format \\ \addlinespace
    \textbf{Uniform fusion} & Sensor data fusion that combines all data into a single common format \\ \addlinespace
    \textbf{Heterogeneous fusion} & Sensor data fusion that combines groups of data to serve multiple purposes \\ \addlinespace
    \textbf{Object tracking } & Continuous identification of an object within the systems range \\ \addlinespace
    \textbf{Single object tracking} & Methods to realize object tracking for a single detectable object \\ \addlinespace
    \textbf{Multiple object tracking} & Methods to realize object tracking for multiple objects \\ \addlinespace
    \textbf{Feature recognition} & Identifying certain parameters of an object within the system range \\
    \bottomrule
    \end{tabular}%
  \label{tab:rel_highlevel}
\end{table}%

%Table 2 Overview of high-level processing methods for capacitive proximity sensors
%Name	Description
%Sensor data fusion	Combining sensor data into a shared representational format
%Uniform fusion	Sensor data fusion that combines all data into a single common format
%Heterogeneous fusion	Sensor data fusion that combines groups of data to serve multiple purposes
%Object tracking 	Continuous identification of an object within the systems range 
%Single object tracking	Methods to realize object tracking for a single detectable object
%Multiple object tracking	Methods to realize object tracking for multiple objects
%Feature recognition	Identifying certain parameters of an object within the system range

\paragraph{Sensor data fusion}
Sensor data fusion in its most general terms describes “the theory, techniques and tools which are used for combining sensor data, or data derived from sensory data, into a common representational format” \cite{mitchell2007introduction}. Using the combined information from various capacitive proximity sensors we are able to generate high-level information that exceeds the capabilities of a single sensor. We can distinguish uniform fusion that uses the information from all involved sensors in one common way or heterogeneous fusion that combines groups of involved sensors that serve multiple purposes, yet are attached to a single system. A simple example for the latter would be a single large electrode sensor that detects the presence of a hand from a farther distance and then a combination of various small electrodes that track single fingers. 
Sensor data fusion often requires taking into account some additional information we possess about the system. A classic example is the precision or bias of the sensor. Various methods, e.g. the class of Kalman filters, use weighted information from several sensor sources \cite{welch1995introduction}. If we know how that a certain sensor is only half as precise as another one working in collaborating, the weighting factors can be adapted accordingly. 

One of the most important additional information we use when fusing data of capacitive proximity sensors, is the geometric layout of the system. This describes position and size of all electrodes that are integrated. Using this information is crucial when trying to localize an object. A simple example would be applying a weighted average algorithm on a set of sensors. In order to determine object location relative to the plane a weighted average algorithm is used. The linear object location $\overline{x}$ is calculated using the sums over sensor positions $x_i$ and sensor values $v_i$ as weight:
\begin{equation}
\overline{x}=\frac{\sum^n_{i=1}{v_i x_i}}{\sum^n_{i=1}{v_i}}
\end{equation}
Using similar methods we are able to determine the location of multiple objects or additional dimensions of the position.
However, it is possible to use other information in the fusion process as well. The electrode material may result in a different response and thus should be treated differently in a fused data representation and can be weighted. Another example is the shape of the electrode that may result in different responses. How to apply sensor data fusion is strongly depending on the application and the desired common representation that is most suitable for subsequent calculations.

\paragraph{Object tracking}
In the previous section about sensor data fusion we have shortly discussed a method to determine the linear position of a single object using a linear array of capacitive proximity sensor. This is a basic example of a group of methods associated to object tracking. In computer vision applications they can be defined as “the problem of estimating a trajectory of an object in an image plane as it moves around a scene” \cite{yilmaz2006object}. The analogy to capacitive applications is viable if we consider a 3D scene and a distinct interaction space instead of a scene. 
Capacitive proximity sensors allow the detection of conductive objects within their range. However, as this presence is determined indirectly using the influence on an electric field it is not possible to get a direct association between the actual distance between sensor and object and the resulting sensor value. The created electric field is only analytically descriptive for very specific, theoretic classes of objects \cite{Baxter1996}. Nonetheless, we are able to get a relative distance measurement. If we combine this proximity value using geometric information about the electrode location we can infer the relative position of an object in the sensing area. The weighted average method presented in the previous section is one option for relative positioning. Another method is trilateration, similar to many radio-based localization applications, that uses the known location of three or more points and the known distance to the position to be determined. In case of capacitive proximity sensing this position is determined relative to the electrodes as there is no absolute distance measurement. 
A more complex example for direct calculation was presented by Smith, who formulated the issue of detecting multiple objects as a forward problem and used numerical methods to estimate the position and orientation of two hands \cite{smith1999thesis}.
\begin{figure}[h]
\centering
\includegraphics[width=0.5\textwidth]{images/prob_methods}
\caption{Generic pipeline of probability based methods of capacitive proximity sensing}
\label{fig:rel_prob_method}
\end{figure}
%Figure 12 Generic pipeline of probability based methods of capacitive proximity sensing
A second class of methods to track objects is not relying on direct geometrical calculations but instead formulates a numerical solution to a probability distribution. The initial assumption is that the probability of an object to be at a certain point in the detection area is uniform. The methods then follow a few basic steps, as shown in Figure \ref{fig:rel_prob_method}. At first the probability is updated based on the current sensor readings and a priori knowledge that we have about the system. Afterwards we try to fit the objects into the resulting probabilities. This step may or may not work, meaning that it may result in no object found. In the latter case the process will have to start at the beginning. If an object is found the probability update may use the current object location in the update algorithm, thus starting with a non-uniform probability distribution.
One example for probability-based object recognition using capacitive proximity sensors was presented by Grosse-Puppendahl et al. \cite{grosse2013swiss}. Using a model suggested by Smith the basic idea is using the assumption that an object may be present anywhere, remove regions where no objects can be present and then fit an object into the remaining space. This method additionally uses particle filters to track object locations over time. This also allows tracking multiple objects. 
Throughout the years various methods have been suggested for supporting multi-object tracking using capacitive sensors. Touch screens often use inversion of the sender signal to reliably detect the positions of multiple points; however, this method can’t be used in proximity applications \cite{wilson2007}. Some of the previously presented methods support the tracking of two or more objects. There are still various limitations, particularly if not only the object location but also various other features such as rotation should be tracked. This is still an area of ongoing research, leading to the next area of high-level processing - feature recognition.
\begin{table}[htbp]
  \centering
  \caption{Feature recognition methods}
    \begin{tabular}{lp{7cm}}
    \toprule
    \textbf{Name} & \textbf{Description} \\
    \midrule
    \textbf{Data-driven  methods} & Directly associate input data to output features using various methods, e.g. machine learning and training data \\ \addlinespace
    \textbf{Model-driven methods} & Input data is manipulating a pre-defined model of the system that is latter mapped to the output \\ \addlinespace
    \textbf{Neural networks} & Computational models using a network of neuron-like objects that are often used in machine learning \\ \addlinespace
    \textbf{Pattern recognition} & Methods that look for certain patterns in a set of input data \\ \addlinespace
    \textbf{Semantic mapping} & Methods to realize object tracking for a single detectable object \\
    \bottomrule
    \end{tabular}%
  \label{tab:rel_feature}%
\end{table}%

%Table 3 Feature recognition methods
%Name	Description
%Data-driven  methods	Directly associate input data to output features using various methods, e.g. machine learning and training data
%Model-driven methods	Input data is manipulating a pre-defined model of the system that is latter mapped to the output
%Neural networks	Computational models using a network of neuron-like objects that are often used in machine learning
%Pattern recognition	Methods that look for certain patterns in a set of input data
%Semantic mapping	Methods to realize object tracking for a single detectable object

\paragraph{Feature recognition}
Feature recognition is primarily used as a term in image processing, traditionally in computer-aided design applications to recognize specific geometric properties of an object but also picture analysis, e.g. in facial recognition \cite{han2000manufacturing,belhumeur1997eigenfaces}. 
In the domain of capacitive proximity sensing, feature recognition can be defined as the acquisition of non-location information from any detectable object. An important feature in industrial applications is the material of an object \cite{Baxter1996}. With regards to recognizing additional features a system was presented by Wimmer et al. - Thracker \cite{Wimmer2006}, a prototype augmenting a regular monitor with capacitive proximity sensors. In addition to recognizing hand position the system is able to detect grasp gestures, which can be used to select items on the screen and perform pick and drop operations. Capacitive sensors can also be used to distinguish between persons and a children’s seat on the passenger side of a car \cite{george2009seat}. 
The methods to recognize the features can be divers, ranging from typical machine learning algorithms, to model-based approaches. An incomplete list is given in Table \ref{tab:rel_feature}. In order to keep this work contained we refrain from a deeper discussion at this point.


%\clearpage 
\section{Capacitive proximity sensing applications}
\begin{figure}[h]
\centering
\includegraphics[width=0.9\textwidth]{images/theremins}
\caption{\emph{Left:} Leon Theremin playing his epnoymous electronic musical instrument \cite{Glinsky2000}. \emph{Right:} The Theremini by Moog Music Inc., released in 2014 \cite{moog2014}}
\label{fig:theremins}
\end{figure}
In the last decades of the 19th and the beginning of the 20th century a considerable number of inventors and scientists performed research on the application of electric systems, sparking innovations such as electric lighting, electric motors, telegraphy, and radio communication. Lev Sergeyevich Termen or Léon Theremin in the American naming was a Russian inventor most famous for designing the eponymous theremin. This early electronic musical instrument could be played without touch. One hand is controlling the pitch and the other the volume by changing the distance to an antenna. Initially designed as a motion detector, this device is transferring the influence of the human body on an oscillating electric field to an audible sound \cite{Glinsky2000}. Léon Theremin can be seen playing the instrument in Figure \ref{fig:theremins} on the left. This instrument is still in production to this day, with the electronic music instrument company Moog releasing a new variety that simplifies the sound production by fitting the distance to a sound on a specific musical scale \cite{moog2014}. The instrument is shown in Figure \ref{fig:theremins} on the right.

Electric field imaging was a research focus at the MIT in the 1990s. A research group in the Media Lab division including Joseph A. Paradiso, Thomas G. Zimmerman, Joshua R. Smith designed various sensing devices and evaluated various applications in in the domains of human computer interaction, smart appliances and reactive systems. They drew inspiration from biological precedents - various species of fish, such as Gymnotoidei can sense their surroundings using electric fields \cite{Smith1999a}. The changing currents created by objects with a different dielectric constant from water can be registered and thus used to avoid obstacles, even if no light source is available. Accordingly, the group named some of their prototypes after this biological precedent, including the LazyFish and School of Fish \cite{Smith1999a}. The research group created various different applications in the domains of human computer interaction, smart appliances and reactive systems.
\begin{figure}[h]
\centering
\includegraphics[width=0.9\textwidth]{images/nec_passenger_seat}
\caption{\emph{Left:} Concept view of passenger seat set to deploy or not deploy airbag. \emph{Center:} Sensor readings for empty seat and adult person. \emph{Right:} Sensor readings for front-facing child seat (FFCS) and rear-facing child seat (RFCS). \cite{Smith1999a}}
\label{fig:theremins}
\end{figure}

In collaboration with NEC a smart passenger seat was created that incorporated capacitive sensors operating in shunt mode to detect if an infant seat is currently present on the passenger seat of car \cite{Smith1999a}. The underlying challenge is that an airbag deployment should be prevented in such cases to prevent potential injuries to the infant. The seat is able to distinguish four different states, “No passenger”, “adult passenger”, “front-facing infant seat” and “rear-facing infant seat”. It uses four sending and four receiving electrodes and classifies the situation according to the current readings - the concept and readings of the sensors in the different situations are shown in Figure x.
\begin{figure}[h]
\centering
\includegraphics[width=0.9\textwidth]{images/lazmouse}
\caption{\emph{Left:} LaZmouse innards \emph{Center:} Joshua R. Smith using LaZmouse \cite{Smith1999a} \emph{Right:} Novint Falcon 3D input device \cite{novint2014}}
\label{fig:lazmouse}
\end{figure}

Another prototype is the LaZmouse that extends a regular mouse with shunt mode capacitive sensors, having one transmitting and two receiving electrodes, to measure the proximity between the heel of the hand from the mouse surface, thus allowing the fingers to remain in the common position and the mouse to be moved around \cite{Smith1999a}. Effectively this creates an input device with three degrees-of-freedom, enabling to perform interactions with a mouse that would usually require a more specialized 3D input device, such as the Novint Falcon that tracks the movement of the moved interaction sphere in three dimensions \cite{novint2014}. Figure \ref{fig:lazmouse} shows on the left, the electronics inside the mouse, the inventor using the device and a graphical representation, and on the right the Novint Falcon.

In another work Paradiso et al. presented numerous applications for capacitive proximity sensors, including smart furniture devices \cite{ Zimmerman1995}. They propose a smart table, comprised of a single transmitter electrode and two receivers that is able to track the position of a hand in two dimensions, a chosen dimension on the table and height. It may be used as gesture input device or to augment video desk applications. They also installed the system in a room, whereas the floor is a single electrode and there are four receivers located on the walls. This allows to infer the location of a person, based on relative signal strength. An additional system in this work is the smart chair, using a single transmitter in the seat and four receivers in the armrests and headrest. It allows to navigate through various audio channels based on head and arm movements \cite{ schmandt1995audiostreamer}.
\begin{figure}[h]
\centering
\includegraphics[width=0.9\textwidth]{images/related_gesture_wall}
\caption{\emph{Left:} Person interacting with the gesture wall  \emph{Right:} Air drawing results, depth estimation results and associated movements on bottom. \cite{smith1998electric}}
\label{fig:related_gesture_wall}
\end{figure}

A final prototype of this group I would like to present is the Gesture Wall, a large interactive multimedia wall, designed for public appearances \cite{smith1998electric}. A plate on the floor in front of the screen is acting as transmitter and four receiving electrodes that protruded from the edges of the projection area, as shown in Figure \ref{fig:related_gesture_wall}. It supports interactive experiences, such as drawing in the air, controlling different audio streams and an interactive video clip.
\begin{figure}[h]
\centering
\includegraphics[width=0.9\textwidth]{images/related_ctk_thracker}
\caption{\emph{Left:} Prototype of CapToolKit  \cite{Wimmer2007a} \emph{Right:} Thracker prototype and visualized grasping gestures. \cite{Wimmer2006}}
\label{fig:related_ctk_thracker}
\end{figure}

Another group that was active in capacitive proximity sensing was located at the University of Munich. Raphael Wimmer and colleagues revisited capacitive sensors in the scope of human-computer interaction. They created CapToolKit, a capacitive sensing rapid prototyping toolkit that allows interfacing eight capacitive proximity sensors and enables a quick design and testing of new applications  \cite{Wimmer2007a}. They also created Thracker - a display augmented with four capacitive proximity sensors that allows to detect the position of the hand in front of the screen and supports performed pick-and-drop gestures \cite{Wimmer2006}. Both devices are shown in Figure\ref{fig:related_ctk_thracker}.
\begin{figure}[h]
\centering
\includegraphics[width=0.9\textwidth]{images/related_handsense_tdr}
\caption{\emph{Left:} Prototype of HandSense and supported grasping types \cite{wimmer2009handsense}. \emph{Right:} Setup of time domain reflectometry sensing and example of stretchable material \cite{wimmer2011modular}.}
\label{fig:related_handsense_tdr}
\end{figure}


In other works they also presented capacitive sensing to allow discriminating the ways an interaction device is held, including distinguishing left and right hand or proximity to a body part \cite{ wimmer2009handsense}. This enables graphical interfaces to be adapted, based on user-handedness, grasping style and spatial cues that can be acquired from this device in collaboration with other sensors. Supported grasping styles and a prototype are shown in Figure \ref{fig:related_handsense_tdr}, on the left.

A final work of this group was concerned with exploring the potential of time domain reflectometry for human-computer interaction \cite{ wimmer2011modular}. This technique is sending a short electrical pulse into an electric conductor and measures the time until the signal returns. Originally intended for finding defects in long cables, such as transatlantic phone lines, high-sampling rates enable to also detect the presence of grounding objects close to much shorter conductors. Wimmer and Baudisch use an image analysis on the screen of an older reflectometer to enable applications, such as location tracking, touch detection and stretchable materials. The setup and an example of stretchable materials are shown in Figure \ref{fig:related_handsense_tdr}, on the right.

\begin{itemize}
\item Add works by Harrison/Disney Research
\item Add works by Hamburg Group
\item Add works by VTT Finland
\item Add works by TU Eindhoven
\end{itemize}



%\clearpage
\section{Sensor systems in smart environments}
In the most general definition a sensor is a device that transforms a physical property into an observable signal. This definition includes traditional systems such as mercury-based thermometers or hair-based hygrometers. Yet nowadays we are usually considering digital sensors that transfer the measured property to a binary signal that can be further processed by computing devices. 
A common variety is the smart sensor that provides additional functionality beyond generating a correct sensing signal \cite{frank2013understanding}. The main goal is to simplify installation and maintenance of distributed sensing systems by having processing close to the measurement device. Early considerations in this domain were put to the standard family IEEE 1451 - IEEE Standard for a Smart Transducer Interface for Sensors and Actuators between 1997 and 2007 \cite{ieee1451}. An additional concept is the Virtual Sensor that includes digital signal processing and conditioning and therefor abstracts the processing steps from devices interfacing the sensor. 
The number of available sensors is very high, but it is possible to restrict them based on application domain. Lewis and Cook et al. \cite{lewis2004wireless,cook2007smart} have proposed a collection for smart environments focused on wireless sensor networks. The overview is shown in table \ref{tab:sen_smart_env}.
\begin{table}[htbp]
  \centering
  \caption{Sensors for smart environments \cite{cook2007smart}}
    \begin{tabular}{rr}
    \toprule
    \textbf{Properties } & \textbf{Measurand} \\
    \midrule
    Physical properties  & Pressure, temperature, humidity, flow \\ \addlinespace
    Motion properties  & Position, velocity, angular velocity, acceleration \\ \addlinespace
    Contact properties  & Strain, force, torque, slip, vibration \\ \addlinespace
    Presence  & Tactile/contact, proximity, distance/range, motion \\ \addlinespace
    Biochemical  & Biochemical agents \\ \addlinespace
    Identification  & Personal features, RFID or personal ID \\
    \bottomrule
    \end{tabular}%

  \label{tab:sen_smart_env}%
\end{table}%
This sensor categorization is based on the property to be measured and is agnostic to the specific measurement technology. Physical properties, such as pressure, temperature, humidity and flow, can also be noted as environmental properties. They are measurements that determine the state of the smart environment, e.g. temperature in different rooms, or the current water usage. Motion properties denote the movement parameters of actors in this environment and can refer to both humans and machines. Angular velocity is important in self-localization of robots in an environment. Contact properties groups the different types of interaction between surfaces in the smart environment and actors. Presence as a group is similar to motion paramteres, but does not require a series of measurements for tracking an actor. Biochemical sensors enable measuring the presence of specifc chemical compounds in the environment and are most suited for measuring pollution or air quality. Finally, identification of actors allows to provide personalized services and can be realized with different methods ranging from tags to biometric systems.

While this listing provides a decent overview of sensing properties in smart environments it is abstracted from sensor technologies. Various types of sensors, including capacitive proximity sensors, allow us to detect multiple of these properties and thus providing a higher flexibility. Therefore it is possible to provide an inverse listing of sensor technologies that allow measuring different properties. A short overview of sensor technologies with this capabilities and that are commonly used in smart environments is given in table \ref{tab:sen_tech_prop}. In the following sections I want to give an overview on how these sensor systems are used in this domain, in order to provide a basis for the benchmarking model that will be introduced in section \ref{ch:benchmark}.
\begin{table}[htbp]
  \centering
  \caption{Sensing technologies and measured properties}
    \begin{tabular}{rr}
    \toprule
    \textbf{Technology} & \textbf{Properties} \\
    \midrule
    RGB cameras  & Motion, Presence, Identification \\ \addlinespace
    Infrared cameras & Motion, Presence, Contact \\ \addlinespace
    Ultrasound sensing & Motion, Presence, Contact, Identification \\ \addlinespace
    Microphone arrays & Motion, Presence, Contact, Identification \\ \addlinespace
    Radiofrequency sensing & Motion, Presence, Identification \\
    \bottomrule
    \end{tabular}%
  \label{tab:sen_tech_prop}%
\end{table}%

\subsection{RGB cameras}
\begin{figure}[h]
\centering
\includegraphics[width=0.4\textwidth]{images/bayer_pattern_on_sensor}
\caption{A bayer pattern on a sensor in isometric perspective \cite{img_bayer_pattern}}
\label{fig:bayer_pattern}
\end{figure}
A RGB camera is an image processing device that processes light in the visible spectrum, similar to the human eye. Modeled after the retina it has three distinct color channels - red, green and blue. There are different methods available to distinguish these channels from visible light, such as Bayer filters (Figure \ref{fig:bayer_pattern}) in front of a single sensor or using multiple sensors behind a prism.
The usage of cameras in smart environments is very common. I will present five different examples and afterwards will specify how they are linked to the properties that were defined previously.
Tabar et al. have been using a combined system of cameras, RF transmitters and wearable sensors in a home care scenario \cite{tabar2006smart}. The cameras are used to improve the accuracy of the accelerometer-based fall detection by eliminating false positives. Once a fall event occurs an algorithm tracks the posture of detected humans in the scene. They used an edge detector to distinguish the human body from other objects and applied a heuristic to differentiate lying and standing.  Additionally a face detector was used to improve the recognition of human objects. Combining this with information from the fall detecting sensor and a RF based localization system they were able to achieve a good reliability in eliminating false positive alerts.
\begin{figure}[h]
\centering
\includegraphics[width=0.9\textwidth]{images/facerec_noise}
\caption{Recognition under random corruption. (a) Top row: 30 percent of pixels are corrupted. Middle row: 50 percent corrupted. Bottom row: 70 percent corrupted. (b) Estimated errors (c) Estimated sparse coefficients (d) Reconstructed images (e) The recognition rate across the entire range of corruption for various algorithms. Newly presented (red curve) significantly outperforms others, performing almost perfectly upto 60 percent random corruption \cite{wright2009robust}}
\label{fig:facerec_noise}
\end{figure}

Pentland and Choudhury provided an overview of vision-based face recognition systems in the domain of smart environments \cite{pentland2000face}. The systems are able to identify users and recognize facial expressions. The proposed applications in smart environments include personalized shopping experiences based on customer recognition, behavior monitoring in child care facilities and emotion-aware systems that react to the user's current awareness. The described techniques include PCA-supported, eigenvector-based classification, face-based localization and systems based on local feature analysis. Newer systems are able to operate well in unconstrained environments, that include varying expression and illumination, ageing of persons, occlusion and disguise \cite{wright2009robust} (example of robustness with regard to image corruption in Figure \ref{fig:facerec_noise}).
\begin{figure}[h]
\centering
\includegraphics[width=0.7\textwidth]{images/rgb_euler_food}
\caption{\emph{Left}: Eulerian Video Magnification to attenuate the human pulse with original (a) and amplified (b) video sequence  \cite{Wu2012}. \emph{Right}: FoodBoard schematics (top), underside view (second row), original, reconstructed and segmented image (third row) and final system (bottom) \cite{pham2013foodboard}}
\label{fig:rgb_euler_food}
\end{figure}

An example for a novel image processing method that is useful in smart environments was presented by Vu et al. in 2012 \cite{Wu2012}. They are using  temporal variances of pixel values to exaggerate spatial movements and color changes that would typically be invisible to the naked eye. The method is called Eulerian Video Magnification and uses a combination of spatial decomposition and temporal filtering applied to adjacent frames. It can be tuned to different time-frequency bands to attenuate different classes of signals. Some of the proposed applications include the tracking of breathing rates of infants by attenuating chest movement, or the tracking of subtle movements, such as vibration in appliances. The example shown in Figure \ref{fig:rgb_euler_food} on the left is using a magnification of colors, in order to identify the heart rate of a person. The latter can be used for personal health applications, e.g. by integrating the system into the bath room mirror to provide an unobtrusive daily measurement and give the user feedback over a longer period of time.

A final example in this section is the FoodBoard, a smart chopping board that uses image processing to recognize the food items that are put on it \cite{pham2013foodboard}. It is shown in Figure \ref{fig:rgb_euler_food} on the right. To enable a thin footprint, ambient light is transferred to a camera using glass fibers. The picture is reconstructed and segmented, allowing to identify different items of food that are placed on it. The classification is based on a combination of Fast-Hessian and color histogram feature extractors. Pham et al. were able to distinguish 12 different ingredients with an accuracy between 59\% and 93\%. The system can be used to support dietary monitoring, give recipe guidance or support visually impaired users in identifying and tracking food.
\subsection{Infrared cameras}
Infrared imaging is using the same sensors that are suitable for visible light imaging. The difference is that they are tuned to collect electromagnetic waves of a lower wavelength that are just outside of the visible spectrum. This allows for distinct applications, such as thermal imaging, as it is possible to  detect heat radiation. In smart environments the most common application is using infrared cameras in combination with infrared light sources. This allows to illuminate spaces without visible artifacts to the user, thus providing imaging capabilities in dark rooms, or very specific conditions that may be required by a certain application. Another very interesting option is to use a specific projection of patterns into the scene. Analyzing the returning infrared light it is possible to infer the depth of specific pixels of the camera. This variety is called a depth camera. Particularly in the last few years the research in this domain has expanded strongly, sparked by the availability of an affordable depth camera/RGB camera combination - the Kinect by Microsoft \cite{zhang2012microsoft}. On the following pages we will present various examples of how this device can be used in smart environments to enable different applications in interaction and activity tracking.

\cite{beck2013immersive}

\cite{panger2012kinect}

\cite{sung2011human}

\cite{Izadi2011}
\subsection{Ultrasound sensors}
Ultrasound sensors allow detecting sound wave signals that have a frequency beyond 20kHz and are thus not audible to humans. Their propagation and reflection properties are similar to audible sound waves, thus the generated measurements can be similar. While there are natural sources of ultrasound waves the applications in smart environments do rely on active systems, that combine sound generators and sensors that measure the resulting signal. By timing the time distance between sending the signal and receiving a response it is possible to measure distances between the sender and different object. One of the earliest prototypes in Ubiquitous Computing designed by PARC was the Active Badge, an ultrasound emitter that was used to identify persons operating in the environment \cite{Weiser1991}. If various receivers are used it is possible to localize the sound source, making ultrasound sensing a popular candidate in indoor localization systems. In Figure we can see a sketch of the basic functionality of ultrasound sensing systems on the left, and an example of localization using three receivers and a single source. We will present four different examples on how ultrasound sensors are used in smart environment applications.
\cite{dutta2005utilization}

\cite{ochiai2013reflective}

\cite{watanabe2013ultrasound}

\cite{priyantha2000cricket}
\subsection{Microphone arrays}
\cite{corbishley2008breathing}

\cite{zhang2008maximum}

\cite{xu2013crowd++}

\cite{harrison2011tapsense}

\subsection{Radiofrequency sensing}
\cite{adib20133d}

\cite{wilson2010radio}

\cite{sugano2006indoor}

\cite{pu2013whole}


%\clearpage
\section{Applications in smart environments}
The field of smart environments is not strictly and conclusively limited and distinguished from other fields in technology, using influences from disciplines including electric engineering, behavioral psychology, computer science or mechanical engineering. Accordingly, it is difficult to formally list or distinguish all applications that are relevant or have been tackled in previous work. Thus, I will refer to previous collections of surveys, books and state-of-the-art in the associated disciplines smart environments, ambient intelligence and ubiquitous computing to get an informed selection of relevant applications that could potentially be supported by capacitive proximity sensors. The chosen collections of applications are taken from Cook et al. that presented a survey on recent developments in smart environments research in 2007 \cite{cook2007smart}. Augusto et al. edited a book on ambient intelligence in 2009, including chapters on various domains and applications. The final source is the book \emph{Ubiquitous Computing} by Poslad, released in 2011 \cite{poslad2011ubiquitous}.

The applications presented in those works are analyzed and grouped into a limited set of higher level applications. With this grouping I have tried to emphasize the technological similarities of the different applications in preparation for the later benchmarking process. The results can be found in Table . On the following pages the different groups will be introduced with some examples and specific information on the sensor measurements that are required. The determined groups are the basis for further references to different use cases and will be referred to throughout this work.
\subsection{Localization}
Reliably localizing and tracking multiple users is one of the main challenges of smart environments. The position of a user is important for systems that acquire contextual information in periodic intervals. For example in order to decide whether or not they are supposed to influence the actual state of their environment via available actuators. In many cases basic motion sensors are able to deliver sufficient information, if a single person should be followed. The tracking of multiple persons typically requires more sophisticated solutions. This is equally im-portant when the system needs to distinguish be-tween users and non-critical actors in the environ-ment, such as pets.
Various indoor tracking and localization ap-proaches have been proposed in conjunction with Ambient Intelligence. There are even specific com-petitions with the intention of comparing the dif-ferent methods’ performances against one another [47]. Three different categories of localization methods can be distinguished, active marker-based solutions, passive marker-based solutions, and marker-free solutions. Both active and passive marker-based solutions require a person to carry some type of tag in order to enable localization. For a number of reasons make this type of solutions less favorable. The cost is higher and some users have the tendency to forget the tags. Marker-free solutions are capable of localizing persons inde-pendently of whether they are carrying additional accessories. Examples for this latter category in-clude using microphones for the detection of subtle noises caused by movement [1], or camera-based approaches [48]. The three main criteria used to compare these localization solutions are the total provisioning costs per area, their reliability, and the amount of persons that can be tracked and distin-guished. One example system based on capacitive sensing is the previously presented TileTrack that uses a combination of transmit mode and center-of-gravity calculation between different floor tiles to calculate the position of multiple persons [28]. A second, already commercialized system is SensFloor that uses an integrated solution of capac-itive sensors and wireless communication hidden below a floor covering that is able to detect the position of several users and other parameters such as falls, based on analyzing activity above single sensor areas or the movement trajectories over time [10].

\subsection{Activity Recognition}
\subsection{Smart Appliances}
Smart appliances are devices that are attentive to their environment [49]. This is usually achieved by integrating different sensors and actuators to pro-vide additional functions and services to a user. Some examples include intelligent furniture that can detect their occupation, internet-connected household items, or single-purpose devices, e.g. providing reminder services. An overview of dif-ferent examples was created by Park et al. [50]. One recent example of smart furniture is a system for object localization using smart drawers and RFID technology presented by Nickels et al. [51]. Another example is the usage of various simple sensors that can be easily integrated into furniture to provide activity recognition [52]. Sato et al. have presented Touché, a swept-frequency capacitive sensor that allows distinguishing different types of touches on any suitable surface and medium [53]. Some examples include recognizing different hand postures in liquids and touching different body parts to control mobile devices. Their system is based on analyzing a broader range of frequencies that have a different effect on the resulting capaci-tance. Using a classification method they are able to distinguish different categories of events. An-other example of this technology is touching dif-ferent parts of a plant to control an interactive art installation  [54]. Capacitive sensing provides the ability to add interactive features to many different appliances and allows for unobtrusive placement.
\subsection{Human-Computer Interaction}
Gesture recognition enables the detection of meaningful expressions of motion by a human body, including the hands, arms, face, head and body [57]. If these expressions are translated into machine commands the result is gestural interac-tion. The most expressive and explicit form of ges-tures are performed by the hands, further distin-guished into free-air gestures and touch gestures that typically involve one or more fingers interact-ing with a surface, the latter being called multi-touch.
Apart from the capacitive touch technologies shortly presented in section 2.2, there are also nu-merous other means of realizing finger tracking on surfaces. Jeff Han showed a low-cost system based on frustrated total internal reflection (FTIR) of in-frared light. This system allows tracking ten or more objects in real-time on large surface areas [58]. Acoustic systems are another popular tech-nology in this domain. Surface acoustic wave (SAW) uses the signal decrease of ultrasonic waves as they pass through an object touching the surface to infer its location  [59].  
Throughout the years there have been various at-tempts to enable the tracking of gestures in free air. Capacitive proximity sensors have been first pre-sented almost 100 years ago by the Russian physi-cist Leon Theremin, who invented the eponymous touch-free electronic instrument [60] The theremin uses two electrodes to control pitch and volume of a generated sine wave. Capacitive hand tracking has been a research interest at MIT in the 1990s [13] and has been investigated recently by other groups, enabling touch control even through thick-er non-conductive materials [32], [61]. Various systems use optical tracking using visible light that may either rely on markers [62] or use methods based on discovering the hands without markers, e.g. by detecting the skin color [63]. 
In the last few years there have been two com-mercially successful and popular systems that track gestures using infrared light for depth sensing. The Microsoft Kinect is using an infrared projector and camera to enable whole body gesture tracking sens-ing of multiple persons at a longer distance [64]. The second device is the Leap Motion that allows fine-grained tracking of fingers and hands in a smaller area above the device. It is based on two cameras and infrared diodes that are illuminating the interaction area [65].

\subsection{Physiological Sensing}
We previously introduced bioimpedance as the response of a living organism to an external electric field [15], caused by the human body being mostly composed of ionized water. Capacitive proximity sensors can be used to measure various physiologi-cal parameters that are related to movement of dif-ferent body parts, including internal organs, most notably the heart. Cheng et al. have presented a system that allows measuring motions and shape changes of body parts using capacitive sensors em-bedded in garment [55]. They were able to detect swallowing and breathing rate. One example for an industrial application is non-contact electrocardio-gram (ECG) sensing in cars, intended to detect drowsiness in drivers. Using three electrodes it is possible to detect the heart rate or even acquire a full ECG through various layers of clothing [56]. MacLachlan presented a system that detects the respiratory rate of a person lying on a bed from a distance of up to 50cm using a single electrode and a highly sensitive sensing method based on spread spectrum methods that are commonly used in wire-less communication [12].
Generally, capacitive proximity sensing is a powerful technology that is able to gather physio-logical information over a distance, while being unobtrusively integrated into various appliances. In applications that require this information, e.g. to detect the state of alertness or fitness of a user it is a viable alternative to body-worn sensors that are more intrusive by nature.

\subsection{Augmented and Virtual Reality}




\section{Disucssion}
In this chapter I have given an introduction to electric field sensing, including different properties, physical background and the influence of materials. I have shown a number of applications built upon this technology, ranging from historic examples to research done in the last years. The field of smart environments is diverse and extensive. Thus, in order to identify the aspects relevant for this work I have introduced several multi-purpose sensor technologies that are frequently used in applications for smart environments. Additionally, I have given an overview of application domains within smart environments that are considered candidates for systems based on capacitive proximity sensors. These related works are necessary to motivate the underlying considerations of the benchmarking model that will be presented in the next chapter. It is necessary to have an extensive overview of potential applications and sensor technologies to derive a benchmarking model that successfully links sensor capabilities to features required for certain applications. 